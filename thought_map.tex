\documentclass[oneside,fontsize=13]{scrbook}

\usepackage[english]{babel}
\usepackage{geometry}
\usepackage{fontspec}
\usepackage{enumitem}
\usepackage{microtype}
\usepackage{hyperref}
\usepackage{url}

% TikZ graph drawing library doesn't work on my machine for some reason
\usepackage{tikz}
\usetikzlibrary{arrows}
\usetikzlibrary{graphs, graphs.standard, quotes}
\usetikzlibrary{graphdrawing}
\usegdlibrary{layered}
\usegdlibrary{force}
\usegdlibrary{trees}



% \setmainfont{TeX Gyre Schola}
% \setsansfont{TeX Gyre Schola}
\setmainfont{fbb}
\setsansfont{fbb}
\newfontfamily\japanesefont{ipaexg.ttf}

\setlist[itemize]{leftmargin=*}
\setlist[enumerate]{leftmargin=*}

%%%%%%%%%%%%%%%%%%%%%%
% custom commands 
%%%%%%%%%%%%%%%%%%%%%0
% command for separation of footnotes in one row
\newcommand\footnoteseparator{\textsuperscript{\bgroup\footnotesize,\egroup}}
% visual break of themes
\newcommand\pause{\bigskip\noindent}
% minor sectioning
% \newcommand\theme[1]{\pause\par\noindent\textit{#1}\medskip\ignorespaces\ignorespacesafterend\unskip\noindent\unskip}
\newcommand\theme[1]{\minisec{#1}}
\setkomafont{minisec}{\normalfont\itshape}
\def\point #1)#2\par{%
\medskip\par%
  \noindent\setbox0=\hbox{#1)~}%
  \kern-\wd0\box0%
  \textit{#2}\par\noindent%
}

% no double spaces after sentences
\frenchspacing
%twoside is for printing of actual book.
%oneside is fine for things like PDFs.
\begin{document}
%\sffamily
%uncomment above for sans-serif fonts.
\frontmatter
\title{How to Make a Complete Map of Every Thought You Think}
\author{Lion Kimbro}
\date{2003}
\maketitle

\tableofcontents

\chapter{License}
This work is licensed under the Creative Commons
Attribution-ShareAlike License. To view a copy of this license, visit
\mbox{http://creativecommons.org/licenses/by-sa/1.0/} or send a letter to...\\
Creative Commons,\\
559 Nathan Abbott Way,\\
Stanford, California 94305, \\
USA.

\chapter{Introduction}
This book is about how to make a complete map of everything you think
for as long as you like.

Whether that's good or not, I don't know -- keeping a map of all your
thoughts has a ``freezing'' effect on the mind. It takes a lot of
(albeit pleasurable) work, but produces nothing but SIGHT.

If you do the things described in this book, you will be \emph{IMMOBILIZED}
for the duration of your commitment.The immobilization will come on
gradually, but steadily. In the end, you will be incapable of going
somewhere without your cache of notes, and will always want a
pen and paper w/ you. When you do not have pen and paper, you will rely on
complex memory pegging devices, described in ``The Memory Book''. You
will NEVER BE WITHOUT RECORD, and you will ALWAYS RECORD.

YOU MAY ALSO ARTICULATE. Your thoughts will be clearer to you than
they have ever been before. You will see things you have never seen
before. When someone shows you one corner, you'll have the other 3 in
mind. This is both good and bad. It means you will have the right
information at the right time in the right place. It also means you
may have trouble shutting up. Your mileage may vary.

You will not only be immobilized in the arena of action, but you will
also be immobilized in the arena of thought. This appears to be
contradictory, but it's not really. When you are writing down your
thoughts, you are making them clear to yourself, but when you revise
your thoughts, it requires a lot of work -- you have to update old
ideas to point to new ideas. This discourages a lot of new
thinking. There is also a ``structural integrity'' to your old thoughts
that will resist change. You may actively not-think certain things,
because it would demand a lot of note keeping work. (Thus the notion
that notebooks are best applied to things that are not changing.)

For all of this immobility, this freezing, for all of these negative
effects, \emph{why on Earth} would anyone want to do this?

Because of the INCREDIBLE CLARITY that comes with it. It may feel
like, doing this, that for the first time in your life, you REALLY
have a CLEAR IDEA of what kinds of thoughts are going through your
head. You'll really understand your ideas. And you'll also see
connections that you were never consciously aware of before. You'll
see a structure and a pattern in your life. You're goals and
psychology will become clearer to you. You'll be clearer too about
what you do NOT understand.

It is like taking a microscope to your brain. You'll see the little
thoughts moving around, \emph{literally}, as you walk them through the maps
you discover within yourself.

You'll see what you care about, quite clearly. You'll be familiar with
your mental terrain. Incredible clarity. Addictive clarity. Vast
clarity. Extraordinary clarity.

You will Love it, if you are anything like me. It will feel natural
and free; There will be a freedom within your mind. You'll create
astonishing things, and you'll find great tools that will help you in
your life after you are immobilized.

Or at least, \emph{it will seem that way}.

Time will tell whether such an experience has been useful to me or
not. I still do not know, and will not know for some time now.

The experience is very much a modern version of the
``walkabout''. Except for instead of going out there somewhere in the
world, you hole up in your mind.

Is it useful? I still don't know.

Thus it is with great hesitation that I present for the public this
work on notebooks. (That is, my notebook technique.)

I want to digress and say something here as well:

I am astonished that there isn't a field of study of notebooks. I have
searched on the net, and while I have found a page here and there on
some type of notebook method, it is almost ALWAYS one of the following
two things:

\begin{description}
\item[The Diary]
   A bunch of entrees, chronologically based, maybe with a TOC, in
   which a person keeps a record of their thoughts.
   AKA ``The Journal''.

\item[The Category Bins]
   A bunch of notes, stuffed into category bins, maybe 2 or 3 levels
   deep.
\end{description}

That's IT. In all the world, people have only been putting their notes
in the above two ways.

Sure, there are a few others, but people aren't comparing notes,
talking about such things.%
\footnote{Ted Nelson in a very special case %
and deserves particular comment.  Sadly, he seems a bit unhinged, and doesn't 
write much about the topic of keeping notes openly on the web.}%
\footnoteseparator%
\footnote%
{Note 
added Later: David Allen's ``Getting Things Done'' system is
actually pretty cool. Sadly, it does not appear on the Internet. But
it's a cheap book. If you are interested in contributing to a study of
notebook systems, this is a must read.}
I would think that something like
intelligence augmentation through notebook study would be one of the
first things that people talk about on the Internet! I would think
that one of the first things we would be greeted with on the Internet
would be, ``Did you know how to use Notebooks to be smarter?'' At the
very least, it would be accessible.

Instead, there is a vast desert.

My solution to understanding this lack is my faith in what I call ``The
Anarchist Principle'': If there is something \emph{really cool}, and you
can't understand why somebody hasn't don it before, it's because \emph{you
haven't done it yourself}. That's DIY for those in the know: Do It Yourself.

Now I have a third thing I want to talk about, before getting on with
the text.%
\footnote
{First, I wrote about the things that my notebook system will do to
you, then I wrote about the dearth of notebook study on the Web, and
now I'm writing about this third thing.}

I am just SPITTING THIS TEXT OUT. I know that my understanding of
personal projects and getting them completed is low. I know my
weaknesses -- that I am bad at getting huge projects done. So what I'm
doing is just SPITTING THIS TEXT OUT.

I figure that if you are reading this, you'd much rather have this
than nothing at all. And that's what's out there, if you aren't
reading this -- NOTHING AT ALL. I mean, you can always keep a diary or a
bunch of category bins, if you like. That's a real no brainer. But
besides those two, and treatises on Ted Nelson's madness, you won't
find a whole lot.

So please excuse the poor formatting of this. It's raw, coercive,
straight text. It's unorganized. It's terrible.

Maybe one day I will improve this. But that day is not today. Today is
a day for spitting text out. With God's mercy, I will learn how to
finish big projects. I pray for that ability frequently. If you can
mentor me in the subject, I will happily hear you out. But I have not
learned it yet.%
\footnote{I actually believe that we should all communicating with
what Robert Horn calls ``Visual Verbal Language''. The problem is that
we don't have good tools to do so. Damn. It's a shame. When the tools
come, there will be a revolution in communication just as big, if not
bigger, than the revolutionary introduction of the Internet. So:
Double Apologies for swabbing a mass of text at you.}

Let's see: We've talked about:
\begin{enumerate}
\item The Terrible Things this notebook system will Do to you.
\item The Nonexistence of an Internet field of Notebook study.
\item How I am just spitting this text out into the world.
\end{enumerate}

Lastly, I want to briefly introduce some of the unique features of my
notebook system. Things that my notebook system does that NO OTHER
NOTEBOOK SYSTEM that I have ever seen does. These are the results of
years of keeping different types of notebook systems, and taking the
best ideas from each.

\point 1) Strategy.

This notebook system allows you to STRATEGIZE. Very few notebooks do
that. I mean, sure, you can start some pages on, ``what will my
strategy be now?'', but then you have to figure out what all your
options are. The notebook system I describe has built in strategy
management. You will always know what your options and priorities are
in notebook management.

It does this with the aid of maps...

\point 2) Maps.

Tables of Contents (TOCs) are TERRIBLE. TERRIBLE TERRIBLE
TERRIBLE. Their main utility is in providing you the next number to
number something.

You want MAPS of Contents, or what I call ``MOCs''. This is like a table
of contents, but far more dynamic. Is an entry really important? Make
it's MOC entry really big. Unimportant? Make it small. Or even pull
out the name, and just surround the page number in parenthesis. If you
are ever investigating that area, you can look it up to see what it
is.

You can put related concepts close to each other, REGARDLESS of the
actual physical position of the pages.

You can move things around. Trace paths of connection through. Make
non-ordinal order apparent. All with Maps.

So don't keep a TOC, unless the material is intrinsically linear (a
chronology, without episode tracking.) Keep a MOC!

More in the text on this subject. This is just a brief introduction
for the sake of introducing some major concepts from my notebook
system.

\point 3) Color and Glyph Management

Abbreviations/Shorthand, and Color. You MUST use a 4color pen. (I
mean, you don't NEED one, but it's so amazingly helpful, that once you
start using it, you'll NEVER want to go back.)

You'll develop shorthands, and you'll want to trust that you can
decipher them later. You'll thus need systems for cataloging your
shorthands, and there is such a system in the notebook system I am
describing.

\point 4) Speed Lists

You will be capturing EVERY SINGLE THOUGHT. Well, every thought that
is more interesting than ``I need to go to the bathroom'', or ''I need to
take the trash out''.

(Actually, needing to take the trash out may enter, as per the Getting
Things Done system. SHOULD you decide to integrate my system with the
GTD system. More later on the subject.)

Speed lists are the answer to the demand. Speed lists are vast lists
of simple thoughts -- about 1-50 words. Generally around one line.

There are two types of speed lists -- pan-subject and
subject. Pan-subject speed lists are for all thoughts, you take a
pan-subj speed list out with you to work or to wherever you are
going. Subject lists you keep in a cache notebook, and you have one
per subject. You'd prefer to just use your subject lists, but
sometimes you have to make do with a pan-subject speed list, and then
transfer out from there to the individual speed lists. More on all of
this later.

In particular: You do NOT want to be just scribbling thoughts out on a
piece of any old paper. You want at least a pan-subj speed, with a few
exceptions. (If you are having a thought-attack, you may need to make
do w/ just any old piece of paper, and then go from there to the
subject pages.)

\pause

Okay: I'm getting too much into details.

So four advantages are:
\begin{itemize}
\item Strategy
\item Mapping
\item Color/Glyph
\item Speeds
\end{itemize}

They work together marvelously. In particular, Strategy, Mapping, and
Speeds all directly affect and rely on one another.

So there we have it.

\begin{enumerate}
\item What this system will do to you. (EXPECT IT.)
\item The nonexistence (to date!) of an Internet study of Notebooks.
\item How I am spitting this text out.
\item The advantages of my system.
\end{enumerate}

The Introduction is over.

\chapter{System Overview}
I will be talking about the following things:

\begin{itemize}
\item Intra-Subject Architecture
\item Extra-Subject Architecture
\item Materials
\item The Question of Computers  -- don't get all excited now.
\item Theory of Notebooks
\item General Principles
\end{itemize}

Let me talk at a high level about what this is all about.


The notebook system roughly divides all of your thoughts into
``Subjects''. What subjects? Depends on your thought patterns. In the
Subjects section, we'll talk about how to divide your thoughts amongst
subjects.

Now, there are two domains: ``Extra-subject'' and ``Intra-subject'': that
is, outside and inside of your subjects.

\emph{Intra-subject}: Within a particular subject, you'll have an
organization. You'll have your speed thoughts for that subject, you'll
have your maps, you'll have your big dissertations (``Point of
Interest'', or ``POI''), you'll have your cheat sheets, your
abbreviations/shorthands (``A/S'') particular to that subject, all sorts
of wonderful things. Most important though, are your speeds and your
maps.


Now, beyond the subject, there is a whole field of all your
subjects. You'll have the ``GSMOC'' -- the ``Grand Subject Map of
Contents'', whereon you'll see a gigantic map of EVERY THING THAT YOU
THINK ABOUT. Just imagine that right now: \emph{Wouldn't you be interested
in seeing such a thing?} When I think about my GSMOC, I see a mirror
of my mind, for the 3-5 months that I kept my notebooks. (The borders
are fuzzy, because I gradually evolved into the notebook system I am
describing to you.)

I mean, that, right there, is worth the price of admission. The GSMOC
is a pretty impressive thing. \verb| =^_^= |

Okay. So there are steps and promises that apply beyond the field of a
single subject, and there are steps and promises that apply within the
field of a single subject.

Now extra-subject and intra-subject \emph{float on top of} your
\emph{MATERIALS}. We're talking about pen and paper and your binders. And
some other things: You'll need those little donut holes things to
protect your paper, and you'll need little stickies to put onto your
paper, your maps. This will help with strategy and other map
management functions. So I have a section on materials and all that
stuff. Great stuff. What to look for in picking a binder. Wonderful.

\pause

There: We've knocked off the first three:
\begin{itemize}
\item Intra-subj
\item Extra-subj
\item Materials
\end{itemize}


\pause
Three more to go:
\begin{itemize}
\item The ? of Computers
\item Theory of Notebooks
\item General Principles
\end{itemize}

Okay, I'll take the ? of Computers last. General principles first,
then Theory, then ?'s of Computers.

\theme{General Principles}
There are many patterns common in the steps and promises of the
notebook. Things such as ``How do I lay out a page?'', the concept of
``Late Binding'' and how it applies to the notebooks. ``Out cards.'' The
use of color. Partitioning strategy. Writing
quality. Psychology. General mapping principles. Important stuff, but
not specific to a particular position in the hierarchy.

\theme{Theory of Notebooks}
Why use notebooks at all? (Partly talked about in the introduction.)
How does this work? Observations about how subjects gestate. How
information flows, becomes knowledge, then becomes wisdom as it
integrates into our life. How thoughts integrate. How the speeds
grow. And a theory of (conscious) thinking. Many things to talk about.

Finally, the \emph{Question of Computers}. My least favorite subject, because
people can get so damn irrational about computers.

I don't know HOW many times I've seen people twiddling about with
their little palm pilots, convinced that because they have
``technology'' on ``their side'', that they are being more effective than
a man holding a piece of paper and a pen. The absurdity of these
devices is astounding.

I know that there are legitimate uses for these things. I see doctors
carrying them around with up-to-date info dictionaries and what not, I
know that they use them, yadda yadda yadda. And yet the simple fact
is, 99\% of the general public using these devices have no need for
them. They'd be much better served with a small pad of paper that they
keep with them, and a pen.

There are exceptions to this: You can argue a good case for using them
to play games (though I'd rather use a Game Boy Advance), or for using
them to use as an address book. Great. I love it.

But for the Love Of God, if you live within the time period of
2003-2005 at the very least, do NOT try to use one of these devices to
keep your notes!

This extends further to computers.

Now: All this will change. IN THE FUTURE, computers will be the way to
go. But we are not there yet, NOR will we be there in the next 3-5
years. Remember: Even if the computer is fast, you still need software
that won't get in your way.

I will address this subject again, later in the book. Feel free to
skip it, if you plan to use paper. But if you are one of the ``I paid
big money for this thing, and it's high tech, and it's sooo cyber,
that it must be better than anything pen and paper can give me,''
please give serious thought to what I have to say.

I'm positioning it later in the book, so that you can have already
have read about maps. I mean, maps right there -- these little devices,
and even my big computer, doesn't get maps right. But you'll see how
this works as we read -- no need for me to go off the deep end now.

\pause
\noindent SO.  In Summary:

\begin{itemize}
\item Intra-subj
\item Extra-subj
\item Materials
\item The ? of Computers
\item Theory of Notebooks
\item General Principles
\end{itemize}

That's what I'm going to be talking about now.

Here, let's put that in order:

\begin{enumerate}
\item Materials
\item General Principles
\item Intra-subj
\item Extra-subj
\item Theory of Notebooks
\item The Question of Computers
\end{enumerate}

By the way -- in case I hurt your feelings about computers -- I want to
add two things:

\begin{enumerate}
\item 
I am an experienced programmer. I've been programming computers
since I was 7 years old, typing in BASICA programs by hand on my
mom's COMPAQ 8088. I formatted her hard drive by accidentally going
into low level format instructions using ``debug'', experimenting
with assembly language, when I was about 10. I am now 25. I love
computers. I just happen to recognize the limitations of where
we are at right now, that's all.
\item Computers will be the SALVATION of this whole system I am
describing right now. So if you feel offended knowing that I am
dumping on them right now, know that that's not going to be the
case for long. Paper is unwieldy, large, requires storage, and a
host of other ills. Copying from page to page to page is just a
nightmare. It is a necessary nightmare, right now, but it is a
nightmare. Computers will save us from it.
\item Hah! I'm sneaking in a \#3. (Part of my ``no-edit policy'' when
spitting stuff out. Apologies.) I WILL DESCRIBE, if I DO
NOT FORGET, just WHAT steps you can take NOW, IF YOU ARE
INTERESTED, to ``get the ball rolling''. There are some easy
programs that you can make right now that would make this system
AWESOME. I just don't have the time to code them up right now. But
I will describe them, and if you like, you can code them up.
Hell, I'll even throw in a description of the ideal computer
notebook system -- assuming I have ``magic paper'' -- and how it will
dramatically increase our intelligence, provided that we
can solve the versioning problem as well. (Note: Ted Nelson and
Company went pretty batty WRT the versioning problem. Did
they solve it? I don't know. I have heard rumors that some of
Ted's protege's work for the CIA now, though.)
\end{enumerate}

Speaking of the CIA -- I want to include in this book somewhere (right
here I guess) a comparison between this notebook system, and an
Intelligence Agency. INTELLIGENCE is having Good Information available
at the Right Time at the Right Place. Notebooks help with that by
moving information to a place where it will be seen at the right time-
when you access the notebook. There are a lot of similarities there
with an Intelligence Agency. Okay. I'm done. Interlude out.\footnote{Yes, I do
recognize the irony. I'm just spitting this out with poor
organization. But damn it, I'm not a skilled writer, and I have SO MUCH
I want to express, so you'll just have to make do for now. Sorry.}

\pause
Where were we?

\begin{enumerate}
\item Materials
\item General Principles
\item Intra-subj
\item Extra-subj
\item Theory of Notebooks
\item The Question of Computers
\item Getting Started
\end{enumerate}

Okay. And be aware I'll probably need to skip back and forth a little
bit. Sorry, just one of the problems of having a straight linear text,
rather than a fully mapped out domain.

\mainmatter
\chapter{Materials}

Some topics for ``Materials'':

\begin{itemize}
\item Paper
\item Pen
\item Binders
\item 3-hole punch
\item donut rings
\item stickies (NOT yellow sticky tabs!)
\item tab dividers
\item pockets
\end{itemize}

And associated issues:

\begin{itemize}
\item Storage
\item Carrying
\item Archival
\item Handling Optimizations
\end{itemize}

So lets start with the materials -- what you need to have with you.

\theme{Pen}

You need a pen. Actually, you need three. And they need to have little
four color clippies -- Red, Green, Blue, and Black.

Theoretically, you can do this all with a black pen, but TRUST ME, you
don't want it. Your ability to very rapidly switch colors will way
more than make up for the nicer line that the G2 gel pens give
you. Really.

You need one to carry with you, you need one for backup, placed in a
trusted place, and you need one to be a backup to the backup. YES, you
really need this. If you are wasting time looking for a pen that you
lost, you are just wasting time. The pen will come back. In the mean
time, you need to write, so you've got to fetch your backup. You have
a backup to the backup. If you have ready access to a store, you need
to buy another pen, should you not find your first pen by then.

These 4-color pens are expensive. Remember: Buy 3. \emph{Your pen is your 
life  -- don't lose it}.  But when you do, don't hesitate to start in with
the backup.

Next: You want to have a list in your notes of the locations to search
for your pen. Mine looked like this:

\begin{enumerate}
\item Jacket 
\item Pockets 
\item Pants 
\item Pockets 
\item Buried 
\item Inside Notebooks.
\end{enumerate}

Re: the last: ``Buried Inside Notebooks.'' IF YOU DO THIS SYSTEM, that
will actually be a VERY common occurrence. Because you'll have 2-3
inches of paper. Those 4-color pens are BIIIIG and FAT. But they
aren't so big that they can't get completely lost amidst a big fat
chunk of paper. Trust me. So actually open up the book and flip
through sections,looking for your pen.

I'm not going to talk about this much; This is just something you'll
find with experience.

So that's the deal with the pen. I'll talk more about what the colors
are for in the ``General Principles'' section.

Next.

\theme{Paper}

You want lots of it. Always have at least 2 reams unopened, of about
150 sheets each.

Get COLLEGE RULE. You want as many lines on these as you can, because
information density is the name of the game. 3 holes, of course, so
it'll go in your binder.

8 1/2" x 11", or the new 8" x 10 1/2"?

Don't laugh -- it's a serious question.

There are trade offs to both.

I used 8``x10 1/2'' for most of my notes. It was good because they fit
within the larger tab dividers. Yeah. 8x10.5 is also a lot
cheaper. With the volume of paper that you will purchase, price can
become an issue.

But if I were to do this again (and I intend to -- I intend to do this
once, for three months, once every 3-5 years, to gain a ``situation
awareness'' I would use full 8.5x11".

Why? It's not really the ``extra bit of page'' that is important (it
isn't -- having a better rule is far more important), but rather that
your paper conforms to the global standard for paper.

You are invariably going to want to include leafs from outside your
notebook system. And you should eventually make your own templated
papers: You'll make standard form sheets, print them onto printer
paper, and include it in your notebook.

Printer paper doesn't come in 8.5``x11''. So you have some big pages and
some little pages. Yuck! When it comes to quickly flipping through
pages to find a particular page number -- yuck! It gets difficult.

So get 8.5x11" college ruled 3-hole-punch paper.

\theme{Binders}

Major important.

First, let me dispel notebooks:
Don't use them. I'm talking about spiral bound notebooks.

I used to use them. I have a huge collection of spiral bound notebooks
in my closet. I love them, they are so cute and self-contained. And
partitioning them is kind of fun, even.

But the binder system just so completely blows them out of the water,
that I will just never go back to those things.

This isn't to say that notebooks don't have a place -- THEY DO! Just not
in this system.

Notebooks are great when you are doing a straight chronology. Or you
are keeping JUST RECORDS. Not a big fat intricate
total-thought-keeping system that I am describing here, but rather,
I'm talking about -- you have a business, and you are keeping records
for it, and so you buy a notebook because it's nice and self contained
and stuff like that. Another nice thing is that you know the pages
aren't going anywhere. There are times where that's not what you
want. And you can turn pages easier. It's just easier.

But this system that I am describing:

Impossible. You cannot do it like that.

In this system I am describing, you MUST be able to insert pages
between pages. And it's so incredibly useful to be able to lift 50
sheets and put them in another binder entirely.

Okay, so, please don't use notebooks. You will die. Quickly.

\pause

Now, on to \emph{Binders}.

What you want to look for:

\begin{itemize}
\item Inside Pockets
\item Transparent Outside Pockets
\item Obstructions on the Outside Spine
\item Sheet lifters
\item Ring Type
\item Width/Size
\item Durable vs. Sucky
\end{itemize}

So, let me start with the last one. I forget what they call the
``non-durable'' ones. They cost less. Maybe ``Economy'' or something like
that. DON'T GET THEM! YES, they are CHEAPER. BUT, even on the budget
that I'm on, you do NOT want them. Because they are going to snap open
when they shouldn't. Believe me, there's nothing worse than being on
the bus, hitting the notebook the wrong way, and suddenly WHAM -- 100
pages on the floor. Luckily they are numbered and you can put them all
back in the original order, but-

\emph{Trust} me -- Go with Durable.

You'll have to unchink both sides to open the ring, but you'll do so
with the knowledge that it's keeping your data safe!

DURABLE! All the way!

Okay, next, we'll talk about width/size and the ring type.

If you are getting, say, a 1``-1.5'' notebook (my carry-about notebook is
somewhere in there), then just get the normal rings. They are three
loops, bound to a metal binding, blah blah blah.

But if you are getting anything larger (and you should have at least
one of these, for your common store -- it's going to be BIG), then you
want to get what I call a ``half-loop''. I'm sure there's formal names
for this stuff, but I don't care. These things look like one half is a
loop (as normal), but the other half is straight, and has a 90 degree
crook at the end. ALSO, it's not attached to the binding of the
binder..! It's attached to THE BACK SIDE of the binder.

These things are SO great. It costs more, but GET IT.


What it does is it keeps your papers from flying out all over the
place when you open your deeply packed notebook. That little 90 degree
crook stops the pages. It's great. You'll have to see it to believe
it, but do. It's wonderful.

So: Big Notebook, get the half-loop. Small notebook, I think they are
all just normal full-loops. Never seen a small notebook with a half
loop.

Sheet lifters. If your binder has a sheet lifter, Awesome. I like
these. I'm not sure why. They just seem to help. This is more of a
spiritual belief on my part; I'm not really sure. But I leave them
there and they seem to be useful.

Now I'll talk about inside and outside pockets, and then the possible
obstructions on the outer spine. Then we'll be done talking about
binders. (It's a fetish thing, I guess.)

The inside pockets are really useful. I use them to store tabs in when
they aren't in use.

Oh -- REMEMBER that you (if you could) bought those pockets, right?
Stick one at the front of every binder. Store donut holes and stickies
in them. That's just the place to do it. And you'll store the tabs'
``guts'' in there too. You know, these long sheets of 1" wide paper,
perforated at about 1/6" in height. You write whatever the tab's name
is on them, tear them off, and put them in the tab page, right? And
then you put the tab page into your notes, and you can quickly flip to
them. Tab page guts -- you know what I'm talking about, right?

Good. (One day, I may, or someone may, put pictures in this
description. Then those of you who don't get ``tab guts'' can see what I
mean.)

In the inside pockets, you'll store larger things like your tab pages
themselves. And when people give you stuff, and of course they didn't
triple hole punch it, you'll put it in there until you get home and
punch it yourself.

Outside pockets. This is really important.

You're going to identify your notebooks quickly by the outside
pockets. You can get away with not doing this, but it's a pain in the
butt. Pay the extra money (this is becoming a theme, is it not? trust
me, I'm not rich, if you haven't picked up by reading this yet -- [HEY,
I'm a PROGRAMMER, and it's the year 2003] but pay the extra money
nonetheless) and get the pockets.

Here's what I do with them:

For archive notebooks, I put the letters that are archived. For
example, ``A-M'' and ``N-Z''.

My common-access notebook (a big fat one) doesn't have covers. I think
that's because I got it for free at a college giveaway, and wasn't
being picky. No matter, it is jet black, and none of the others are,
so I can easily identify it.

My carry-about notebook has, ``default'', two pictures of Lions on
it. My name is Lion, so I put Lions in there, and people are able to
put it together that it's mine. The Lions are smiling, and it
communicates something of my nature to people. I think.

But usually the ``default'' isn't there. I keep a variant of the GTD
system running (``Getting Things Done'' by David Allen), and so I
generally have my day's alerts, options, and chores on the very
front. (Not that this is strictly defined in GTD, but I've adapted it
a bit.)

And I usually have on the back cover, covering a Lion, a general plan
for how my day will go out and bus trips (http://transit.metrokc.gov)
for the day. It is very, very useful.

Finally, you want to look at the spine, if you have outside pockets,
and make sure it is not obstructed. Frequently there are three ``bolts''
on the outer spine, and they sometimes pass it through the transparent
pocket on the spine. NOOOOOOOO! We don't want that!

That means you can't stick an identifying paper back there! Or if you
can, you can only dig it in half an inch. No, that's not for you! You
want to be able to put a paper in there that has the name of the
binder on it, so you can quickly ID it when a bunch of binders are
stacked in a row.

There. I am done dissecting binders. If I omitted something, mail me
at \url{lion@speakeasy.org}

\theme{3-hole punch}

Again -- you're going to want to print out sheets, and then include
them. Or you're going to want to include things that people give
you. Very well then, you're going to need to x3 hole punch it. It's a
wonderful tool to have, and it will go a long way. I absolutely adore
mine.

\pause 

Lets go through these small items quickly:

% \begin{itemize}
\theme{Donut Rings}

I don't know what the official name is for these things. They are
flat, round, have a hole in the middle, and they reinforce paper.

When you have a lot of papers in your notebook, they will eventually
start to rip at the holes. The rip will grow, and grow, and the next
thing you know, your paper doesn't stay inside your notebook. The
solution is to, when one hole tears, immediately reinforce all three
with these donut rings. I don't know if you need to, but to be safe, I
put 6 O-rings to a page. Three on the front of the holes, and three on
the back.

I've never had a problem since. I've never seen a donut tear.

\theme{Stickies}

Okay -- these are NOT yellow sticky tabs!

What these are, are these little tiny stickers that look like small
rectangles. They are about .5" wide, if that. You can stick and
unstick and restick them to paper, AND THE PAPER DOES NOT TEAR OR DROP
INK AS YOU DO SO.

These are AMAZINGLY useful.

You will use these extensively as you STRATEGIZE over your notebook.

A brief explanation for now:

Strategy is ultra-time-sensitive. It also involves a lot of
prioritizing, and the priorities will change -- rapidly.

You don't want to mix up your rapid-change stuff with your low-change
stuff. That is, you don't want permanent marks on your pages for
things that are changing rapidly. So you use these stickies.

On your GSMOC (Grand Subject Map of Contents), you'll have stickies
pointing you to major important areas of work or thought. You'll take
them off when they cease to be important, or when you fulfill
them. The same goes for the subject maps within each subject.

That's basically it. Small idea, but EXTREMELY useful. I'll write more
about it when it comes time to talk about it.

\theme{Tab dividers}

You will use these to keep your subjects apart, and a few other
things.

\theme{pockets}

Now I REALLY don't know what these things are called. My girlfriend
got them for me by stealing a few from work. When I saw them, I
understood why.

These are little pockets, that you can stick ANYWHERE. They have a
plastic white back, and a transparent front. The back and front form
the pocket, which opens from above, and is sealed around the edge. But
the back ALSO has a sticky thing. You peel off a layer, and you can
stick the whole pocket ANYWHERE. This is VERY useful.

I use the pocket to store the following things:

\begin{itemize}
\item my donut holes
\item my stickies
\item stamps (as in, postal stamps)
\end{itemize}

It has worked like a charm.

\pause

So in recap, your shopping list is:

\begin{itemize}
\item paper  -- get about 8 reams, college rule, 8.5x11", to start with.
\item pen  -- 3 four-color pens.
\item x3 hole punch  -- get it at a thrift store if you want it cheap.
\item donut holes  -- get 1-2 packages of many sheets.
\item stickies  -- get 1-2 packages of many sheets.
\item tab dividers  -- get about... 50 tabs. To start with.
\item pockets  -- if you can find 'em, get at least 4 or 5.
\end{itemize}

Now we talk about transport issues:

\begin{itemize}
\item Storage
\item Carrying
\item Archival
\item Handling Optimizations
\end{itemize}

Storage, Carrying, and Archival will be one big topic here. It's all
intermingled.

SO.

You are keeping notes. You have papers. Here is a sort of scale of
your papers:

\begin{enumerate}
\item scrawled notes on fortune cookie papers, backs of napkins, etc.,.
\item scrawled notes on blank paper.
\item notes collected onto pan-subject speed lists.
\item notes collected in your carry-about binder.
\item notes collected in your common-store binder.
\item notes in the archive.
\end{enumerate}

There is another category, hovering around 5.5: special purpose
collections. For example, I have a binder for ``Computers''. In it, it
has subjects such as ``Networking'', ``Debian'', ``Programming'',
``Software'', ``XSLT'', etc., etc.,.

I should mention there is also item ZERO:

\point 0) stored in your mind on a peg list.

I'll talk about that later. If I forget to, mail me, and let me know
that I forgot: lion@speakeasy.org. Yes, I know that I could keep a
list of promises to keep here in my Emacs buffer. But to be frank,
after having been keeping so many lists for so many months, that I
really just don't feel like making one. Pardon my rudeness, but if you
actually DO what I am describing here, you'll understand what I'm
talking about. Back to the subject at hand.

So let me describe each of these sources.

\theme{Scrawled notes on fortune cookie papers}

Some times, you just flat out DON'T have your carry-about binder
with you. And you don't have your pan-subject speeds paper. And you
don't even have a blank paper. And your peg list is full, or you don't
feel like cycling it.

So you just have to make do with what you have.

You put a note on the back of the envelope and stuff it in your
pocket. Or you take that fortune cookie slip out and write on it. Or
whatever.

GREAT!

I mean, it sucks. But at least you got that thought! Good for you.

\theme{Scrawled notes on blank paper}

Or maybe you have a blank piece of paper in reach. Write the thought
of it, and put it in your pocket.

\theme{Pan-subject speeds}

But if you can, be prepared in the morning, and put a pan-subject
speeds page in your pocket.

I'll talk much more about speeds in my exposition on ``Intra-Subject
Architecture'', but a little bit should appear here.

The Pan-subjects speeds page is optimized to have graduate-student
rule. This is beyond College rule. You want 40+ lines on a
pan-subjects speed page to cram thoughts into. Again: DENSITY is the
name of the game.

Furthermore, the pan-subject speeds is partitioned. It has:

\begin{itemize}
\item Transcription Check-off
\item Subject
\item Hint
\item Content
\end{itemize}

You can put whatever you want in there. Mine also has a place for a
``Psi'' marker. That's where you list what type of thought it is, in
terms of ``Principle'' or ``Observation'' or ``Warning'' or ``Possible
Action'' or ``Goal'' or ``Problem'' or ``Starting Point'' or a host of other
glyphs. I'm not going to talk about these because they are beyond the
scope of Notebooks. They go more into mental techniques; Has to do
with mental structure and the anatomy of thought. While related and
quite fascinating, I'm just not going to go there. Whole 'nother
discussion for a whole 'nother day.

The point is, the format is malleable. Include whatever you want. I
also have a date marker at the top of the page, for the Chrono
archives. Whatever you want.

DO NOT PUT THOUGHT NUMBERS on the Pan-Subject Speeds page though. Bad
idea. The purpose of the Pan-Subj speeds is to be a TEMPORARY
placeholder for ideas.

So what are these four things:

Transcription Check-off: You check the box after you have moved the
idea OUT to where it needs to go. Don't check it when you first put
the thought in.

Subject: This will tell what subject the thought will go
into. Remember: The subjects are the big things divided by the tab
delimiters that have their whole own infrastructure on their own, that
I will describe later.

Hint: Now, this is a quick 1-2 word, maybe 3 word, description of how
this thought fits into things.

Something I learned late, but that is very important, and very
essential to this whole process, is that:

\begin{quotation}
\noindent When a new thought appears\\
It doesn't do so in a vacuum\\
It does so in a context.\\
\end{quotation}

Words to the wise.

So the ``hint'' describes the context. This is
VERY IMPORTANT!

The context is fresh in your mind when you get the thought! It would
take a while to recognize the thought, and then identify the context,
if you didn't.

I used to try to think of every context a thought could fit in, and
then try to place it in as many places as I could. WHILE THIS IS THE
STRATEGY TO PURSUE WHEN USING A COMPUTER SYSTEM\footnote{See
http://speakeasy.org/~lion/weird.html to see an example of this}, this
is NOT the strategy to pursue in the paper system!

Besides, the thought is MOST useful in the ORIGINAL context, 95\% of
the time.

And your hint -- that's going to be USED. In some respects, it's EVEN
MORE IMPORTANT THAN THE THOUGHT ITSELF! Because, as you will see if
you do this for a while, it is STRUCTURE and INTEGRATION that is
important -- the actual contents of the thoughts are far less
meaningful. Once you have the structure in front of you, the content
because almost obvious..! We'll use that fact in a bit, as we shorten
titles to just Speed Numbers. Don't worry about that now, though.

And then there's the content of the thought itself.

Now, say you're in a hurry -- right? You just want to jot down a
thought. You're running medical records, and you can't carry your
carry-binder with you as you do so. Hey, there are limitations in
life. But you were good, and folded up a pan-subj speeds with you to
carry around. You unfold it, and write down the content of the
thought, greatly abbreviated, into an open ``content'' slot.

Do you have to fill in the hint as well? And the subject?

\pause
NO!

\pause

Just wait for break. In break, you can flesh out the content if you
like, and you can also fill in the subject and the hint. It won't be
hard. Just don't wait a whole day to get to it -- do it SOON.

Now focus your thoughts on the very next work step, because you want
to STOP thinking ASAP.

Note that the pan-subj speeds paper is FAR better than a blank piece
of paper, because it provides order and space to fill in. Believe me:
When you start transcribing off the pan-subject speeds to the speed
pages, you'll understand how useful this is.


% NEXT:

\theme{Notes collected in your carry-about binder}

Your carry-about binder will be YOUR BEST FRIEND.

That's right: You are going to carry this EVERY PLACE THAT YOU
CAN. Going to the movies? Riding the bus?

Wherever you go, your carry-about binder is going with you.

Thus you will want to be very particular, even religious, about your
carry-about binder.

(Note: As mentioned, there will be times where you will be ripped
apart from your carry-about binder by force of circumstance. If you
can, bring a pan-subject speed with you. Always keep your carry-about
well stocked with pan-subj speeds so that when you depart, you can
carry a catch away with you.)

A ``Catch'': ``Catch'' is a word I use to describe any device that is used
to keep thoughts as they come.

There are two basic types of thinking: Intentional and
Incidental. Intentional is you sitting down, thinking some issue
out. You'll be doing that, mostly amidst POI's (``Point-of-Interest
Pages''). But most of your thoughts will come while you are
out-and-about. So you'll have to catch them. There are various traps,
called ``catches'', that do this. The speed lists are the first good
line of defense. You have some poor ones to: the aforementioned
napkins and fortune cookie slips and envelopes, and blank pages. You
also have the peg's\footnote{``Tie Noah Ma Rye Law Shoe Cow Ivy Bee Dice Tit
Ton Tomb Tire Towel Dish Tack Dove Tub Nose...''  -- yes, I chose
Dice-Tit-Ton, I know... Though Toes-Tot-Tin were harder to work with.}
But those require a lot of processing and rotation and can get pretty
funky when overused.

\pause

Back to topic.

Your carry-about binder. It shouldn't be too thick -- not more than
1.5". But it shouldn't be one of those thin things either -- you're
going to be carrying A LOT of STUFF in there.

What kind of stuff is going to be in there?

You're going to have:
\begin{description}
\item[Blank Paper] Say, 30-50 sheets.
\item[Blank form paper] such as pan-subj speeds, blank speeds, map pages,
and whatever other templated form paper you invent.
\item[A Zillion Speeds and References] Speed lists for your myriad
subjects. Some will be accompanied by references lists.
\item[Other Stuff] (I personally keep a lot of GTD related materials in
there.)
\item[Perhaps a Subject] Sometimes you carry a subject around from the
common store, because you are processing it, or adding new content to it.
\end{description}

You won't be carrying ALL of your subject speeds in there -- only the
speeds that you are still filling out. For example, if you have 140
speeds in a subject, and the last page of speeds starts with S127,
then the last page, with S127-S140, is the only speed page that will
be in your carry-about binder. The rest are back wherever the subject
is presently residing (probably the common-store binder, or maybe in
the archives).

All of your references go there, however, because you want to be able
to give people references quickly. When you talk with people about a
subject, show them your list of references, so that you can recommend
good references to them.

\theme{Notes collected in your common-store binder }

Particularly, the notes will be organized into subjects. You'll also
have a place called ``CHAOS'' (which will be quickly dumped to archive,
because it is nearly the most useless thing you will have, though very
occasionally used), and a place called ``UNPLACED'', for pages that are
important but haven't been placed, and for pages that would be placed,
but that they aren't numerous enough to warrant a full-on
subject. (You'll indicate the subject that they WOULD belong to at the
top. Organize A-Z by would-be subject. When reach about 5-10 pages,
make a full-on subject for them, with all that entails.)

But mostly, the common-store is just the subjects you have been using
lately -- say the last 20-40 days.

Occasionally, you'll go through the common-store, and take subjects
out that you haven't touched lately, and put them into...

\theme{Notes in the archives}

These are big binder that store old subjects that you won't be putting
anything into for a while. Start with A-Z, split into A-M/N-Z, and
further as you fill them up. Make sure they have transparent covers
and transparent spines that you can put papers into in order to
declare their letter ranges quickly.

The ``Chaose'' subject -- a non-subject, should go under ``C''. Dump chaos
into the archives frequently.

The archives have a second use as well: In addition to storing
subjects that aren't being used, it is also used as an archival space
for subjects that ARE in use, but have archival content.

Some subjects have old junk in them, but old junk that you still want
to be able to follow up links to. You mark old junk with a red mark at
the bottom of the page (I use the Japanese/Chinese mark for ``Old''),
and then you store it at the end of the subject space. (We'll talk
about this later, in the Intra-Subject pages, discussing page layout.)
The archival content is at the back. When you decide to get around to
it, you can take the archival content and throw it into the Archive
subject. Even though you are still using the majority of the content
in the common-store binder, or perhaps even carrying it temporarily in
your carry-about binder.


So. We're about done discussing Materials; The last topic is handling
optimizations. That is, tricks for dealing with papers.

I'll talk about papers you are going to throw away, and then I'm going
to talk about handling speed lists.

\theme{Paper you will throw away.}

Put a gigantic big ``X'' over any paper you are throwing away. You don't
want to keep running back and forth to the trash. Just start a stack
of pages you are throwing out. Put a big X on them as you decide to
throw them out. In RED, if you can.

If you have a page that you are GOING to throw away, but are still
using, temporarily, put a DASHED X on the page. That signifies to you
that the page is on its way out, but still in use. THEN, when you are
done with it, put a solid X over the dashed X.

\theme{Speed lists.}

It is always best to put a speed onto the subject page's speed that
the speed is going to.

Let me make this clearer: You do NOT want to use the pan-subject
speeds list! Yeah! You don't! Even though we made them! Because it's
another transcription step, and we want to minimize
transcriptions. What you want to do is put it on the destination speed
list first.

The only reason we have the pan-subject speeds lists is because we
don't always have access to the carry-around binder, where we are
storing the latest speed list for every subject.

But when you CAN, when you have access to the carry-around, put the
thought directly into the carry-around.

NOW: Frequently, you'll be thinking about some subject, but thoughts
about another subject are also coming to you. What you want to do is
to TAKE OUT those speed lists that thoughts are going to frequently,
and you want them close by your side. That way you don't have to go
rifling through dozens of speeds. You just have 3-7 by your side, and
work through those. Much quicker.

Next: When you have a big pan-subject speed list, with multiple
entrees to a single subject, you want to use that to your
advantage. You want to check them all off onto the one subject speed
list while you are there. Yes, seems like common sense, but I had to
figure out a lot of this stuff over time, so I'm telling it to you,
even though you may already know. Just in case you don't.

But remember: Avoid using the pan-subject speeds.

And now, having just told you that, I am going to give you another
case where you should use pan-subject speeds. Some times, you are
trying very hard to work on one thing, but thoughts just keep coming
at you from all angles. But you are trying so hard to stay on one
topic, and don't want to deal with all of the maintenance promises. In
this case, use the pan-subject speeds. Yes, it means more work for you
later, but, at least, you get to concentrate on your task at hand, and
trust that everything is caught into your pan-subject speeds.

There you are.

That is what I have to say here about handling optimizations:

\begin{itemize}
\item Trash X's
\item Pull out speed lists that are frequently accessed during a writing
session.
\item Transcribe pan-subj speed lists in batch.
\item Avoid pan-subj speeds, save when you absolutely need them, either by
being unable to carry your carry-about binder, or by difficulty
concentrating amidst flipping from speed list to speed list.
\end{itemize}


So in recap:
\pause


We talked about:
\begin{itemize}
\item Raw Materials
\item The carry-about, common-use, and archival Binders
\item Handling Optimizations
\end{itemize}

So you know what I have to say about the materials that the notebook
system rests on.

Next, I'll talk about general principles that apply across the entire
notebook system.

Then, we'll go into the intra-subject architecture, followed by the
extra-subject architecture.

Then I'll talk about the Theory of how this all works together.

Finally, for those techno-philes out there (and you are many), I'll
write about the Question of Computers. Why they suck for what we are
trying to do, why it doesn't HAVE to be that way, describe a simple
program, that, if written, could alleviate 50-90\% of the burden of
this system (albeit at a cost...), and I'll describe my notion of the
ultimate note-keeping computer system.

I will also talk in that last section about the versioning problem, a
problem that plagues even the existing notebook system, as I have
described it, though it is a bit more manageable on paper. Maybe Ted
Nelson has solved it. Maybe he hasn't. I don't know. He's not telling
us. I do not believe it can be solved. Not in a way that we really like.


\chapter{General Principles}
This is a description of some general principles, some general themes,
that apply to the entire note-keeping process.

\begin{enumerate}
\item Information Presentation issues:
   \begin {itemize}
   \item Page Layout
   \item Partitioning
   \item Info Density
   \item Page Numbers
   \item (Maps)
   \end {itemize}

\item Process:
   \begin {itemize}
   \item Late Binding
   \item Out Cards
   \item Tolerance for Errors
   \item ``Starting in the Middle''
   \item ``Divide when Big''
   \end {itemize}

\item Writing Form:
   \begin {itemize}
   \item Color
   \item Quality
   \end {itemize}

\item Psychology

\item Maps
\end{enumerate}

We'll start with Information Presentation issues:
Information Density,Partitioning, Page Layout, Page Numbers.

Information Density has to do with how much information we can cram on
to one pages. There are times where you are going to want a loose
density, and times where you will want very tight density.

When you are working with things like MOC's, TOC's, or any other
form of presenting raw data, then you want to make things as tight
as possible.

There are many ways of doing this, but one of the best ways is to have
a template that helps you write small and cram things together. For
example, I have standard form speeds (both subject and pan-subject)
that keep ~45 lines of text -- far more than a college rule. It makes
you write small. And it's not just height -- when you write small, you
write small in width too, so something that once took 3 lines now only
takes 2.

Information density is a MUST for tables of contents. No double
spacing, unless you love flipping pages and scanning with your eyes!
You want to be able to see as much as possible in as small a space as
possible.

On the other hand, there are times where you will want things spaced
out. If you are writing in a POI, you'll want to have plenty of room
for comments from the future. You'll want to have space to interrupt
yourself, or maybe later draw diagrams. You will want LESS information
density.

So keep these things in mind as you work on your notes.

% Next: 
\theme{Partitioning.}

Partitioning will be a recurring theme as you keep your
notebooks.

Let's take the example of a single page: Do you have a space for the
Title? How big will you want it to be? How about the page number? How
much space will you allocate for revision? How about the page's date-
do you want to leave space for that?

Content. As mentioned in information density, you'll want space for
future comments. Perhaps you are anticipating a lot of work in the
future, so you'll allocate more space for that possible future
content.

Now lets get off the page, and talk about namespace.

Whenever you create a system for naming things, you are working in
partitioning. You have only so many letters. True, you have infinite
glyphs, but they are kind of hard to make indexes out of -- they have no
intrinsic ordinality, the way letters do.

Some time you may want to reserve a space of page numbers for some
particular thing to be filled in in the future. We'll talk about page
numbers in a moment.

Partitioning is difficult for me to talk about in the abstract, so I
just want to leave you with understanding that ``Partitioning is
something that I'll be spending some time thinking about.'' When the
particulars of your immediate situation become clearer to you, you'll
see what needs to be done. You will have options. The strategies in
this book will describe many to you. Over time, you will gain skill in
partitioning.

\theme{Page Layout.}

The last two topics have been pretty vague: ``Think about info density,
think about partitioning.''

This one is going to be pretty specific.

On a given page, you can find the following things:

\begin{itemize}
\item Content
\item Date
\item Title
\item Page ID (``Page Number'')
\item Sequence Identification
\item Archival Mark
\end{itemize}

You are probably familiar with the first four, the last two may be a
little bit of a mystery to you.

The details of the first four:

Content will fill most of your page. I need not explain it.

The date goes in the top right corner. It reads something like
``(Sunday) 25 May 2003.'' I use the Japanese characters for Sunday,
Monday, Tuesday, Wednesday, Thursday, Friday, and Saturday. I highly
recommend learning those particular characters. They are not hard,
they are very useful, and they look far more different than one
another than the letters for Sunday Monday Tuesday etc.,. You'll be
able to mark down the day of the week a lot quicker, and you'll have
greater information density. Highly recommended.

The Title appears at the top of the page, centered. Usually the title
will include some sort of identification. For example, a title may
read: ``POI\#26  -- the Kitty Model''. The title is ``the Kitty
Model'', ``POI \#26'' is the identification for the sequence of pages:
This is Point-of-Interest \#26. You needn't always have a title, nor
need an identification. But it's best to have both.

Titles are most important for POI's, because they delimit the
boundaries of the POI. Anything that goes beyond the boundary of a
POI's point of agreements or title are basically lost, as far as
retrieval at a later date goes. That is, if it isn't described by the
title, you won't be able to find it. Stay in bounds, spark new POIs
when you need to. I'll talk about this more later, when talking about
POIs.

Now we have the ``Page ID''. It's a whole topic on it's own, so I'll
talk about it after I talk about ``sequence marks'' and ``archival
marks.'' For now, let's just say that the page number goes in the
bottom right corner, so that you can flip pages and find what you are
looking for.

\theme{Sequence Marks.}

When you create POI \#28, it may consist of 1 page, it may consist of 3
pages, it may consist (you want to avoid this) of 27 pages.

But you don't want to have to turn the page to see if there is more
material or not. That's a waste of time. So what we have are ``Sequence
Marks''. They effectively say whether there is another page in the
sequence or not.

As we see when we talk about Page ID's, some times you can tell just
by the Page ID. (``P27'', alone, means that there is no next page, that
this is a one-sheeter. But if it reads ``P27-1'', than you know that
this is the first of several.)

But generally you can't.

In the bottom right corner, you put a little glyph -- a little arrow
pointing to the right -- to mean ``Continues on the next page.''

And if there is nothing on the bottom right corner, that means that
you were too busy to put the arrow there, but that you can still,
probably go to the next page as well.

However, if you see a little ``box'', a little square, drawn in the
bottom right corner, then that means that that's the end of the
sequence.

Unless! Unless! You might extend the POI (or whatever sequence it is-
maybe some Research, or a Reference, or whatever) later, in which case
you need to put an arrow through the box.

The box is made transparent, so that you can later put something in it
to cancel the box.

So if you see a box with an arrow on top, that means that once this
was the last page of the sequence, but that you later extended it, so:
Turn the page.

The sequence mark appears ABOVE the page ID.

\theme{Archival Marks.}

Archival Marks appear to the LEFT of the page ID. (This is all in the
bottom right corner of the page, now.)

The archival mark is a RED mark (unlike the BLUE page numbers and
Sequence markers). It can look like whatever you want, I personally
use the Japanese Kanji for ``Old''.

It looks like this:

\pause

{\null\hfill\japanesefont\fontsize{58}{60}\selectfont 古\hfill\null}

\pause
% \begin{verbatim}

%    |
%  -----
%    |
%   ---
%   | |
%   ---
% \end{verbatim}

Looks sort of like a tombstone.

If you put that mark there, that means that the information on this
page is no longer required in the subject, in the common-store
binder. However, it may still be the target of some links you set up
some time long ago, so you want to keep it around. (In the Archives!
Not carrying it around with you everywhere.)

So what you do is POINT to the MOST RECENT information on the page-
put a note in red saying, ``SEE ALSO: (page id of more recent
information)'', and then at the very bottom of the page, to the left of
the page ID, put your red glyph for ``Archive.''

Archival pages are in the back part of your section, in terms of
physical page layout. Not the VERY back -- that special page is for
abbreviations and shorthand. But generally, you throw archival stuff
out back. Then when you want to save space, you take all of the
archival stuff, and merge it into the archive binders.

Finally -- Page Numbers.

I use the word ``Page Numbers'', but I should really be saying ``Page
Identification'', because it's actually much more than a number.

Here's are some actual ``page numbers'' from my notebooks:


\begin{quotation}
  \noindent ``Notebook S27-S47''\\
  ``GKI REF 1-II-1, 1-III-1''\\
  ``Mental Technique P7-1''\\
  ``P2-3''
\end{quotation}


The first one means: ``This is a page in the Notebooks subject,
representing Speed Thoughts numbered 27-47.''

Next: ``This is a page on the Global Knowledge Infrastructure; It's a
commentary on reference number 1; In particular it is commentary on
sections II and III of the document.'' (In the references list, you
would see that reference number 1 was ``Towards High-Performance
Organizations: A Strategic Role for Groupware'', by Douglas C Engelbart
in June 1992.)

Next: ``This is a page on mental techniques, the first page in POI \#7.''

The last, ``P2-3'', says nothing more than ``This is page \#3 of POI \#2.''
How do you know what the subject is? Because it's in the tab ``Personal
Records.'' Since you don't move pages around (we'll actually talk about
that later on -- there ARE times when you do -- see the section on ``Out
Cards'' below), there's no need to worry that you won't be able to put
it back, unless there's some freak disaster (such as hitting an
economy binder the wrong way, pages spill out, and then something
happens to ALSO, further, put the pages radically out of order -- I've
never seen that last part happen).

So the parts of a page ID are:

\begin{center}
  Subject  -- Segment  -- Segment ID  -- Page ID
\end{center}

It's a little different for reference segments, because they adapt to
the form of the book that they are commenting. But I'm getting ahead
of myself.

The ``Subject'' part is optional. You don't HAVE to repeat the subject
over on every page. I'd argue that it's not even good to do that,
unless you have good reason to believe that your binder is going to
explode and all the papers fly out in completely different, unordered,
directions. If you fear that kind of thing, put the Subject on every
single page. Or at least an Acronym for the subject. (Replace ``Global
Knowledge Infrastructure'' with ``GKI''.)

There ARE places where it'll be good to put the subject on EACH page,
and you'll even want to spell it ALL the way out.

In particular, I am thinking of the Speeds pages, and your P and P
pages. Your latest speeds pages, and your P and P pages, from myriad
subjects, will all be living right next to each other. You will need
to flip between them, thus necessitating the appearance of the subject
name in the Page ID. More than that though, you will need THE FULL
SUBJECT NAME spelled out, because otherwise you are going to have to
expand out the full name of the acronym when you are ordering the
speeds. ``Does MTK come before or after MP?'' quickly grates on the
nerves. (This is ``Mental Techniques'' vs. ``Metaphysics.'' N is \#12, and
T is \#19, so MTK does indeed come before MP, even though by acronym,
it would appear to go the other way.) So just spell everything out on
your speeds and on your P and P's.

After the (optional) subject is the (required) segment.

The segment signifiers I use, in no particular order, are:

\begin{description}
  \item[PJ] -- ``Project''
  \item[POI or P] --  ``Point of Interest''
  \item[RS] -- ``Research''
  \item[REF] -- ``Reference''
  \item[A/S] -- ``Abbreviations/Shorthand''
  \item[P and P] -- ``Purpose and Principles''
  \item[I] -- ``Index''
  \item[SMOC or M] -- ``Subject Map of Contents''
  \item[S] -- ``Speed Thought''
  \item[Cht] -- ``Cheat Sheet''
\end{description}

We'll talk about these segments more in ``Intra-Subject Architecture.''
All you need to know for now, is that there are these segments, and
that they have a short identifier, and you'll be sticking that
identifier in your page ID.

Most common will be:
\begin{description}
  \item[S]   (Speeds)
  \item[M]  (Map, or more appropriately, Subject Map of Contents [SMOC])
  \item[P]   (POI, the Point of Interest), and
  \item[REF] (Reference).
\end{description}

Sometimes I use ``R'' rather than ``REF'', but it's problematic because it
is easily confused with ``RS'' -- Research. Quite different things, though
similar.

Immediately following the Segment Identifier, you will have a NUMBER.

That number can mean one of either two things:

It can be a TOC \#, or it can be a VERSION \#.

They are only slightly different.

The TOC number means ``Ordinality in a table of contents.'' Even if you
aren't keeping a table of contents yet (there's not much reason to
make a table of contents over only 2 or 3 POI), you still have the
notion of ordinality, and that the pieces in the segment are in some
sort of addressable order. So that's the first.

The second, ``Version number'', is when you have things that don't really have
a table of contents.

Consider A/S (``Abbreviations and Shorthand'') for example. You never have
multiple A/S's. There's just one -- the A/S. Holding all of the
abbreviations and shorthands that you use.

Ah, but maybe it's getting over stuffed. Maybe you've filled out all
your hash tables in the A/S section, so you need to make a new
version -- ``A/S2''. You'll copy all of the original A/S (just ``A/S'';
though you can write in a ``1'' if you like) into the new, larger
tables, you'll archive A/S1, and then just use A/S2.

\pause

There you go.

\pause

The same goes for the maps. You usually start with just a single page
map. But eventually, you need to scrap it, and replace it with four
pages of map. So the first map was v1, and the next map is v2.

Your first map page was just ``M'', meaning ``This is a map page, the
first map page ever, and there isn't even a sequence for it, it's just
a single page.''

But your next map pages will be ``M2-1'', ``M2-2'', ``M2-3'', ``M2-4'',
denoting their pages within Map \#2.

You can later expand out with ``M2-5'', ``M2-6'', ``M2-7'' if you like.

And eventually, you'll do a major reorg, and you'll go afresh with
``M3-1'', ``M3-2'', ``M3-3'', and so on.

So these are more like Version numbers, in this case, rather than TOC
entry numbers.

Finally, after the subject, segment, and TOC/version number, you have the
page ID.

Most of the time, this is straight forward.

You start with 1, then you go to 2, then you go to 3, yadda yadda
yadda.

But there are two special things to note:

\begin{enumerate}
\item It's totally different in the REF segment.
\item Sometimes you want to put a page between two existing pages, so you
give it a ``half number'' or a decimal value. For example, if you want
to put two pages between P7-4 and P7-5, so what you do is you make
P7-4.3 and P7-4.7. Hey! There's no binder police. You can do whatever
you like, as long as it works for you.
\end{enumerate}

By the way -- I want to briefly comment on that principle. ``There's no
binder police.'' I'm writing this complex system to you, explaining how
I made it up, and how it works. What's most important is that you get
the IDEAS here, not that you actually replicate my entire system
exactly. In fact, I hope that you \emph{DON'T}. For one, you are living in
a different mind than I am, so you are going to probably want to put
things in a different way than I do. But more than that, I WANT TO
HEAR NEW IDEAS. I want to know what people do with this. And if you
just say, ``Hey, I did it exactly like you,'' well, what growth is there
in that? I mean, it might be good for a little while, but I really
want to see what else is out there. My system changed in a major way
at least once every 2-4 months. And it was always a positive
change. So I want to hear what you all do. And remember: There's no
Binder police, like my girlfriend always tells me about cooking. ``You
want to put paprika in there? Throw paprika in there. There's no
cooking police that are going to go after you.''

So get the meat of what I am saying, the IDEAS on how you can organize
stuff, and then adapt it to your domain.

THEN TELL ME ABOUT IT LATER! Yeah! I'd be astonished to hear that
people are doing this -- for one -- but to hear that you even carried it
FORWARD and tried out NEW Things. That'd just validate my life right
there, on the spot. <laugh>

Okay. So where were we. Decimal pages. All's fair in love, war, and
binders. Decimal pages if you like. This isn't BASIC programming,
where you have to renumber if you want to put something between lines
2 and 3.

\pause

But References.

\pause

I have found that it is best to annotate references by using the
book's own organization.

For example, say a book (or web page) is organized into three parts
(I,II, and III), and those parts are divided into chapters (1,2,3...)

Then as you annotate, USE THAT STRUCTURE.

The page ID for comments on chapter 3 of part II should begin with
``II.3''.

At the VERY END of the book's structure id, THEN put your normal page
numbers, ``1,2,3...''

So for example, if you wrote three pages to go along with ``II.3'', they
should be ID'd ``II.3-1'', ``II.3-2'', ``II.3-3''.

Or more completely, assuming this is reference \#7, in subject
``Robots'': ``Robots REF7-II.3-1'', ``Robots REF7-II.3-2'', and ``Robots
REF7-II.3-3''.

And that's that for page numbers and Information Presentation issues!

We talked about page layout, partitioning, info density, and page
numbers.

Maps are related, but we'll talk about them independently.

Next we'll talk about general PROCESS principles:

\begin{itemize}
\item Late Binding
\item Out Cards
\item Tolerance for Errors
\item ``Starting in the Middle''
\item ``Divide when Big''
\end{itemize}

I want to start with my favorite of these: ``Tolerance for Errors.''

This is ALL IMPORTANT.

You can't do this and be a perfectionist. (Well, okay, it does require
some sort of perfectionism to insist on recording and integrating
every meaningful thought. But lets ignore that for the moment.)

I had a good friend in high school. Every day, he would make sure that
the entire classes notes fit onto 1 page. This wasn't done for any
good reason, it was just the sort of thing like "step on a crack break
my mothers back", and you just get into it and can't stop. So he write
REALLY REALLY SMALL on the page. And each page was perfectly,
identically formatted.

That will absolutely NOT work here.

Now, suppose you are stuck in this. Just say.

Then THERE IS A CURE.

What you must do, first, is realize that the imperfection is
imperfect, because it is getting in the way of optimal experience of
life.

The second, is to INTENTIONALLY FUCK UP YOUR PAGE. And you must do it
a different, unique, creative way, each time, until you no longer have
a phobia of imperfection.

Take a big fat pen, of the ``wrong'' color, and put a big line down the
middle of the page.

Or!

Intentionally -- On Purpose -- Completely Mis-ID the page. Say it's a page
from another section, another segment, and a TOC entry ID number like
``Infinity'' or draw a little happy face where the page ID goes.

Put ``Yesterday'' as the date. Or whatever.

Just mess it up. On purpose.

And then sigh a breath of relaxation.

You've screwed the virgin. There's no need to worry now.

Similar to this notion of Tolerance for Errors is "Starting in the
Middle."

Suppose you have a new idea on how to organize your notebooks. That's
GOOD! You want to evolve your system. Most of your ideas will be good!
You'll have some bad ones, but all in all, most will be good, and
you'll want to encourage the process of evolving.

What you DON'T want to do is go back to your months worth of previous
notes, and adapt them all the the new system.

Absolutely not. If you do that, you're going to be stuck forever in
your old thoughts, whenever you get a new idea.

So the trick is to ``Start in the Middle.'' Just start NOW with the new
system.

If you want, you can partition out part of your name-space for a new
experiment. Maybe have a segment named ``X'' for a while, until you
figure out whether you like it or not. Then you can rename it if you
like. (When devising naming systems, always leave ``outs'' if you can.)

So we've talked about tolerance for errors, and starting in the
middle. These are process issues we're talking about, again.

Now lets go back now and talk about Late binding, and out cards.

Out cards: When you move a page from one place to another place, you
need to put an ``out card'' in the old place. That is, you put a page in
the old place that the same PAGE ID as the old place, that points to
the new page.

That's because sometimes you have links to the old place. You don't
know, and you don't care to keep track. If you had bidirectional links
all over the place (this seems to be one of Ted Nelson's favorite
ideas), it would take forever to do (you couldn't refer to something
without actually digging it up and then linking back), and you'd have
all these irrelevant links all of the place. Sometimes a forward link
matters a lot more than knowing that you are linked. Anyways.

You don't know if you are linked to or not, or by how many. I suppose
you could count, but it seems like a waste of time. The solution is
the OUT CARD.

If you find that a bunch of out cards are next to one another, you can
just consolidate them into one, with a wide-range page ID. For
example, ``POI3, Pages 4-7''.

Now that we've talked about out cards, it's easier to talk about late
binding. Late binding is a common theme in the notebooks.

You want to do work that doesn't apply to the present moment, and
that might be rendered completely unnecessary, AT THE LATEST TIME
POSSIBLE.

\pause
A demonstration.
\pause

You make a page, but later move it. So you have an out card.

Now you move the page AGAIN, so you have two out cards.

Out card 1 points to Out card 2 points to the page.

Now, suppose you find a link to out card 1. "That's interesting,
what's this?`` You find out it goes to Out card 2. ''Curious and
curiouser!" Finally you find the page.

Now, generally, if you follow a link, you are more likely to follow it
again in the future. It's a subject of thought and what not. So what
you do, after looking up the final link, is that you go to outcard 1,
and correct it to point not to card 2, but to the final
destination. And then you go to the ORIGINAL link, and have it not
point to card 1, but cross that out and put in the final destination.

Yeah! That's late binding. You fix it all up when you ACTUALLY FOLLOW
THE LINK.

But you don't do that before, because it would just take too long and
be too boring to fix up everything before hand. There's no need. Just
do it at the last possible moment.

Finally, ``Divide when Big.''

Sometimes a subject gets BIG. REALLY BIG.

And as it grows, you start to see dissimilarities where before you
didn't.

It's a little like Mozart or symphony music. One symphony sounds
pretty much like another symphony, if you don't listen to them a whole
lot. ``Oh, there's some classical music playing.''

But then, as you start to listen, and think about what you are
listening to, you'll start to notice distinctions and connections,
where you didn't notice them before. And as you do so, you'll see this
new structure.

That same exact thing happens with the musical stream of thoughts
going through your head.

The most dramatic example in my case was the fate of two subjects:
``Society'' and ``Metaphysics''. I now laughed thinking that I had them as
just singular buckets. But I can't really blame myself, because: How
should I have divided them?

The subjects are not logically arranged, by some sort of cosmic
organization. They are arranged subjectively, by their own connections
in our lives. The process of keeping these notebooks exposes the
connections in your mind. They give you a MIRROR to understand your
mind and thoughts.

Okay, so what happened to ``Society'' and ``Metaphysics''?

They blew up!

``Society'' became: Military. International
Powers. Meetings. Festivals. Communes. Anarcho-Science. Global
Knowledge Infrastructure. Democracy. Social Ideologies. Social
Goals. Electronic Collaboration. Activism (which later gave way to
Strategy.)

Wow! To think, it was all just one subject before.

And ``Metaphysics''? Spirit. Mind
First. Admonishment. Ethics. Values. Imagination. Personal Identity.

So: Divide when Big.

It'll help you focus your thoughts. I'll talk a little bit about how
to do it in ``Extra-Subject Architecture'', talking about how to spawn
out subjects from existing subjects. (Easier than you think. It's
MERGING subjects that's terrible. But that's pretty rare. Happens, but
it's rare.)

SO. We've talked about process. We've talked about dividing when big,
about tolerating errors, starting in the middle, we've talked about
late binding and out cards. This is all process stuff.

Before we talking about information presentation, like page layout,
partitioning, info density, page numbers. We're talking all together
about general principles.

Next we'll talk about Writing Form, Psychology, and Maps.

Writing Form first.

There are two major things to mention here: COLOR, and QUALITY.

And this is where paper and pen really start kicking the computer's
ass, and the computer-fanatics don't even know it.

Discussing quality is really made a lot easier by contrasting it with
computers, actually.

A long time ago, I stored all of my thoughts in a computer text
file. It was actually an AWESOME system. The computer has so many
advantages that the paper world doesn't. For example, you don't have
to put a thought in just ONE place -- you can easily put it into 5
different places! I call it ``Multi-cat'' or ``Multiple Categorization.''
It's easy -- just put tags. (It baffles me to this day why people who
make computer notebooks DO NOT do this more frequently..! There's all
this notebook software out there, and you STILL have to put a thought
in one place, and one place only..! They just have a single category
tree, and you have to put a thought in a single place. To do
otherwise, you have to copy and paste or something. Terrible.) Any
ways.

For all this awesomeness in the computer, you are unconsciously pulled
into a problem:

ALL OF YOUR TEXT looks EXACTLY THE SAME.

I mean, lets ignore the obvious problems with including
picture. (Yeah, like you really want to scan in every single image you
make. And like you really want to be so punished for visual thinking-
the BEST thinking.) Just consider straight text.

In the computer, ALL of your text is EXACTLY THE SAME.

YES, YES! I CAN HEAR YOU COMPUTER-PEOPLE'S COMPLAINING.
``But you can use FONTS!'' But you can make it Bold! But you can make it
Italics! yes! Yes! YES! I know it! You CAN do all those things.

But that doesn't make it \emph{FAST}. In keeping notes, you don't want to
constantly be dicking around with your UI. You want to be able to JUST
WRITE.

It annoys me enough that to switch pen colors, I have to flick a tab
at the top of my pen. But at least I don't have to move my hand away
from my pen, or move to a completely different section of the screen
to set a font, or move the cursor around. To change fonts for a
segment of text, you have to do all that stuff. And you STILL don't
get all of the variation you want.

And look: there are ADVANTAGES to having SLOPPY TEXT.

It tells you something about the development of your thoughts. When
you see sloppy text, that means "This is just a quick idea I spat
out.`` When you see regular solid text, that means ''This is something I
thought over for a while." You have your own writing style, and it
communicates to you things that are important to you, though you may
not consciously register it. (Actually, that's good: Unconscious
communication is far stronger, and doesn't get in the way of your
thinking.) All of this, all of this telegraphing, disappears on the
computer.

The diagrams, the writing style, all disappears.

And consider maps. Maps are basically the backbone of this whole
operation that I'm describing. You just can't do it on the computer
easily. It takes WORK on the computer to say, ``This text is tiny,'' to
give the nuances of positioning, size, quality, all of that stuff.

And icons. I use icons all over the place. That's hard to do in the
computer.

So, by contrasting with the computer, I have described the kinds of
things you want to concentrate on in your notebook. USE DIAGRAMS
EVERYWHERE. They are FAR better than coercive linear text. And USE
VARIABLE WRITING STYLES. Write sloppy, write neat, and everything in
between. It communicates to you. Use shorthand and abbreviation. Know
Gregg's script? Use that when it suits you. You can late-bind decode
it later. It's probably not as important as something else.

\pause 

Okay.

Enough on that.

Now lets talk about color.

Your pen has four colors: Red, Green, Blue, and Black

You will want to connect meeting with each color.

Here's my associations:

\begin{description}
  \item[RED ]:     Error, Warning, Correction
  \item[BLUE ]:    Structure, Diagram, Picture, Links, Keys (in key-value pairs)
  \item[GREEN ]:   Meta, Definition, Naming, Brief Annotation, Glyphs
  \item[BLACK ]:   Main Content
\end{description}

I also use green to clarify sloppy writing later on.
Blue is for Keys, Black is for values.

I hope that's self-explanatory.

If you make a correction, put it in red. Page numbers are blue. If you
draw a diagram, make it blue. Main content in black.

Suppose you make a diagram: Start with a big blue box. Put the diagram
in the box. (Or the other way around -- make the diagram, than the box
around it.) Put some highlighted content in black. Want to define a
word? Use a green callout. Oops -- there's a problem in the drawing -- X
it out in red, followed by the correction, in red.

Some times, I use black and blue to alternate emphasis. Black and blue
are the easiest to see.

If I'm annotating some text in the future, and the text is black, I'll
switch to using blue for content. Or vise versa.

Some annotations are red, if they are major corrections.

Always remember: Tolerate errors. If your black has run out, and you
don't want to get up right away to fetch your backup pen, then just
switch to blue. When the thoughts out, go get your backup pen.

BY THE WAY -- I forgot to mention this in the materials section, but
it'll do just fine here.

Those four color pens -- I think they're made in France or something. At
any rate -- YOU CAN SWAP COLORS. For example, say that you have one pen,
but it ran out of black. So you start using your next pen. But then
say that you run out of BLUE in the new pen. You CAN open up the pen,
pop out the blue, and put it in the newer pen. Yeah! The procedure is
difficult to describe. You just have to yank really hard on the
ink. Then push it into the new pens place. It works! It's not
advertised, but it works! So there you go.

Key-value pairs: Sometimes you have a big hash. For example, in
abbreviations lists -- you'll have letters A-Z running down the left
side of the paper. One line may have, say, 3 key-value pairs in it. In
my ``People Abbreviations'', for example, under ``MNO'', I have ``MT'',
``NC'', ``NH'', ``ME''. Those letters are written in BLUE, because they are
keys. The values, written in black, are ``Michael Turner", ''Noam
Chomsky``, ``Napoleon Hill'', and "Michael Ende.'' Quite an interesting
collection of people, no? That's why they get to be two-letter
people. <smile>


\pause

NEXT, the Psychology of notebooks.

I want to talk about being excited, ``stewing", and ''The Kitty
Model". All three very different things, but all about the
psychological aspects of notebooks.

Being Excited: Be excited about keeping your notes. Imagine what can
come of it! Experience the vision. You are building CLARITY. You are
organizing all of your thoughts together, and seeing what it adds up
to. The results *WILL* surprise you, and you *WILL* see things that
you have NEVER seen before.

Next: Avoid STEWING.

Stewing is what I call it when you are just floating over your
notebook, putting things in, maintaining it, and being overall pretty
directionless. Just watching connections form.

I suppose it's all right for a little while, and that it has it's
uses. You certainly have to do a degree of processing as you keep your
notes. But if you just found out that you spent 3 hours stewing over
your notebook -- you want to, and can, avoid that. Focus on the priority
tabs, and decide on thoughts to calculate out. You have problems in
your thoughts: figure out solutions. Go for "major notions per
minute", don't get so bogged down in details.

\pause

Finally: ``The Kitty Model.''

So-called because my girlfriends name is Kitty.

I really want to just scan in the page that she made (and that I
included in my notebooks.) However, our scanner has broken for the 3rd
time, and we really think it's dead now, SO, I'll just have to tell
you what ``The Kitty Model'' page depicts.

Words in parenthesis are either cartoon images or Kanji.

I am ``Lion''. ``Kitty'' is my girlfriend. ``Kitten'' is our daughter. (I
should introduce you to my family. Amber is my girlfriend. Our
daughter's name is ``Sakura.'')

Straight from ``Notebooks P26'':

\begin{verbatim}
------------------------------------------------------------
POI \#26: The Kitty Model                     (Mon) 26 May 2003

(Lion) ``Oh! I Thinked a Think!''      (Lion) Writes think down.

10 years later.

(Lion)                  ``How will I develop all these thinks?''
(2 stacks of notes.)    ``One at a time I guess.''

(Kitten) - ``Daddy!''
(Lion) (a page)     "OK this is a good think,
                     let's develop it a bit."

(Kitten) - ``Daddy, I'm getting married!''
(Lion) - ``Oh shit! I got another think!''

10 years later...

(Kitten) (Lion) (4 stacks of notes.)
``Daddy?''   ``Shit. I have all my thinks written down!''

    ... (grave marker) <- Lion

    (Kitty) (Bonfire) <- Lion's thinks.

                                                P26
------------------------------------------------------------
\end{verbatim}

Note the page ID in the bottom right corner, the title on the top, the
date in the top-right corner, and the page contents in the middle.

This cartoon speaks for itself. Particularly, on the immobilizing
features of the notebook system, and the perils involved here.

Contemplate deeply on this image.

Otherwise, you may find yourself in very dangerous territory.

I'm fucking serious. If you don't worry about this, then you are going
into major spiritual self-damage. If you don't believe in that kind of
thing, then consider damage in terms of however you contemplate your
life. But it can be REALLY BAD to do this for prolonged periods of
time.

I'm not saying it isn't a good idea to do this for a little while, or
even periodically. But you don't want to do this for much longer than a
few months at a time.

Unless you are a monk. In which case -- go for it. {:)}=

\pause

Let me know how it goes.

FINALLY.

I want to talk about Maps.

They are \emph{so critically important to this whole thing}.

Because they are the ASSEMBLY POINTS. That's where ALL OF YOUR
THOUGHTS come together into one place.

Your POI, your Speeds, your Research, your References, just
everything. You assemble it all together on the map.

And the map you construct -- that MAP is FAR MORE IMPORTANT than the sum
of the CONTENT of all of your thoughts. Because, with that map, you
can reconstruct the WHOLE THING.

Take this book for instance. What I've basically done, is taken my two
map pages from my ``Notebooks'' notebook, and I'm just going over the
maps I have. Serializing them out into text.

For example, on ``General Principles'', there's a link to ``Psychology'',
with ``.33  .37   p26'' next to it. In green, next to ``p26'', it reads
``the Kitty Model.'' Can you guess what .33 and .37 are about? They're
about being excited, and about not stewing. If I had put under ``.33''
the word, in green, ``Excited,'' and under .37 ``Don't stew'', I wouldn't
even need my speed lists. I could just reconstruct the whole thing for
you here in text, based on the map.

So there is that.

Now I want to talk about constructing maps, in the particular. (For
this, I'm going over to my ``Visual Language'' notes, and looking at the
SMOC, pulling out the section on ``MAPS''.)

And there I have it:

Multi-dimensional MOC's, Possibilities, in Explanations, and Why
Better.

For the sake of writing this book, I'm going to skip the possibilities
of mapping, and explanations, and focus in on Multi-dimensional maps,
and maps the are better than TOCs (in most cases.)

(I'm demonstrating this to you, so that you can see, as I go with
this, how they work.)

I see off in the distance ``Mental Coercion.'' It's it's OWN topic,
within this SMOC for Visual Language (``VL SMOC''). The big things in
this area are ``MAPS'' and ``Mental Coercion." Between them is the ''Why
Better field.`` Surrounding that are "Uncoersive'' (which itself links
to ``Mental Coercion''), ``Structure'' (which links to the hypertext
movement), ``Enables Strategy'', and ``Only read NEW ideas.''

Those things are written in tiny little lower case letters. And the
are surrounded by lots of little dots with small numbers by them-
references to the VL Speed Lists.

And before we go much further: YES, I do realize the irony here. YES,
I do realize that my writing here is terrible. I'm not a writer. And
YES, I realize that this should be written AS A HYPERTEXT. And YES, I
realize that for being such a strong proponent for visual language,
that this document should be VISUAL. And YES, I realize that it should
be mapped out, rather than a big long didactic mentally coercive text.

YES, I realize all of these things. But I also know that my computer
skills of actually inputting those things -- are actually pretty
poor. I'm a great programmer, but I don't know how to use
photoshop. And the tools out there are pretty poor for the kinds of
things I am describing.

And I realize that if I try to write in those ways, that this book
will never appear on the Internet.

And I further realize that there is NEAR ZERO content on the Internet
that has to do with the kinds of things I am describing.

SO. In conclusion. I say ``better a little than none at all.''

I only believe that, if you are actually reading this, if there is ONE
PERSON in the world interested in this subject -- that you will be
grateful that this work appears, even in this ultra-crummy form.

YES, I have talked with people who have read MY ENTIRE WEIRD
FILE\footnote{\url{http://speakeasy.org/~lion/weird.html}}, SO. I believe
people will read this. And that someone may even follow the directions and do
something like this themselves. Please contact me if you do.

I MAY (I would say ``Will'', but I don't really know for sure) make a
2nd draft of this document. (Whoa! Concept!) And if I DID, I would
put it in DOCBook. Than you'd at least have a table of contents, and
organized pages. As much as I hate tables of contents for their
weaknesses. And I MAY even scan in pictures and diagrams from my
notebooks, and if I DO, the text would become incredibly more clear
and accessible. And it is CONCEIVABLE, THAT, IN THE DISTANT FUTURE, I
would put together this all in a mapped, hypertext, icon including set
of pages. It would be a lot of work, but I could go that far. If
people were interested.

So the lesson is: If you are slaving through this, if there is a
Single Soul out there actually reading this: You MUST let me know. It
is a MORAL IMPERATIVE.

Thank you.

Back to maps, from Irony.

I want to write about why maps are better, and about how to frame the
first page of a bunch of map pages. I also want to write about
creation techniques.

Maps are better than TOCs because:

\begin{itemize}
\item They are mentally uncoersive.
\item They reveal structure in ways MOCs cannot.
\item They enable Strategy.
\item Incredible (useful) Subtlety
\end{itemize}

``Mental Coercion''. Let me describe this idea for a moment.

Think about a Shakespeare play.
Now think of a map of, say, the Earth.

The Shakespeare play is ``Mentally Coercive.'' To get it, you have to go
through the whole thing, start to finish. You can't watch it
backwards, and get the same thing. It's possible, but difficult, to
just look at portions of threads that interest you, without
significantly processing other sections. You have to scan a lot, if
you want to do that.

Now contrast that with a globe. If you are interested in a particular
thing, you can just go straight to that thing. Focus in on the State
of Washington, or whatever have you.

In your mind, the rest of the globe disappears. You're just looking at
Washington, and Canada, and Oregon, and what not. ("Yes, I am a
US-ian and Seattle-ite. Anything north of the border is just
'Canada'. Quebeque, BC, what are those things? It's all just Canada,
to my untrained ear." No cruelty intended. Apologies to the rest of
the world for the incredible harm our country inflicts.)

That's what I mean when I talk about ``mental coercion.''

It is my opinion that most books (textbooks in particular) are
unnecessarily mentally coercive. I believe that you could also write
fiction that was not mentally coercive, and still get around "but do
they understand the build-up?" problems that hyper-text fictions
have. But I am not here to talk about hyper-text fiction, I am here to
talk about maps right now.

So one: Maps are mentally uncoersive. Much of the remaining advantages
are based on this.

Next: Maps reveal structure.

Maps reveal structure in ways that TOCs, by nature of their forced
ordinality, CANNOT.

How could I possibly represent the links from MAP  -- to Why better -
Uncoercive  -- MENTAL COERCION in a TOC?

Both ``Map'' and ``Mental Coercion'' are ``high level'' constructs.

Can I imagine:

\begin{enumerate}
  \item MAP
    \begin{enumerate}
      \item Why Better
      \item Uncoercive
    \end{enumerate}
 \item MENTAL COERCION\\
..?
\end{enumerate}

Not only does it not work, but ``Uncoercive'' should be connected
BENEATH why better, and we're also screwed up because as soon as we
put in item 3, the link is broken between Uncoercive and "Mental
Coercion".

No, that's all wrong. I am convinced that the only reason that we do
TOCs is because we just haven't built the tools to make Maps. We are
being beat up by the constraints of our medium of expression.

Fortunately, this will all change in the future. Scott McCloud and
Robert Horn etc. all are hard at work at correcting this mistake, now
that we have the computers that can express what we really WANT.

Complex structure cannot be represented by a TOC. It can only be
represented by a Map. Even then, there are still problems, (for
example, non-graphable interconnections), but we are still light-years
beyond the TOC.

\pause

Next: Maps enable Strategy.

You can zoom in on precisely what you want to read.
To be STRATEGIC, you need CONTEXT. Without context, you cannot make
strategic decisions. With a TOC, you are limited to TWO pieces of
context: What's above, and what's below. (Actually, you also get to go
back an indentation level, and you can also look at children of a
super-topic. So that's two more.) So you are confined to a grid. But
we don't want that. We want to be able to go every which way, in order
to more fully see the context, the terrain, so that we can make
strategic decisions about what to read, or what to write.

Finally: You have the possibility of incredible subtlety.

I'm not talking useless or "This is so incredibly subtle, you will
never even get it."

I mean -- that you can position things, precisely, in order to make
statements that require no words. This goes back to the sort of
``unconscious communication'' idea I mentioned. It's BEST when you can
communicate complex ideas, without even speaking a word -- and people
``just get it.''

What am I talking about here?

I'm talking about how you can position ideas that are related close to
one another, and you don't even have to assign a label to the group of
ideas.

Or you can position one idea right smack-dab between two other ideas,
if there is a relationship between them. \emph{And people will get it.}
People can figure out what you can mean. And even if you don't draw a
line between them, people will pick it up.

This blends into my next topic, which is constructing maps.

When you create a map, as per my system, you have two basic types of
``materials''.

You have your LINKS, ``Hard content'': That is, your speeds, your POI,
your References, your whatever. Even other maps. Every thing you keep
in your subject, appears as ``Hard content'' on your map.

Then you have your MAGNETS. These are words that ``pull'' on the hard
content. They build your structure.

Here's an example from my notebooks, particularly ``PFT''  -- Public Field
Technologies. I wanted to make a map of what PFT meant to me. I made a
big list of all of the public field technologies:

\begin{enumerate}

  \item  Visual-Verbal Language
  \item  Self-help Books
  \item  Personal Notebooks
  \item  Home Organizing
  \item  Community
  \item  Co-ops
  \item  Communes
  \item  Community Dollar Networks
  \item  Free Software Dev Pratices
  \item  Community Democratic Self-Rule
  \item  Babysitting Networks
  \item  Community Public Papers
  \item  Community Wireless Networks
  \item  Festivals that INVOLVE Participants
  \item  Toolshare Networks
  \item  Activist EDU Networks
  \item  OpenSpace Technology (OST)
  \item  Social Blueprints
  \item  Social INET Organizing Blueprints
  \item  Group-Help Books
  \item  Arguments Databases
  \item  Collaborative Mapping
  \item  Groupware
  \item  Wiki
  \item  Anarcho-Science
  \item  Collaboration Techniques and Study
  \item  Field Advancement Study
  \item  Visual Facilitation
  \item  Public Field Technology self-Study
  \item  Open HyperDocument System (OHS)
\end{enumerate}

That's a list of what I call ``Public Field Technologies.'' But I don't
want to get lost talking about it all right now. The focus is on the
mapping process right now.

First that was just an unnumbered list. Then I numbered it. (1-30).

Then I started to look for patterns. I tried a few ways, and then I
realized that I could handle a substantial number of the items by
making a scale:

From Individual, to Family, to Clan/Tight-Community, to Loose
Community, to Global. Yeah!

So 2 and 3: Self-help books, Personal notebooks (this!), those are on the
``Individual'' end of the scale.

Then on Family, there's ``Home Organizing.''

You don't want to actually write out ``Home Organizing'', because it's a
lot of space, and a lot of writing. You just want to put ```4'' on the
map. That way, if you decide to move it later on, you just cross out
the ```4'', and put it somewhere else. Much easier. Much more agile.

Once it's all solidified and you are happy with it -- you can turn on
the Green, and expand out the numbers. But for now, you want just
numbers out there.

So the word ``INDIVIDUAL'' appears, pretty big, on the page. That's a
``MAGNET'' word. It's ``attracting'' `2 and `3 to itself. They are right
next to it.

Now let me point out something interesting:

`10 is ``Community Democratic Rule.'' Where did I put that?

It's not attached directly to a magnet word! Actually, it appears
BETWEEN two magnet words: ``Clan,Tight Community", and ''Loose
Community."

Clan/Tight Community has, immediately connected to it, ```6 co-op'' and
```7 commune". And Loose community has connected to it ''`5 Community
(Local)``, "`14 festivals involving participants''. Interestingly
enough, it also has some magnet words on it's sides -- "Community
Communications Line`` (w/ 12 and 13 attached) and ''Community Resource
Collection" (w/ 8,11, and 15 attached).

But Democratic Self Rule, \#10, floats between them.

So this is an example of some of the subtlety that maps allow, that
TOC's do not, and how they work out. Yeah!

Incidentally, for those who wonder:

The line from Individual  -- Global was just one half.

The other half is centered around Collaboration, and Communication
itself (Visual-Verbal Language).

So there you are. You should be able to map things now, at least
crudely. Your skill will increase with practice.

Now I want to talk about what to do when maps get big, and multiple
categorization of maps.

When maps get big, you want to rebuild them, and have a "distant
view". You also want to respect multiple-categorization. Frequently,
there are three ways of looking at the same thing, and you will want
to capture all of them.

The first map in a sequence of maps should be a MAP of MAPS.

Oh, by the way. Yes, you can have icons and pictures and smiley faces
on your maps. THERE ARE NO MAP MAKING POLICE. YOU CAN DO IT HOWEVER
YOU LIKE!\footnote{%
I kind of like Tony Buzan. I kind of don't. I think that his
rules are a bit constrictive. WHY must you use millions of colors? WHY
must you draw them LIKE THAT? I don't want to. I think it's a waste of
time. And I want to draw the maps how I want to. I don't find your way
particularly perfect or anything like that. And I don't think that the
ability to draw maps requires certification or anything like
that. Okay. I have too much of an Anarchist in me. Drawing Power for
the People! Much more a Mark Kistler guy. draw3d.com  -- YEAH! You're
ALL Creative Geniuses! YEAH! Okay -- I'm done.}

So have a map of maps at the beginning. And have super-maps as you
need them: Maps that give you a birds eye view of other maps.

And have teleporters and warps from map to map. Really, you can do
\emph{whatever you want}.\footnote{%
I swear, I have just been touched by the spirit of Mark Kistler, by
the mere thought of the man. Por la Sociedad Libre! I swear -- that man
wears too much black and red -- and that big red and black star? The
Raised Fist holding a pen? His insistence on the intrinsic value of
people? Hmm...}

\pause

ahem.

So. You now know why maps are cool, and how to make them.

You won't just make them in your SMOC and GSMOC, you may also make use
of them in your POIs, as I did with the PFT map.

And we're done with this section! We've discussed the general
principles!

A brief rehash:

\begin{itemize}
\item Information Presentation (page layout, partitioning, density,
  page numbers)
\item Process (late bind,out card, errors, start middle, divide when big)
\item Writing (color, quality)
\item psychology (the kitty model)
\item maps
\end{itemize}

Next, we'll talk about the architecture within a subject.

Then we'll talk about the super-architecture, binding all of the
subjects together.

Then a bit about the theory of notebooks, and finally, the question of
computers.


\chapter{Intra-Subject Architecture}
Within a subject, you have a large collection of papers. They have a
logical organization (into segments), and a physical organization (the
sequence of papers).

The major segments are:

\begin{description}
  \item[P and P] purpose and principles
  \item[SPpeeds] speed thoughts
  \item[SPMOC] subject map of contents
  \item[PPOI] point-of-interest studies
  \item[RPS] research
  \item[RPEF] reference
  \item[PPJ] project
  \item[IP] index
  \item[CPht] cheat sheets
  \item[AP/S] abbreviations, shorthand
  \item[XP] experimental, temporary (UNLINKABLE)
\end{description}

At least, those are the major segments I have hammered out well. There
are MORE segments that I would like to practice, formalize:

\begin{description}
  \item[CEP] chronological episode
  \item[TD] topical deliberation
  \item[DD] data dictionary (definitions)
  \item[L/T] lists and tables (high info density)
\end{description}

Something to recognize here is that you can make up whatever you
like. However, you don't want to just make up a new thing every time
you have a new thought or format. You want to think about your
divisions, and create new ones sparingly. If you can fit something
into an old one, and nothing suffers, then preserve the old
situation. It's only when you have something really ``new'', that is
best served by a new category, that you will do well.

I haven't studied and thought out the details of why this is the case;
It is just something that I happen to notice. With practice, you can
flesh this out. One day, YOU can write a great explanation on how it
works, and we can consolidate everything into one huge glorious
document.

Now -- YOU DON'T HAVE TO HAVE ALL THESE SEGMENTS. Remember: Late bind,
late bind, late bind! Only build what you have to when you need it.

I should add here also that -- for many of these, you'll want to make
TOC's for them. When you start a subject -- you know, you've collected a
few pages in the ``Unplaced'', all with the same subject marker. And
after it reaches about 5-10 pages, you say, "Well, let's make this
into a full-on subject now." So your subject starts with roughly 5-10
pages. You don't have to start writing a POI TOC if you only HAVE 2
actual POI. Wait until you actually could USE a POI, before you write
one. After you have about 10 POI, THEN make a POI TOC. Late bind.

So we have segments. What else do we have in the subject?

We have the physical organization to talk about as well -- the way
papers are physically laid out, from front to back.

It is the shorter topic, so I'll describe it immediately.

% ToDo: remake this map using TikZ
\tikz [rounded corners]
\graph [spring layout, sibling distance=18mm, level distance=18mm,node sep=10mm]
{Physical layout -> 
    {
      Title layer -> {
      The Subjects Tab Page,
      SMOC {(subject map)},
    },
    Lookup layer -> {
      S {(speeds)},
      TOCs {(tables of contents)},
      I {(index)}
    },
    Contents -> {
      Just about Everything Else,
      Archival Store
    },
    Quick access -> {
      Cht {(cheat sheets)},
      A/S {(abbreviations)}
    }
  }
};
\begin{verbatim}
PHYSICAL LAYOUT

  (Title Layer)
The Subjects Tab Page
SMOC    - subject map

  (Lookup Layer)
S       - speeds
TOCs    - tables of contents
I       - index

  (Contents)
(Just about Everything Else)
(Archival Store)

  (Quick Access)
Cht     - cheat sheets
A/S     - abbreviations
\end{verbatim}

(Note: The P and P page does NOT go in the subject. The P and P pages, one for
each subject, are collected into a grand P and P collection area. More on
this in the ``Extra-Subj Architecture''.)

That is:

You start with the actual tab page, that delimits the subject in your
notebook. You know -- it's a big yellow/tan sheet, it has a plastic tab
sticking of the edge, and you slip a little paper in. On that little
paper you slip in, you place the name of the subject. Simple as
that. Your subject starts with that.

Then you have the SMOC -- this may be one page, it may be many pages. If
you have many pages, the first one should be the page that points to
the rest of the MOC pages, or presents your ``super-map'', or whatever.

By the way -- SMOC means ``Subject Map of Contents.'' That is, it's a MOC
that applies over a subject, rather than a GSMOC, which we'll talk
about in ``Extra-Subject Architecture''.

Following the SMOC, you store your Speed Thoughts. Now realize -- you
WILL be missing some of your speeds -- the latest ones, in fact. Because
you are carrying those around with you, in your carry-about
binder. But most of the time you are dealing with your subjects,
you'll be in your common-store binder, or maybe even in an archive
binder. But the speeds that are not on the latest page, you will store
right after the SMOC.

Why? Because your SMOC will refer intensively to your Speeds. You'll
have little ```28'''s and ``33'''s that you are going to want to collapse,
by using the Speed lists. You don't want to have to fish around into
the middle of your binder, looking for the speeds. So put them RIGHT
AFTER the MOC.

If you are in the process of doing a LOT of work with a particular
map, you're just going to want to open up the binder, pull out your
speeds, close the binder, and work with the pages side by side.

So immediately after the MOC is a convenient place.

THEN, you follow the older speeds with the TOC's -- the table of
contents for the rest of the stuff.

Again, for similar reasons. You'll have ``(5)'' or ``P5'', however you
choose to notate it, on your map. And you're going to not want to go
fishing through the contents of your subject. You're going to want to
just glance at the TOC, and see that POI5 is about ``Naming'', or
whatever.

(BTW, after you perform a lookup on a map, and you are pretty sure the
item won't be moving around a bunch, switch your pen into green, and
write a 1-3 word description/mnemonic next to the link on the map.)

Ah -- there's a very SPECIAL TOC -- that is, your references list..!

References can be either ``expanded'' -- meaning that you've actually gone
to the work of analyzing them on paper, or ``not expanded''. Meaning you
just keep a reference to it, so that you can write bibliographies, or
refer it to friends, and what not. Remember that reference page
numbering partially adopts the actual reference's structure. So, the
actual reference pages serve double as a TOC.

Sadly, the pages can be in only one place at a given time. I keep them
in the carry-about binder, immediately coupled with the latest speed
list, so that I have it on hand to cite to friends who are interested
in something I am talking about. It also helps in libraries and
bookstores when I decide to make good use of my time by looking things
up. More on references later. The point is: The References pages are a
special form of TOC over your reference analysis, consideration.

Now: What order do you put your TOC's in?

Put them in alphabetical (expanded, not abbreviated) order.

Let's suppose you have 4 research entries, 2 expanded references, 13
POI, and 1 project.

\begin{verbatim}
Research
Reference
Point of Interest
Project
\end{verbatim}

...alphabetically:

\begin{enumerate}
\item Point of Interest      -- POI
\item Project                -- PR
\item Reference              -- REF
\item Research               -- RS
\end{enumerate}

After the TOC's, you place your Index.

I'll describe it later, but for now: It's basically an A-Z/123/Symbol
mapping from a subject, to ALL of the resources you have on that
subject.

It's LATE BOUND -- that is, it isn't current. Maintaining a current
index would take for ever, and you'd only use some parts of it
anyways. Maintaining it would be a constant interruption. Bad Bad Bad!

What you do is -- whenever you find yourself flipping through your
notebook looking for all occurrences of a subject -- a minor subject,
since major subjects already appear nicely on your MOC -- then you cache
your results onto the index page. More on it later.

After the Index, you have pretty much ``Everything Else'' that hasn't
been already described.

How do you organize it? The same way you organize the TOC:
Alphabetically, by full expansion.

If you have 3 references, say REF3, REF9, and REF21, then they should
(obviously) appear in the order REF3, REF9, and REF21 -- numerically.

After you have placed ``Everything Else'', then you have the ``Archives''.

Archival pages are the ones with the red glyph at the bottom denoting
an ``Archival Page''.

You organize ALL archival pages in Alphabetical order, based on
Segment.

YES -- ALL of them. Even including your Maps, TOC's, and Indexes, A/S's,
whatever. It all goes in alphabetical order, when it comes to the
archives.

Do remember: When your archival section grows unwieldy, or if you just
want to ``get rid of it'', you can pull all those pages out, and merge
them into the archival binder's place for the section.

Finally, you have your cheat sheets, and AT THE VERY BACK, your
abbreviations/shorthands lists.

Remember that the EASIEST places to reach within your subject are: The
VERY FRONT, and, the VERY BACK. So we keep important, frequently used
things there. The very front is the most important map of all of your
maps, and the very back is your most frequently accessed abbreviation
sheet.

You SHOULD be using abbreviations -- LOTS of them. Unfortunately, I
haven't written much on abbreviations, but there's a reason for that:
Many people have already written a lot on the Internet on the
subject. Look it up on the Internet. If you want to get really wild,
use Chinese/Japanese Kanji, or use Gregg script. Best: Use a visual
language's iconography.

So, that, in short, is the paper layout.

Now, let's talk about the individual segments themselves. Then we'll
talk about some of the experimental segments.

\theme{PURPOSE and PRINCIPLES}

The P and P is unique in that it has ZERO presence in the actual subject
pages, unless it is an old version in the archival pages.

The purpose of the P and P is to determine what goes IN the subject, and
what goes OUT. It describes the BOUNDARY OF THE SUBJECT.

If it turns out that something goes OUT of the subject, the P and P page
also gives you some hints on where to send it.

There are two ways that I have denoted P and P pages.

The old way is to take a page in half, make the top have ``INCLUDES'',
and the bottom half ``EXCLUDES''.

Form:
\begin{verbatim}

--------------------------------------------------
    (subject name) P and P                date

INCL

  * (inclusion)
  * (inclusion)
     * (exception)
  * (inclusion)



EXCL

  * (exclusion)          (target)
      * (exception)
  * (exclusion)          (target)
  * (exclusion)          (target)

                            (subj name) P and P(ver\#)
--------------------------------------------------
\end{verbatim}

Here's an example from my books:

\begin{verbatim}
--------------------------------------------------
   Personal Psychology P and P            [no date!]

INCL   * Clearly Psychological Forces
       * Self-Image
       * Motivation
       * Feelings
       * Self-Help Techniques

EXCL   * Non-Mechanical Forces  (->MP)
       * National Forces        (->MP?)
       * Very Broad Modeling of my Life (->MP)
       * Gender Studies         (->SOC?)
       * Imgn                   (->IMGN)
       * Values, Goals          (->Values)
       * A.C.T.S. functional details (->ACTS)
       * Inter-Personal Psychology (->PPL)
       * Though Focus Techniques (->MTK)

                        Personal Psychology P and P
--------------------------------------------------
\end{verbatim}

Let's note something first though-
This page is out of date!

So as a demonstration of Late Binding, let me fix this now.

``MP'' -- that is, ``Metaphysics'', was blown up a while back. I don't think
many of these redirects are correct now.

What is pointing to MP? ``Non-Mechanical Forces'', ``National Forces'',
and ``Very Broad Modeling of my Life.''

I know right off that ``Very broad modeling of my life'' should go into
``Personal History.''

Switch the pen to red, cross out ``MP'', and replace it with ``PHist''.

How about ``National Forces'' and ``Non-Mechanical Forces''-? Those DO
belong in Metaphysics. Just to be sure, though, I check the GSMOC, and
see if the ideas would rather gravitate elsewhere. The GSMOC suggests
proximity to Spirit, Values, Imagination, Personal Identity... No,
it's none of those. So we'll keep it in MP for now. If there are
enough related thoughts in MP, these subjects may ``break out'', be
ejected from MP, but for now, they'll live in there. The closest is
``Spirit", the purpose of which is explained in the ''Spirit and Awareness
P and P" page, but glancing at the page makes it clear that the ideas
don't fit in there. MP it is.

So that's how P and P works. It tells you what to include, and what to
exclude. And the things excluded, it tells you where else to put
them. (Or maybe not. If there is no such place yet, just leave the
target blank.)

There's another way to do P and P-

That is to make a diagram. You put a large circle in the center with
the P and P's subject. Then you draw lines out to words representing other
subjects. What is included goes center-ward, what is excluded goes to
the extremities, to the subjects that are their actual targets.

When inclusion and exclusion are along an axis, the axis takes the
form of a line, with subjects at either end. Put the
exclusion/inclusion specification at the ends of the line. That way
you can visually see how to cut topics.

There is usually only a single P and P page per subject; I have never seen
one grow beyond one page.

Now you understand P and P.

Next:

\theme{SPEEDS}

Your speed thoughts pages are ideally built by a computer. When I get
around to putting this online, I'll also place the Word documents that
include my templates.

Remember: We want info density for our speed thoughts. Pack as many
onto a page as you can.

The format of a speed thought page looks like this:

\begin{verbatim}
--------------------------------------------------
        (Subject Name) S__-S__         ___________
V # Hint @ Content
_|_|____|_|_______________________________________
_|_|____|_|_______________________________________
_|_|____|_|_______________________________________
_|_|____|_|_______________________________________
 |(and so on...)
_|_|____|_|_______________________________________
_|_|____|_|_______________________________________
                    []      (Subject Name) S__-S__
--------------------------------------------------
\end{verbatim}

A bit of explanation is in order.

The ``V'' is some sort of glyph that means ``checked off''. I use a
downward pointing arrow, but you could just as well have a check mark,
or just a dot, or whatever.

That column means whether the given speed has been mapped or not.

You'll collect speed thoughts quickly, probably faster than you can
map them. Every now and then, you'll go over your speed thoughts and
map them -- preferably from most recent to oldest (most recent tends to
be more immediately relevant, and worthy of thought). You can go in
any order that you want. But you want to keep track of what you've
mapped, and what you haven't.

The ``\#'' column is where you number the speed thought. If it's speed
\#47, the number ``47'' should appear here. (On the first row line of the
speed -- a three line speed has the number appear only in the first row-
the rest, leave blank.)

Note: When you are mapping speeds onto the map, you're going to want
to do it like this:  ```47''. Just a single dot, to denote that 47
refers to a SPEED THOUGHT. Speed thoughts will, by far, populate the
integrated MOC. It'll look like constellations -- lots of little black
speed map stars, with blue structural lines and magnet words revealing
the underlying structure of your thought.

Believe me -- it's beautiful when you see it all done out.

Next comes the ``Hint''. The ``Hint'' is a 1-3 word description of the
CONTEXT that the speed thought lives in. This is MAJOR important! Why
is it so important?

Because when you are mapping your speed thoughts, you don't want to
have to keep recognizing the content of the thought -- you want to just
put the thought where it goes. Thus the aid of the context hint.

Next comes a funny little ``@'' sign column. Actually, I use the
character ``Psi''. You can omit this column if you want.

I use it to provide some information on what \emph{type} of thought it
is. This connects into something I call ``Icons for Thought'', and it's
part of my MTK (Mental Technique) notes. I'm not going to describe the
system here; This is a book about notebooks. Maybe some day I'll write
about it. A brief description will do though: Some thoughts are
``problems'', ``goals'', ``questions'', or ``incentives." Some are ''starting
points.`` Some are requests to "analyze'', to pick apart, and some are
requests to ``articulate''. Some are notes on ``maps'', some are ``rules''
or principles, some are ``names'' or ``borders''. Some are ``see alsos''
(but NOT references, which go in REF), some are ``quotations'', some are
``hazards'' or ``rebounds''. There are many variety of types of thoughts.

Free free to skip that column.

Lastly, there is the content.

Some times, you'll just put a word in there, or maybe two words. Some
times, you'll fill three rows of content.

Put in what you are comfortable with. Lean toward the terse, away from
the verbose. Use abbreviations and shorthand.

You can use the speed lists as a ``to-do'' sheet as well -- maintenance
events that you want to see show up later. Check them off in the first
column when you complete them.

Again. There are NO Speed Thought Police. You can lay out whatever you
want. Add columns, subtract columns. Whatever you do -- let me know about
it, or let the world know about it somehow. I want notebook creation
to be a creative science, after all..! Your thoughts and experiences
MATTER!


Now, I've presented the description of the Speed page, but I also want
to talk about some issues connected with Speed thoughts here.

\begin{itemize}
  \item Pan-Subject speed-thought lists
  \item Growth Process (Memento $\rightarrow$ Speed $\rightarrow$ Articulation)
  \item ``Completing'' a speed
\end{itemize}

Remember that there are Pan-Subject speed thoughts. The page and form
for a Pan-Subject speeds page looks exactly the same, except that
instead of \#, you have ``Subject'' -- where you tell what subject's speed
list is the target. And instead of checking off when you've mapped it,
you check off when you've transcribed the speed thought to the
appropriate speed list.

Next: When you are recording speed thoughts, there is a sort of
``growth process'' -- a scale of articulation.

\begin{verbatim}
0.   the idea
0.2. (repeating in your mind?)
0.5. (paged?)
1.   Memento
2.   Speed
3.   Articulation
\end{verbatim}

First you have an idea in your head. You might repeat it in your mind
to not lose it, you might add it to a peg list and review it
periodically, until you have access to paper.

There are strategies for holding thoughts in your head. Very briefly-
take the thought, reduce it to a short, few-syllable, word. As you
pack in thoughts, cycle through the words.

When you UNPACK, unpack only a single word first, for each item, until
you have them all out. Then go over the list again, adding a SECOND
WORD. After you have two words out, you're pretty safe. Then add a
third. Now you're solid. Now go over the list and give a single line
description.

Don't start with two words -- just go parallel, striping one word first,
then the second, then the third, then you are safe.

Your 1-3 word description is what I call a ``memento''. Then if you
expand it out a bit, I call it a ``speed''. 1-5 lines, tops. Anything
more, and you should probably be writing a POI entry, or some other
``articulation''.

The speed lists should contain memento's and speeds.

So, you have a view of where the speeds fit in the scale of ``an idea''
to ``full on articulation''.

Most thoughts are best left at stage 1 or 2. Just place them on the
map, and check them off. Some thoughts, however, you will need to
delve into.

Be sure to do so strategically.

When I talk about MOC's, I'll talk about strategy. IF I FORGET TO,
MAIL ME AND LET ME KNOW! Using strategy, you can figure out what to
articulate and what to leave un-expanded.

Yes, I'd use a speed list to maintain these promises, but hey -- I'm
going to build that list later. (I am an experienced notekeeper, not
an experienced book writer.) I'm aiming for raw content right now. In
future book experiments, I'll try other stuff. Right now, I'm just
racing to the end.

What more?

Ah -- Speeds to Completion. You want to eventually be ``DONE'' with a
speed thought.

Generally, the speed thought is ``done'' when it's mapped.

What happens when the map is fundamentally changed? You move to a new
map version? Well, when you redo maps, you want to lose as little
information as you can. You will invariably lose SOME, because your
old ways of looking at things are frequently wrong, or deficient in
some way. If you like, you can mark a RED check into the speed's box
when it is ``retired''. I personally haven't done this. If you put red
checks in all of the boxes, check the archive box [] at the bottom of
the page, and you can safely put the speed list in archive.

I find it best to LOOK FORWARD, rather than LOOK BACKWARD, in the
notebooks. (Psychology note!) Thoughts die. That is good. They are
reborn, symbolically, in your new map structure.

Is there anything else I want to say about speeds?

I once thought it would be a good idea to take speeds that were taken
off a map, and transfer them ``back'' to the speed list. That is, white
out it's checked box. As I said above, I think it's best to just let
them die. We WANT to forget old thoughts. And as Michael Ende likes to
point out, that something has entered our mind and then been
forgotten -- it still leaves a trace on us, in our unconscious. I agree
with Michael Ende. Let it go. That thought HAS helped you, carried you
forward. It contributed to helping you recognize a new map, a new
order. It's time is done now.

There. I have said what I want about speeds. I'll talk about how they
can ``navigate'' over maps in the maps section. Which -- is -- coming right
up!


\theme{SMOC}

The Subject Map of Contents. In the ``General Principles'' chapter of
this book, I already wrote a lot about SMOC. I want to fill in some
holes here, now.

In particular, I want to talk about:

\begin{itemize}
\item page layout
\item strategy
\item trickling speeds over the map
\item icons
\item transitioning maps
\end{itemize}

That is, I want to write about how a page is laid out, how to use the
SMOC to make strategic decisions, how to trickle speeds (and other
entries, but mostly speeds) over the map, icons on the map, and
transitioning from an old map structure to a new map structure.

A map has a simple page layout:

\begin{verbatim}
--------------------------------------------------
                 Map Title
(creation date)
(freeze date, once frozen)



               (your content here)





                     []      (subj) SMOC(v\#)-(page\#)
--------------------------------------------------
\end{verbatim}

I'm not going to write about what map content looks like -- go back to
the ``Maps'' section in ``General Principles'' to learn about that. I'm
going to talk about specifics in SMOC pages here.

The creation date -- list that first. That's when you make the map. Once
you retire a map, you give a ``freeze date''. That's when the map is
done. (Check the ``archive box'' [] at the bottom, too.)


\theme{Next: Strategy.}

After you accumulate, say, 20 speeds, a POI or two, and a few
references, and whatever else you have, it's time to get a good
overhead view of your thoughts on your subject. That will both suggest
places for thinking to plug in holes, and show you the ``boundaries'' of
your thought, so that you can expand those boundaries.

Almost always -- when you complete a map, you'll suddenly have an
avalanche of thoughts..! Not just immediately, but over the next few
days as well. Your mind, upon seeing the structure, will suddenly have
a ground to go further from. You've turned on the lights in the
present room, and can now find the door to continue to the next.

Now: Your map is 2D, but time flows linearly. You need a path of
progress. What you do is this:

You take out all those little sticky tabs that I told you to buy in
the materials part. They are about 1/2`` wide, at most, and maybe 1/4''
tall. You pick the subjects and locations that need the MOST work-
somewhere between 1-10 of them.

You write red words onto 1-10 of your sticky tabs, describing the work
to do. Then you place the red tags onto the map.

THOSE are your options. THOSE are the places where you should likely
devote your attention. As discussed in the theory section of this book
(next-to-last chapter), the source of input to your notebooks isn't
the speed lists -- the source of input to your notebooks is your
ATTENTION, which THEN produces speedlists.

Now: A note about these strategy tabs.

You don't have to wait for a remapping effort to make these. Any time
you have a thought about ``what to work on'', you can make a sticky, and
put it on your map. If there IS no place on the map for it, just put
it out floating in space on the map. That's just fine. You'll map it
out later.

AND: As things become unimportant to you (happens a LOT), just take
tabs off. Throw them away.

If you want to remember to put a tab on, but you aren't there at the
moment, just put onto your speed list, "\#43:blahblah:Remember to work
on BlahBlah." Then when you are processing your speeds, and you see
that, if it is still important to you, check off the speed, make a red
sticky, and put it on the map in the right place.

The speeds don't just catch ideas -- they also catch work
requests. Remember that.

So, where were we... I want you to leave this strategy session with
this in mind:

You have your sticky tabs. You stick them onto the map, to indicate
where work needs to be done. You take them off when you they become
irrelevant to you, or when you complete whatever issue it is. Then you
just throw the little sticky in the trash. It's work is done.

You do NOT want to write priorities on the map. The priorities change
VERY frequently. They should be going on and off pretty
frequently. You'll mess up your map if you keep writing all over
it. No need to replace it that often.

Anything else I need to say here?..

One last thing:

This really belongs in the ``extra-subject architecture'', but it's
related, so I will describe it here.

You may take off ONE sticky, the most important one to you at the
moment, and stick it on the GSMOC. I haven't described the GSMOC yet,
but for now, just know it is a map of all of your subjects.

That way, when you are pouring over all of your subjects, you'll see
what is the most important first thing to think about in that
subject. At least, what you thought was most important the last time
you were in there. (Things change quickly.)

And also: Try to keep only one sticky per strategy idea -- try to avoid
keeping copies at multiple levels. I tried multiple levels once, and
it just became a maintenance nuisance. Whenever you have multiple
levels, just take ONE item from the lower level, take it off of the
lower level, and stick it onto the higher level. Have only one
higher-level item for each lower level island that exists.

Now I'm done talking about strategy.

So, we've talked about the simple page layout, we've talked about
strategy -- next, we talk about trickling speeds over maps.

Okay: So you have a big list of speeds, and an empty map.

You make a new map version.

Suppose you were on Map \#1. But you have 100 speeds, and Map \#1's
getting old. Now you are making Map \#2.

Let's suppose -- actually, that you need a ``temporary map" -- a ''scratch
map." A wise idea -- because you might make mistakes, no? And you'll
want to correct them.

What should you number the scratch map? (Or should you keep it at
all?) I say ``YES!'' You should number it \#2! Not 1.5, and don't throw
it in the trash. Just call the scratch map ``M2''. Then when you make
the ``real'' map, label it ``M3''. That's TOTALLY OKAY.

And besides, I've been surprised by how many times the ``scratch'' map
ends up being the ``real'' map. And you are going to be interrupted some
times, too. So just treat the scratch map as a real map, and don't be
afraid of growing numbers.

We have a versioning system. USE IT!

So you have either no map (you are making the first one), or you have
a poor one, and you have a big list of unmapped speeds.

The procedure is as follows:

\begin{enumerate}
\item Take an idea off the speed list, preferably from the bottom.
\item Think, ``How do I think about this idea, in terms of structure?''
\item Build missing structure, if it isn't there.
\item Put the speed thought on there.
\item Check off the speed thought.
\item Are all speeds done? Or are we satisfied? Or are we interrupted?
   If No, go to 1.
 \end{enumerate}

Here's an example.

Here's a Speed List:

``Electronic Collaboration''

\begin{verbatim}
# | Hint                 | Content
--|----------------------|------------------------
1 | Structured Email     | Should people structure their Email? ex name-sys
  |                      | f titles, 1 email/topic address. ref to
  |                      | ``Struct Considered Harmful.''
--|----------------------|------------------------
2 | Wiki, Populating     | Can't just have wiki, MUST have ppl to
  |                      | populate it, or pop yourself. ppl add
  |                      | later. note: opposes ``if build, will come.''
  |                      | some will, but few, in my experience.
--|----------------------|------------------------
3 | Map Software         | (?) Collab make/chg map of subject. refs:
  |                      | pt to webpage, books, individuals, orgs,
  |                      | articles, ... diff sizes, colors, fonts...
  |                      | also: big changes affect many should be psbl
--|----------------------|------------------------
4 | Wiki, Canonicalizing | ways make changes, harden w/ time
--|----------------------|------------------------
5 | Software Map         | keeps research map, 2d collab and maintained.
  |                      | ppl submit refs for incl, ] vote,
  |                      | pass=good. fail: apply elsewhere, correct
  |                      | xyz, mark spam. Book, WP, PPR, PROJ...
\end{verbatim}

Looking at the hints, we see we are talking about email structures,
wiki, mapping software. But there are OTHER ways to cut this as
well. What structures does this suggest?

First, let's take the obvious ones: Types of software.

\begin{verbatim}
                 wiki            map software
                     \          /
                      \        /
                       SOFTWARE
                          |
                          |
                        email
\end{verbatim}

Okay, so there's one map. And we put those thoughts on it, too:

\begin{verbatim}
              `4
                 wiki            map software
              `2     \          /     `3  `5
                      \        /
                       SOFTWARE
                          |
                          |
                        email  `1
\end{verbatim}

Okay, but, I get the feeling that this is sort of shallow, don't you?

What else can we do with this?

Well, the first was talking about STRUCTURE -- structuring people's
communication. That's definitely something I want to think about.

And the second -- that's about RECRUITING people, and POPULATING your
collaboration space, right? We're going to want to keep our eyes on
those.

Now, let's look at the third: We're talking about mapping software,
but it also has to do with CHANGES, and it suggests VERSIONING to my
mind as well. These are concepts we're going to want to have structure
for.

S4 is similar. It's also about making changes, and hardening with
time. I wouldn't be surprised if I had those ideas at a similar time,
in fact. Another idea for ``CHANGES'', maybe versioning as
well. ``CHANGES'' and ``VERSIONING'' are pretty close to one another, no?
We'll represent that graphicly, in the map.

S5 is smilar to S3, wherever S3 is, we'll have S5 as well. We talk
a bit about VOTING, no? And it's connected to changes as well.

So supposing that we wrote those key words we realized on paper, we
get the following:

\begin{itemize}
\item STRUCTURE
\item RECRUITING
\item POPULATING
\item CHANGES
\item VERSIONING
\item VOTING\footnote{%
Notice that we've got some ideas that we're capturing as we go over
these lists? I used to throw these away. But I think these
deliberations actually have some value, after having done these for a
while.

I call it ``Topical Deliberation''. My new experimental segment is
``Topical Deliberation'', or TD, and I record it by straight chronology
in it's TOC, and on their pages.
}
\end{itemize}



Now -- before we go on -- I want to remind you of something, something I
said earlier. I said that THE MAPS WE FORM HERE are MORE IMPORTANT
than the ACTUAL CONTENT ITSELF!

Keep your eye on that, and reflect on that, as we move on here.

So, we need to be a map.

After playing around with it, and remembering our connections, we draw
the following:


\begin{verbatim}
           STRUCTURE              RECRUITING


                                 POPULATING         `4 `2       `3 `5
                                                      wiki    map soft
                                                         \     /
                                                        SOFTWARE
                 CHANGES------VOTING                       |
                  /                                       email
           VERSIONING                                      `1
\end{verbatim}

How's that? Isn't that neat?

Yes, it'll get better.

Let's note some things here -- like, some of the subleties expressed in
the map. (Note: Many are missing, because this is on a computer, and
not on paper. I only have two levels here: lower case and upper
case... But I digress.)

Recruiting and Populating are not DIRECTLY linked, but we can see an
\emph{implied} connection between them, just by proximity.

And look at how CHANGES and VERSIONING are close together, and bound
by a line. VOTING and CHANGES are connected by a line too, but we've
let ``VOTING'' go out a ways. Isn't that interesting?

In a TOC, you just smash everything together. And you can't express
much more than ``above'', ``below'', ``indented in'', and ``further out.'' You
can simulate a forest of trees, but your trees can't intermingle (into
webs), and you only have so many levels, and you can only place your
trees in a row.

Most depressing.

But maps are ``alive''. They give you warm fuzzy thoughts and
feelings. YAH! Some times they even feel electric. RAH! Pikachuuu!

Okay. Now lets populate the rest. Do we just want to automatically put
our speeds on there? No: We want to consider them 1x1. Just because a
speed helped SUGGEST some structure, doesn't mean we necessarily want
to place it in CONTEXT with that structure.

S1: That was the one about email, and wondering if people should
    structure their emails. Refered to ``Structure Consider Harmful.''

Let's look in the references -- there it is. "Structure Considered
Harmful." It's REF\#8. Let's put that on our map. (NOTE: Putting
references on maps is difficult at times, because a book usually talks
about a LOT of stuff. If you want to, you can link in individual
chapters of a book, but you'll need to note them on the references
section. Generally, references appear at the ``top'' of trees, or by
``key nodes'' in webs, because they refer to so much amidst the
children. Sad but true. In this case, we are lucky. The map is new,
and so we mostly only have ``tops of trees'', and there is perfect match
between the reference, and the top of this tree.)

I think that S1 is fair game for ``Structure''. Same with REF8.

\begin{verbatim}
           `1    REF8
           STRUCTURE              RECRUITING


                                 POPULATING         `4 `2       `3 `5
                                                      wiki    map soft
                                                         \     /
                                                        SOFTWARE
                 CHANGES------VOTING                       |
                  /                                       email
           VERSIONING                                      `1
\end{verbatim}

Now we are running into the ugliness of computers. Sorry, I just can't
easily make ``REF'' appear in ittie bittie capital letters, above and to
the left of the number 8 -- superscript.

Computers are so frustrating in these primitive days of ours.

Now we'll go through S2-S5 a little quicker.

S2: This says that you can't just make a wiki, you have to get people
to populate it as well, or you have to populate it yourself.

Okay, this is very relevant to populating, so we'll put it there.

S3: This says you you want mapping software where people can make
changes. It also says that people should be able to make big changes.

Now, I have some reservation here, because it also has a LOT of stuff
about the mapping software that doesn't have to do so much with
changing in the abstract. To it's grace, it DOES say that big changes
should be possible, and that's an idea that abstracts -- there is a
general idea of big changes, and little detail work.

What I'd probably do if this was on paper is take out my green pen,
(green = icons, markup, meta,...), or maybe (tolerate errors!) my blue
pen (blue = structure), and put a little letter ``a'' by irrelevent
stuff for this purpose, and a ``b" by where it says ''big changes affect
many should be possible." So the designation on the page would be
```3b''.

Okay, pretend I made those edits, and we'll put ```3b'' on.

S4: ``ways make changes, harden w/ time.'' Talking about wikis.

If I were using my ``Psi" icons, I would have put the ''This is a
starting point for thought; Reflect on this" icon there. But I don't
want to go into that system; That's a Mental Techniques thing, not a
notebook thing, for my writing purposes. Maybe some other day.

So, we DEFINITELY want to put `4 onto the map, by changes. The idea of
hardening with time is interesting.

S5: Now -- this is about voting, and what to do with bad votes, and what
not, on the subject of Electronic Collaboration. Only a little is
really about software maps. This could apply equally well to wikis,
say, or any other type of collaborative system. Well, not email,
right? Well, maybe so, I can conceive of that. So this definitely
attaches to voting.

So this is what the map looks like when we are done:

\begin{verbatim}
           `1    REF8
           STRUCTURE              RECRUITING

                                      `2
                                 POPULATING         `4 `2       `3 `5
                                                      wiki    map soft
                                                         \     /
                `3b  `4          `5                     SOFTWARE
                 CHANGES------VOTING                       |
                  /                                       email
           VERSIONING                                      `1
\end{verbatim}


Ah -- Now isn't that interesting?

If it were just a TOC, it would look something like this:

\pause

\noindent ELECTRONIC COLLABORATION
\begin{enumerate}
\item   SOFTWARE
   \begin{enumerate}
   \item Wiki
   \item Email
   \item Map Software
   \end{enumerate}
\item  CHANGES
\item VOTING
\item  VERSIONING
\item  RECRUITING
\item  POPULATING
\item STRUCTURE
\end{enumerate}

Ref8, as traditionally done, would appear in the back of the book, not
associated with ``STRUCTURE''.

The connection between ``RECRUITING'' and ``POPULATING'' is not
apparent. I mean, sure, they are next to one another, but so is
``VOTING'' and ``VERSIONING'' and ``RECRUITING''. So you don't go looking
for those patterns, with so much nonsense there.  (You could put in
section delimiters here, but most people don't, and you still have
to put those section delimiters in a row -- a web isn't possible.)

Isn't that interesting?

But there's so much more! We just have only 6 items here -- when it gets
much larger -- say, 100 items, or 200 items, the differences become much
more dramatic and apparent..! The superiority of the MOC becomes far
greater.

Now: We've trickled speeds onto the map.

Sometimes, you will become aware that further work is needed, over
time.

Consider, that in my notebooks notebook, I used to have, attached to
``New Section Ideas'', about 9 speed thoughts.

Whoah! 9 of them! Gets a bit unwieldy. So what you do, when one part
starts to build up, is look up the items, and then further divide
them.

As it was, 4 of the ideas were related to chrono/episodes. So I
etended a line out from New Section Ideas, put ``Subject Chrono'' on the
end of it, crossed out S121, S112, S114, S120, and moved them to orbit
``Subject Chrono''. Great!

Whenever an area is getting congested, grow out.

Going in the other direction: Don't build too much unnecessary
structure! If you've got only 2 ideas in a location, don't put 8
magnet words out there. Just park those 2 speed thoughts next to one
magnet word, temporarily. As you build more speed thoughts there, THEN
build structure AS NECESSARY.

And a note about the strucuture: This isn't supposed to be some
``absolute cosmic eternal perfect ontological structure.'' This is the
associations YOU make in YOUR head. It's a map of YOUR mind. I don't
even think a good ACEPOS can exist. TOLERATE ERRORS. (If you're having
a hard time tolerating errors, again, intentionally fuck things up.)

Anything else about trickling speeds over the map?

Put them on the map. Park them somewhere. Push them out when they get
too close.

Um: Push ``Down'' too. If a section of the map gets too dense, start
another map page. Draw a big blue or red dashed line around the map
section that was dense, and write ``M7'' next to it, if the next map
page is Map Page \#7.

Oh -- and don't try to fit too many maps to a page. You have to balance
not-wanting-to-flip-to-the-next-page and
not-wanting-to-have-to-replace-the-whole-page because of a big problem
on just one map. You have plenty of paper -- USE
IT. \emph{BUT}... Information density, information density... It's a
natural tension, until we get these maps computerized. (See the
``Killer Easy Notebook App'' in the ``Computer Question'' chapter -- the end
of the book.)

Ah: HAVE A TEMPORARY SPACE ON YOUR MAPS FOR SPEEDS. You might want to
draw a little parking lot ``U'' shape for the hard-to-place
thoughts. And you might want to draw a little black hole too -- that's
for thoughts that should probably go to some other subject somewhere
else, and you just aren't taking the time to transcribe them over
yet.

YES: You CAN have fun with this. Draw whatever you like. It's OKAY,
it's YOUR MIND after all. \verb| =^_^= |

Of course, if you are not like me, I guess you can do it crystaline,
or however you like. If you're a bohemian, you can type it all up on a
typewriter, if you like. Whatever floats your boat.

OKAY. I'm feeling ``done'' here. Anything else? No? Going.... Going...

-Let me say something for a moment.

Remember: There are two types of thought -- intentional, and
incidental. This writing process -- this is mostly \emph{intentional}. The
framework I am writing to you is from my INCIDENTAL analysis. I
collected speeds (incidental), a few POI's (intentional), mapped them
out, and almost the Entire Structure of this book is based on the
resulting structure. However, that structure isn't everything. You
also have to ``reach out'' with your thought. As I write and expand
this, I am also ``reaching out." Thus responsible for all of the ''Um,
anything else?"'s.

Yes. When I write a second draft, or whatever, I'll take all this
stuff out. (I'll leave this old one around, for those who prefer this,
too. God bless your souls.)

Um -- oh!

Write BIG THOUGHTS BIG. For example, if you have a speed thought that
is WAY MORE IMPORTANT than the others, make it's ``dot'' bigger, and
make it's number BIGGER. How big? In proper proportion to the
neighboring territories..! And the same in reverse. If it's
mindbogglingly unimportant, write it so small that you need a
magnifying glass to see it.

OH! OH!

And another thing.
This is what I was fishing my unconscious for.

Your speeds -- if you have TWO SPEEDS that are almost COMPLETELY
IDENTICAL (happens more than you'd think!), identify them with a SLASH
between them.

For example, if speeds 47 and 98 are almost identical, put it on the map
like this:

```47/98''

But if S84 and S33 are VERY SIMILAR, but warrent individual attention-
put a COMMA between them:

```84,33''

There you are.

Done with talking about how speeds park on the mpa.

Now I can continue.

Icons.

ICONs? ICONS on the map? What the hell did I mean by ``ICONS'' on the
map? Let me go back to my Notebooks notebook... There it is! It says
``icons'' in green, next to ``P7'', right there next to ``maps''. (I'm
telling you my process, so that you can see that the notebook system
works, and how it works.) So I look up POI 7. Ah-HAH! There it is!

``Subject MOC Icons  -- Type Recognition Icons.''

OKAY, so this is about how you pin stuff onto the map. Unfortunately,
now I am really mad at our primitive computer technology, at this
ASCII I am using to write this in.

When I put this in DocBook/HTML, I'll be happy.

Here, briefly, are my link designators, on the map:

\begin{description}
\item[`\#]            a speed thought
              (the ```'' is just a little dot, top-left corner)
\item[(\#)]           a POI entry
\item[P\#]            also a POI entry!
              (the ``P'' is just a little p, top-left corner)
              (this form is standard, and applies to everything below.)
\item[R\#]            reference  -- special notes below
\item[Cht\#]          cheat sheet
\item[RS\#]           research
\item[PJ\#]           project
\item[J\#]            also a project
\end{description}

\theme{REFERENCES}
I use the kanji for ``book'' ({\japanesefont 本}), in tiny form, top-left of the ref\#, to
further denote that this is a ``book''. The letters ``ws'', or a little
icon of a web, denotes a website.  The kanji for ``person'' ({\japanesefont 人}) means a
person, the kanji for ``people'' (3 persons, {\japanesefont 人人人}) make an organization (that
is a reference.) Make up your own icons, if you don't know kanji! It's
fun and easy. {:)}= Just write them down and keep them somewhere in
your notebooks system. I DO believe ever notebook keeper, by this
system, should keep a ``notebooks'' subject.

These are just my icons. Make up your own!

But I STRONGLY recommend just `` `\# '' for a speed thought, since they
are the most common, and the `` (\#) '' shorthand for your POIs, because
they are 2nd most common.


Now the last part of talking about maps:

\theme{Transitioning your maps.}

I've already talked about map transition a bit here, I'll try to
repeat myself to a minimum.

As I wrote before: Keep your ``scratch maps''. Give them version
numbers.

If you are only changing a PAGE of a map -- that's a situation I've
never run into, but it should be solvable. I'd version off that page
number, using letters. ``M3-5c'' would be version 3 of the whole map,
version c of page 5. Archive ``M3-5b'' back with where you archived
``M3-5a''.

As you make a new map, you'll find that some old POI and Speed
thoughts and other things are now obsolete. They are either notes on
an old structure that no longer exists, or they are commentary on
things that are no longer important to you.

It's up to you what to do: If you want, you can go to the old thought,
and place a note reading, ``Idea obsolete; See XYZ'', where XYZ is
whatever is still present, but responsible for putting the old note to
sleep. Or, you can just not write anything.

It depends on how much time you have, how important that idea is to
you, how frequently you linked to that idea (or a predecessor pointing
to it), a number of other factors.

Personally, I think it's healthiest to LOOK FORWARD as you keep your
notebook. Let the dead bury the dead.

For stuff that is still on the new map: Just move it on over. If
magnet word ``A'' had number 124, 56, and 200 around them before, put
them around it now too.

If in doubt, put something on the new map. You can cross it off later,
if you like.

And that's it for the SMOC section! Let's recap:

\begin{itemize}
\item page layout
\item strategy
\item trickling speeds over the map
\item icons
\item transitioning maps
\end{itemize}

There's a simple page layout. You can use little stickies to keep your
strategy in order. Speeds (and other content) build the structure of
the map. (More accurately, your \emph{attention} builds the structure of
the map -- remember: This is a map of your mind.) There are icons that
you use (such as ```55'', ``(12)'', and ``REF5'') that you place on the
map. When you transition maps, keep your scratch work, and you can
point obsolete entrees to the new structure as you like.

We've talked about P and P, we've talked about Speeds, we've talked about
the SMOC; Next up are the POI!

\theme{Point-of-Interest (``POI'')}

A POI is like a journal entry, but specific to a particular point of
interest. The title specifies the point of interest.

I have three things to say about POI:

\begin{enumerate}
\item Content under the POI
\item Transgressing the POI boundary
\item Interlinking POI
\end{enumerate}

The first has to do with content within a POI.

What you write in the POI \emph{MUST} be consistent with the title of the
POI.

The title usually outlines a problem that you are trying to solve
(``These components are interfering with one another.''), or a goal that
you are trying to reach (``A Theory of XYZ''), or something that you
want to articulate in greater detail (perhaps you have collected a
bunch of speed thoughts that are related, and you want to describe
their interrelationship), a question you want to answer, a subject you
want to reflect on, whatever.

Then you write. You try to solve what you want to solve, or reflect,
or analyze, or whatever.

Anything that doesn't have to do with the title is effectively
``lost''. When you look on the MOC, and are looking for a piece of
information, if it's buried away in some POI \emph{with a title that
doesn't describe it}, then you can't find that piece of information.

Thus the great importance of KEEPING THE POI ON TOPIC.

Which takes us to (2): Transgressing the POI boundary.

When you get off topic, you are ``transgressing the boundary.''

If you catch yourself early enough, just get a new page and make it
the beginning of a new POI.

Sometimes you catch yourself late, though. In that case, CIRCLE IN RED
what has transgressed. Start a new POI page. Put a link from the red
circled part in the old POI to the new POI page, and vice versa. Then
continue in the new POI as if everything was fine.

If, at a later point, you decide that this particular POI is important
enough, and the link annoying enough, you can always make a new
version of the POI page, which brings us to the next subject.

(3) Interlinking POI

POI can be linked topically, or by version.

Topical links are easy -- you just say, ``See also: XYZ'', where XYZ is
the id of the other resource. For example, if you are referring to
POI\#25, you just write ``See also P25'', and maybe a little note on what
P25 is about.

If you are linking beyond the subject, include the subject as well:
``See also GKI P25'', were GKI your target subject.

(I personally use an icon in place of ``See Also''; I recommend doing
the same. Just make one up. Mine looks like O---O with a circle
around it.)

How do you version POI? Not the same way as maps are versioned.

To make a new version of a POI, just start a new POI entry. Then link
the new version back to the old, and vice versa.

If you ever follow a link to the old version, update the link to point
to the new version, wherever the link's source is.

If you get a long chain of versions, you can put additional navigation
information by the links -- you can write "First version, P4; Last
version, P14; Latest version, P46". However you like.

And that's it for POI! They are very simple, really.

\theme{``RS''  -- RESEARCH}

Research pages are like POI, but they are particularly about
researching some problem using other people's comments.

POI are ALWAYS ORIGINAL to YOU.
RS is ALWAYS BASED on OUTSIDE RESEARCH.

There is a bit of a blur between them. You will have to exercise your
own judgement. There are better guidelines I could write, but I'm in a
hurry, dammit.

I should distinguish RS from REF (reference).

REF is your own notes, attached to ONE, and ONE ONLY, REFERENCE.

Furthermore, REF comes out of attempts to UNDERSTAND A GIVEN
REFERENCE.

Abstract models that form in your mind because of a REF should be in a
POI (as an articulation of your thoughts). But the attempt to decypher
the reference itself goes into REF.

But RS is for when you are bouncing back and forth between multiple
references, and you have a train of thought going.

The title of the RS reflects the train of thought going on.

Inside of an RS, you refer to several REF's.

It can be as formal or informal as your needs meet.

If you are writing an RS, but don't want to take the time to cite your
references, \emph{that's okay}. You realize that you can only lookup later
what you write. But sometimes, that's not your purpose. You're just
trying to establish a line of thought, citations be damned.

You STILL want to KEEP that paper. It is a valid RS, though rough.

If you later care about it more, and want to add the full citations,
you can just write them on the paper, or make a new version of the
RS. Whatever you like.

\theme{REF  -- references}

Reference pagination is something I've already talked about. Remember
that it works basically like this:

  (subject) REF(reference \#)-(reference page scheme)-(page number)

  So for example, if you wrote three pages of notes while reading a
book on the Noosphere, your three page numbers might look like:

\begin{verbatim}
  GKI REF13-II.4.A-1
  GKI REF13-II.4.A-2
  GKI REF13-II.4.A-3
\end{verbatim}

  That is ``GKI" for ``Global Knowledge Infrastructure'', ``REF13'' being
the 13th reference in you references list, ``II.4.A'' meaning ``part 2,
chapter 4, section A'', and 1 (or ``2'' or ``3'') being your \emph{personal}
page number.

  I want to say something important here:

  \pause
  IF AT ALL POSSIBLE, JUST KEEP YOUR NOTES IN YOUR BOOKS!
  \pause

  If your book belongs to you, and it has a granger, then by all
means, grangerize it! It's \emph{your} book! It'll be worth MORE to you if
you put YOUR thoughts in it. That's what a granger is THERE for: So
that you can write in it. So do so. Write in your books, whenever you
can get away with it. No need to keep paper in your binders when
you've already got it between your book covers. AND you don't have to
do any expensive linking operation. So there you go, those are my
thoughts on the matter. I've said it.

  (Some people are religious about their books. Ah well.)

  Keep your notes on whatever isn't original to you in here. Books,
interviews, notes on people, the backs of bubble gum wrappers,
letters, whatever.

  Okay, REF is done. The hardest part about REF is the page
numbering. After that, it's obvious what to do.


\theme{PJ  -- Project}

  The PJ segment is where you keep notes that have to do with your
projects, \emph{if they aren't complete subjects in themselves!}

  If you are going to be working on a project for some time, make it
it's own subject. (It'll probably be predominantly POI, with a Chrono
segment, and many strategy notes, as well.)

  But if you have a small project that, won't take more than 3-5
sequences, just keep the notes for it within your subject, and label
them with the PJ segment identifier.


\theme{I  -- index}

  Some times, you'll be looking for ALL of your thoughts on a
particular subject.

  If you do that, you may repeat the search again in the future. Why
go to all that work, AGAIN?

  So what you do is you cache the results of your lookup. The Index is
the place to do it.

  I recommend making a printer template for index pages, keep the
template printouts in an informal ``blank papers'' or ``templates''
section in your common-store binder. That way, you don't have to keep
making these things over and over. (Note: This is something that I
have NOT done myself, but believe would be a good idea. Perhaps when I
put this online, along with accompanying format pages, I'll make an
index as well, in both 1-page and 3-page format.)

  To start an index for a small subject, just do this:

\begin{itemize}
\item Get a blank piece of ruled paper.
\item Write ``Index'' at the top.
\item Put the letters A-Z on the left.
\end{itemize}

  Now, whenever you start a search for a subject, pull out the index
sheet. Write, IN BLUE (because your search word is a ``key''), the
search word, NEXT to the letter that the word starts with.

  So for example, if I'm looking in my Social Ideology subject for
everything having to do with ``Anarcho-Socialism'', I'm going to put
that word next to the letter ``A''.

  Then start your search, beginning over the MOC. If you don't find
what you are looking for (MOC miss), then you may have to start going
page by page (generally happens when your MOC isn't up to date.) Write
the results (the ``hits'') as you find them. Then put the index page
back.

  Sometimes the page you were looking for, wasn't even in the subject!
That's okay. When you find the page, link to it from the same index
you started the search from. Your brain is messy. The notebook system
is messy. It's GOOD, it's USEFUL, it WORKS, but part of the reason it
works so well is because it TOLERATES ERRORS. (If you don't tolerate
errors, you are going nowhere with this all.)

  By the way -- in case I forget to mention it later: It's good to have
an index like this for YOUR ENTIRE Notebook system, as well. Keep the
Index at the front of your common-store binder. Link words to subjects
that they are featured in.

  You can start with one page, but as you search, and as your subject
grows, you may need to expand to a 3-page index.

  Just give each letter three lines.

  Symbols and numbers:
  I put numbers after Z.
  I also put all symbols after the numbers.
  Depending on your symbol system, you may be able to find a way to
form a hash key (perhaps a circle, if half your symbols involve a
circle), you may not. Just part of the trick of dealing with symbols.

  That's it for indexes.


  \theme{Cht  -- CHEAT SHEETS!}

  Cheat sheets are great!

  I do this a lot in my computer notebooks.

  You take your most commonly used pieces of information, and fit them
all onto a single cheat sheet. You use it to work with. It will
probably be highly abbreviated. Good organization is
important. Information density to the max.

  The cheat sheet goes near the VERY END of the subject, so you can
find it quickly.

  If you have several cheat sheets, you may need a TOC for them. You
can put the TOC in front of the cheat sheets (that is, near the back
of the subject), or with the rest of the TOCs near the front of the
subject -- it's up to you, really.

  The cheet sheets are followed by:


  \theme{A/S      -- abbreviations, shorthand}

  The A/S is placed at the far end of the subject, because you will
want quick and ready access to it. You will probably want to even take
it out of the binder, if you will be writing a lot.

  As you invoke abbreviations, or create symbols or shorthand, record
them on the A/S. That way, in a year, you can figure out what in the
world you were talking about.

  DATE YOUR A/S sheets. It's important, so that if you stumble across
an old POI, and you find a symbol that's not on the present version of
the A/S, you can go back into the archives and find which A/S was
relevant at the time. If you can remember to, FREEZE your A/S sheets
as well: Write the date that you stop using the particular A/S sheet.

  A/S sheets frequently look like index pages -- you have the letters on
the left, and you write keys in blue and values in black.

  You will probably want a ``global A/S sheet'', that you keep in your
carry-about binder, and that applies over all notebooks. Mine
personally is 4 pages long, with different sections.

  It has entries like ``t: To'' and ``f: Of'' and ``w: With", ''WO:
Without``, as well as a person table ''NH: Napoleon Hill NC: Noam
Chomsky ME: Michael Ende".

  By the way -- I should probably put this somewhere else, but when you
are dealing with English names, you get good packing with the
following division:

\begin{verbatim}
AB
C
DEF
GHI
J
KL
MNO
PQR
ST
UVWXYZ
\end{verbatim}

  Don't believe me? Try it! You'll be amazed how evenly names fill
into it.

\pause

  Now.
  I've described the major segments that I use:

\begin{description}
\item[P and P]     purpose and principles
\item[Speeds]  speed thoughts
\item[SMOC]    subject map of contents
\item[POI]     point-of-interest studies
\item[RS]      research
\item[REF]     reference
\item[PJ]      project
\item[I]       index
\item[Cht]     cheat sheets
\item[A/S]     abbreviations, shorthand
\end{description}

  I now want to describe segments that I am EXPERIMENTING WITH.
  That is, what MIGHT WORK.

  Remember: There's NO BINDER POLICE.
  And there's NO ENORMOUS INSTITUTION telling you what is cutting edge
and what is not, what you can research and what you can't, and who
will ignore you without proper credentials. Nothing of the sort. (And
if there were, you may have good reason to ignore it or no.)

  So make shit up, and post to the web the results of your
experiments. At the very least, email me. I'm interested.

  The FIRST experimental segment is ``X''.

  That enables you to experiment within a segment, without worrying
that the rest of your notebook system will fail.

  The special thing about ``X'' pages is that they are temporary, and it
signals to you: ``DO NOT LINK TO THIS PAGE!'' Because it might be gone
later. It's extra-volatile. BOOM! Your notebook just went up in
flames! AUGH!!!

  If you find that you are relying on your experimental pages later
on, than that's pretty good. That means it isn't really experimental
any more. Just white out your X's, or cover them up with a big blob of
ink, or turn the X into a star, or something like that.

  Now, there are four particular experimental segments I am working
out, lately:

\begin{description}
\item CEP    chronological episode
\item TD     topical deliberation
\item DD     data dictionary (definitions)
\item L/T    lists and tables (high info density)
\end{description}

  We'll go over each in turn:

  \theme{CEP  Chronological Episodes}

  I've always had subjects that were NOTHING but your traditional
diary/journal.\footnote
{What? Nonesense you say? That's not according to the
system? Hey! Fuck you! There's no binder police! Tolerate errors!"
}
Most notably, I do that in the ``Strategy'' subject, which is
intrinsicly temporal.

  But we can standardize the concept for other subjects that need it,
and we can make it stronger as well.

  For example, the events in our lives aren't just individual frames;
They frequently belong to threads, or ``episodes.''

  So I have been experimenting with creating GANTT style charts that
map out episodes. You write the ``TOC'' (no, not a MOC, because time is
intrinsicly linear) side-ways, and ``up-down'' identify threads,
``across'' identifies time.

  Your first thread should be ``unthreaded'' or ``non-episodic.'' Anything
that is not episodic, or theme bound (``how I feel today'') goes in
that top band.

  For your themed entrees, or progression in episodes, you use the
lower bands.

  So that's something I think is worth looking at more.

  (Presently, I am in a mobilization phase, not a vision phase, and am
thus NOT working my binder system, so I don't have a chance to try it
out right now. Let me know how it works and how it doesn't work.)

  NEXT:

  \theme{TD  -- Topical Deliberation}

  Remember how -- when we're constructing a new version of a map, or we
are focusing on where to place something-

  We're making a bunch of judgements. Some times we even use paper to
help us form those judgements. I would call those things "topical
deliberation."

  I'd store 4-6 topical deliberations to a page. This is sort of like
speeds -- we can keep 45 speeds on a single page, and id them
individually. Similary, I'd put TD1, TD2, TD3, TD4, and TD5 on a
single page (they are usually brief). These are our hard-earned
``judgements'', so we can refer back to them in the future.

  Then mark them on your MOC. They don't need a single ``spot'' though-
the topical deliberation usually applies to a REGION or an AREA. So I
think it'd be best to put a dashed line around the region of
controversy, that the TD clears up, and then label it with the TD. I'd
put it in blue or green (since the magnet words and structure lines
are already blue). Not red -- it attracts to much attention. And it's
not normal content -- definitely not black. Green is nice and easily
ignored. You can consider it GREEN as in ``markup for the map.'' Which
it is.


\theme{DD  -- Data Dictionary}

  If you do this long enough, you'll find yourself making up
words. And the meaning of those words may change (sigh), or spawn off
new words (better).

  (Confucious thought our problems came from shifting language -- words
meaning something other than they meant. Ted Nelson experimented with
interpreting this literally, but ran into some problems with it -- too
many new words constantly springing into existance, I believe it was.)

  I'd keep track of this in a data dictionary. It would probably not
be A-Z, since definitions are BIG, and we don't want to partition one
page per letter -- that's 26 pages, of which you may only use a few.

  Probably best would be to just add entries as you think them
up. Then, periodicly, type it all into a computer, alphabetize it,
print it out, and then stick it back in your notebooks. Keep additions
on new pages, and modifications in red. Then go back to the computer
when the time comes, and lather, rinse, repeat.
\footnote{%
Much better way!

Partition your ``dictionary'' pages into 4 squares. A single data
dictionary page has 4 DD (or maybe use ``DEF''?) entries.

ex: First pages has DD1-DD4, next has DD5-DD8, then DD9-DD12, yadda
yadda yadda.

Now what you do is have a hash TOC for the DD's..! You just put the
alphabet on the left side of the page -- your keys -- over 1-3
pages. Start at 1, then later expand to 2, then later to 3 -- and then
set your keys up as the words that are defined, and the values to the
DD\# of the word defined.

There you go. 
}

You can also link to the DD's from the SMOC wherever it would be
helpful to be reminded of particular terminology. Yah!

This is the way! (Yes; It's FF Tactics.)


\theme{L/T  -- Lists and Tables}

  I frequently find myself maintaining lists and tables within POI
entrees. These are things I access frequently, and would probably best
go near the end of the subjects, to share space with cheat sheets.

  At the very least, they just don't ``feel like'' POI. So I want
another section for them.

  L/T is my answer to this tugging feeling.

  I have not tried it yet.


  SO!

  There you go! Those are the segments of the intra-subject
architecture!

  We've talked about physical pages layout (may want to reread, now
that you've seen the segments logicly described), and we've talked
about the segments themselves.

  We're half-way through the book! Woo Hoo!

  Coming up next, is the Extra-Subject architecture.

  After you understand that, you have everything you need to start
working this system..!

  Then if you want to keep reading, you can read the "Theory of
Notebooks``, and "The Question of Computers.''


\chapter{Extra-Subject Architecture}
Topics:

\begin{itemize}
\item the physical representation of the complete system
\item GSMOC  -- the Grand Subject Map of Contents
\item Subject Registry  -- hash of all subjects
\item Subject sectioning  -- how subjects are made, gestate, interconnect
\item Process of Constructing and Linking a New Subject
\item Special Subjects  -- Chrono, Strategy, Zeitgeist Tracking, and others
\end{itemize}

These things bind everything together in the system.

THE PHYSICAL REPRESENTATION OF THE COMPLETE SYSTEM

You have three major categories of binders:

\begin{itemize}
\item Your carry-about binder (medium)
\item Your common-store binder (enormous)
\item Your archival binders (size irrelevant, but enormous is good)
\end{itemize}

Your carry-about binder will contain the following:

\begin{itemize}
\item Blank Paper
\item Blank Speed Pages
\item Blank Pan-Subject Speed Pages
\item GSMOC
\item Subjects Registry
\item Subject Speeds and Refs TOC
\item Global A/S
\item Optionally, 1 or 2 subjects (``locally cached subjects'')
\item Other Stuff
\end{itemize}

Your common-store binder will contain the following:

\begin{itemize}
\item Blank Paper
\item Blank Template Pages (such as indexes, maps, tocs, etc.,.)
\item Subject P and P's
\item Chaos
\item Unplaced Pages
\item As Many Subjects as you can Stuff In There
\end{itemize}

Your archive-store binders will contain the following:

* THE REST OF THE SUBJECTS.

Archive-store binders are kept by alphabet ranges.  For example, if
you had two binders, they might be ``A-M'' and ``N-Z'', depending on how
you decided to balance them.

Chaos is stored in the archive-store binders under ``C''. It might be
big. Every now and then, you might just want to dump out whole
sections of ``Chaos.''

What IS Chaos?

Chaos is just papers that have hoards of thoughts on them with no
obvious subject placement, or that are so hopelessly beyond recovery
(or take so LONG to recover) that you might likely just throw it out,
but that you'd like to give it ``one last chance''. After staring at it
for a while, though, you decide, ``Nah. Toss it.'' And you do.

Or you don't. I've occasionally found a jewel in there. Whatever you
like.

The Archives are pretty straight forward, you just store stuff in
them.

Now let's consider the common-store binder.

I frequently call it the ``subject cache''. It's where the subjects that
you have been using for the last 2-6 weeks go; Ones that you are
accessing frequently.

Whenever you need some space in the common store, you can do two
things:

\begin{itemize}
\item Put archival pages in your subject on interest into a corrosponding
  subject in the archive. You have fewer pages now, AND a more nimble
  subject. But you can only do this if you have archival pages in your
  present subject.
\item Put a whole subject, ideally one you haven't touched in a few weeks,
  into the archive. Done! This is the most common way of clearing up
  some space.
  This includes ``CHAOS''. Just take all the chaos pages out and put
  them into the archival's CHAOS section.
\end{itemize}

You may need to buy (or scrounge around for) another archival binder,
but there you go.

Now, in your common store, what are we keeping again?

\begin{itemize}
\item Blank Paper
\item Blank Template Pages (such as indexes, maps, tocs, etc.,.)
\item Subject P and P's
\item Chaos
\item Unplaced Pages
\item As Many Subjects as you can Stuff In There
\end{itemize}

Put those sections in the following order:

\begin{enumerate}
\item P and P
\item Paper (Blank first, then Templates last.)
\item Unplaced
\item Chaos
\item Sections
\end{enumerate}

Most of these you are already pretty familiar with. I'll talk a little
bit more about P and P and ``Unplaced'' here:

\theme{P and P:}


You want EVERY SUBJECTS P and P here, arranged alphabetically. Almost all
are 1 page only -- I've never seen a two pager. If you have older
version, the older version should NOT appear here -- only the most
recent. (The older version get's an archive bit set, and you throw it
in either the subject's archive back pages, or into the archival
binders.)

That's pretty simple.

\theme{Unplaced:}

These are for ``proto-subjects'': You have a few pages on the subject,
but you don't have quite enough pages to warrant actually putting a
tab-delimiter in place, using some of your template pages, and what
not. You just want to give the pages a temporary rest stop, and see
what happens. Maybe one day you'll have some more thoughts on the
subject, maybe you won't.

Important considerations: DO NOT NUMBER THE PAGES. There IS NO SUBJECT
yet, and thus no pagination.

HOWEVER, you DO want to CLEARLY IDENTIFY the subject that they \emph{would}
go in, both at the bottom of the page (as if it were a full page id,
just without anything more specific than the subject name itself), and
at the top-left of the page. (I don't know why. It just seems to
``work'' for me. You don't really have to.)

When an ``unplaced'' subject reaches a sufficient size, say 7-10 pages
roughly, then take your pages out and promote the pages to a full
subject. TaDa! Start a MOC, maybe a relevant TOC, fill out the tab,
update the GSMOC, registry, and you're done.

There -- we're done with the common-store notebook -- ALSO a relatively
simple subject.

And now for the carry-about notebook.

This is a MID-SIZED binder. It's not one of those really ultra-thin
binder, but it's not a gigantic daddy-long legs binder
either. \emph{mid-sized}. It needs to be COMFORTABLE to carry.

IT MUST BE DURABLE. Moreso than the others must be.

This thing is going to get pretty messed up. You're going to be
constantly fiddling with it. It is going to get dropped. Lost. DROOLED
ON.

Keep it safe. PROTECT IT WITH YOUR LIFE. You're going to hold an
incredibly vast amount of information in there -- your latest speed
thoughts on every single subject. That's information density. And
you're going to be carrying it into a hostile environment: THE
WORLD. The rest of your binders -- the common store, the archive, they
are not going to be in the world. They are going to be in a pocket
universe called your book shelves. But your local cache, your
carry-about notebook -- that's going to be with you in every where,
except where there's water.

It contains:

\begin{itemize}
\item Blank Paper
\item Blank Speed Pages
\item Blank Pan-Subject Speed Pages
\item GSMOC
\item Subjects Registry
\item Subject Speed Pages and Refs TOC
\item Global A/S
\item Optionally, 1 or 2 subjects (``locally cached subjects'')
\item Other Stuff
\end{itemize}

Here's their order in your binder:

\begin{enumerate}
\item Other Stuff (?)
\item Blank Paper, Blank Speeds, Blank Pan-Subject Speeds (in that order)
\item GSMOC
\item Subjects Registry
\item Subject Speeds and Refs TOC
\item Global A/S
\item MORE ``Other Stuff'' (?)
\item Optionally included Subjects (?)
\end{enumerate}

Question marks mean ``optionally present.''

Let me start with the easy ones first, then tackle the complex things
last.

\theme{Other Stuff:}
I have a hybrid between this notebook system, this ``thought'' system,
and the GTD system (``Getting Things Done'', David Allen). I call the
combination ``ACTS'' -- the Act-Communication-Thought System. It requires
pages in my carry-about binder. They go either at the beginning or the
end.

Frequently, people will hand you stuff, like fliers, or whatever. Just
keep it at the back or front of your binder.

Blank papers -- nothing needs be said. Keep it well stocked.
Global A/S -- we've already talked about this, in the Intra-subject
architecture. This is the same thing, just for things that apply to
your entire notebook system, or for roughly more than 2-3 subjects.

\theme{Optional subjects}

Some times you are busy performing some maintenance operation (mapping
speed thoughts), or are working on a POI or something, and you just
CAN'T leave the subject. So you're going to have to carry it with you.

Easily done! Just open your common-store subjects, pop out the subject
you want, and place it into your carry-about notebook, at the very
end. Work on it while you are away, over the objections of your
girlfriend, and then when you get back, you can pop it right back into
the common store.

Now we're just left with ``hard stuff'':
\begin{itemize}
\item GSMOC, Subject Registry
\item Subject Speed Pages and Refs TOC
\end{itemize}

We'll tackle subject speed pages/refs TOCs first.

Things to keep in mind:
\begin{itemize}
\item Subject speeds and references are together.
\item Organized alphabeticly.
\item Only LATEST speeds,...
\item ...but ALL REFERENCES.
\end{itemize}

That is:

You take all of the latest speeds and all of the reference TOCs for
ALL of your subjects. Then you arrange them ALPHABETICALLY (fully-
expand out your acronyms) by subject, in pairs (speed-references
pairs). The latest speed page comes FIRST, followed by ALL of the
references.

Or, not. You could just keep the references in their respective
subjects, most likely in the common-store or the archives. But I
prefer to keep the references with me in my carry-about, so that I can
add to them when I talk with people, and so that I can share them with
interested people.

Or you could keep the references seperate from all of the
speeds. However you like. This is just the way I've done it.

It's important to remember that only the latest speed needs to be in
the carry-about notebook. Since there is a higher risk to data that's
in the carry-about, I try to keep as much as possible in common-store
or archives. Speeds are VERY dense, so all the more reason to be
careful about them..!

Now, that wasn't so hard. Finally, the GSMOC and subjects registry.
Then we get on talking about:

\begin{itemize}
\item Subject sectioning  -- how subjects are made, gestate, interconnect
\item Process of Constructing and Linking a New Subject
\item Special Subjects  -- Chrono, Strategy, Zeitgeist Tracking, and others
\end{itemize}

So. 

\theme{The GSMOC and Subjects Registry.}

\begin{itemize}
\item What they are like, what they look like.
\item Placing subjects on the GSMOC.
\begin{itemize}
  \item proximity
  \item tight vs. loose connections
  \item mental association, \emph{not} logical connection
\end{itemize}
\item ``1/2 subjects''  -- subjects without contents.
\item What to do with the Subjects Registry.
\end{itemize}

The Grand Subject Map of Contents (GSMOC) is a map of all of your
subjects. It will probably start out being just a single page, and
will likely baloon out to be several pages large.

You'll use it for a number of purposes, not limited to:
\begin{itemize}
\item Finding subjects for your thoughts.
\item Discriminating the subject to place a thought into.
\item Organizing the subjects of your thought.
\item Reorganizing the subjects.
\item Locating thoughts that you've thought before.
\item Keeping track of subjects.
\item Being amazed at. Reflecting on who you are and what you are
doing. (Or not doing, as the case will likely be, at that particular
moment.)
\end{itemize}

The GSMOC isn't alone; It has the subject registry right behind
it. That is, two pages (at least), with a hash of your subjects over
their first letter. That is, you have ``ABCDEFG...'' written down the
left side, one letter to two lines. As you create subjects, you'll
list them in the hash.

You want a LIST of all of your subjects, in addition to your map.
Why? Because you're going to have some sorts of tags that you are
going to attach to subjects, that should be accessible in the context
of the map. The map is AWESOME at helping you find stuff by area, by
field, but if you already know the NAME of your subject, and are
holding that NAME in mind, rather than a vague intention, the LIST is
going to help you a lot more. Sort of like the difference between the
general google index lookup (you know -- the one that everyone uses and
that hangs out on their front page), and the google directories (which
are actually DMOZ).

You may well have some subjects that are NOT MAPPED. I listed those in
GREEN on the directories. In particular, I have a holdover from
previous days (pre-system) called ``Lists.'' It shows up in green on the
directories, but nowhere on the maps.

(You don't have to do it that way. Nowadays, with what I know now, I
would just make a place on the maps for ``unplaced'' or ``unplacable'',
and parked the ``subject'' name in there.)

The ``Unplaced'' section itself (!) -- for pages that are unplaced, also
appears in the Grand Subject Registry (GSR). Again -- it could go on the
map too in a ``unplacable'' location (I might even draw a little
warehouse picture next to it, or draw a little empty car-parking-spot
``U'' around it, or something like that), but I happen to have put it in
the GSR.

I also keep track of my archived subjects in there, as well. For
example, I used to have ``ACTION'' and ``REBOUND'', before I started the
present notebook system, which has assimilated and canibalized them
somewhat. They appear on my GSR, with a little red ``old'' character
(the Chinese character I showed before) next to them, indicating "You
don't put stuff in here, it's old.

Incidentally -- they also appear on the GSMOC. They are drawn faintly,
and also have the ``old'' character next to them. But it is not my
promise with myself that it will be the case -- My promise is that every
subject has an appearance on the GSR. But they don't need an
appearance on the GSMOC. It's just a good idea.

As you work with the system, you will feel the desire to have a GSR -- a
place where you can flag subjects conveniently, all in one place, but
without using the GSMOC. You'll LOVE your GSMOC, but you'll want the
GSR as well.

So, I am done talking about the GSR.

Now, what does the GSMOC look like?

Well, it depends on if it is small or big. GSMOCv1 will be so puny, in
fact, that it should include the GSR, on the same page..! That's what
I did.

The top 2/3 of the page represent your map.

GSMOCS \emph{ALWAYS} get convoluted with time. In fact, this applies to any
map. Then there's a revolution -- you toss the old map (well, archive
it), make an interim map (remember! give it a full version number!),
and then (perhaps) rewrite the interim into a nice new clean well
organized map. This happens with programming, too. Lots of systems. So
there it is.

Then the bottom third is just your GSR. It doesn't even have to be
alphabetical, because it's so small. Just hunt and peck at what's
there.

When one page is no longer enough, you'll do what I said above. And
remember: The FIRST page of your new, multi-page map, is just a map of
your maps. I drew the small maps in miniature on the front page. Sort
of like maps of the USA: The first page is the whole usa, and it shows
you the subdivisions over the maps to the rest of the USA. And they
usually have this special cutout for Hawaii and Alaska. And that's
what your first page of your GSMOC looks like. Include little page
numbers, in blue, to the actual pages themselves -- the page numbers
have got to \emph{stick out.} You could even put it in red, though blue is
more consistent with the color scheme I've described. (Blue:
Structure, Page Numbers, etc.,.)

Now remember: Your GSMOC is also a major strategy point. You're going
to be putting those little tabs on the page. So size your subjects
accordingly. We want DENSITY, but we ALSO want to be able to see the
strategy tabs. You'll know how big your strategy tabs are, once you've
bought them.

Ah -- I knew I was fishing around in my mind for something.
Okay: We're moving along to placing subjects on the GSMOC.

\begin{itemize}
\item proximity
\item tight vs. loose connections
\item mental association, NOT logical connection
\end{itemize}

Then we'll talk about half subjects.

I sense that I am missing some things. Unfortunately, I cannot
articulate what they are. My notebook system has done what it can for
me here. The notebook system works -- it works GREAT, but it doesn't
always give you EVERYTHING, and sometimes you'll feel that there is
``something missing''. Sometimes the feeling is wrong, but often it is
right. Even the notebook system, an amazing and wonderful catch, even
it, is bounded by the laws of time and priority. It works far better
than not having one at all, but it is not immortal or omnipotent. If I
was carrying out my notebook today, there would be two results:

\begin{enumerate}
\item This section WOULD be complete, in the notebook.
\item You would not be reading this, because I would be
immobilized. Because of the way the notebook system works.
\end{enumerate}

Remember the dialectic between INTENTIONAL thought vs. CAUGHT
thought. The INTENTIONAL thought is you actively going out into a mine
and getting yourself some thoughts. That's what I'm doing now, as I'm
writing. I'm using the notebook to GUIDE me, based on the content and
structure accumulated before.

I ASSURE YOU: If I had NOT been keeping my notebooks, this document,
that you see in front of you now, would NOT have been possible to
write. I might be able to write some thing here, and a little
something there, but it would be nowhere NEAR as structured (yes, I
know the structure is not visible yet, but it's there), nor nowhere
NEAR as complete (even with the tiny things that I miss), as it is
now. You'll experience the sense of ``completeness'', as well as the
hightened sense of ``incompleteness'', as you practice the notebooks,
and envision Xanadu\footnote{%
Ted Nelson's name for the infinite mental
noosphere database; I \emph{think}. I may have misunderstood what he
meant.}
in your mind.

(DIVERSION ALERT! DIVERSION ALERT!)
Thinking about Information Architecture will be EXTREMELY important to
society in the future. All these programmers wondering, "Why aren't we
reusing each others components?" Yes, very significantly, our
languages and practices are limiting us. Quite severly. But even if we
had the best reuse languages mechanisms and what not, we STILL need it
to be easier to figure out what other people have written. The
RETRIEVAL problem is MASSIVE. Even with Wiki's and automatic lists and
Orielly books and stuff like that, it's STILL enormously
difficult. Just reading the Lex/Yacc book is extraordinarily
intimidating, if you're just trying to make a simple command language
for your app. Yes, we know that simple languages turn complex
quickly. But don't beat people up about it. Write a better explanation
of how to use Lex/Yacc. Or better yet, realize that its complex, and
write a simplified version, and then take the time to hook it into our
programmer's social system. And suppose you DO intend to write a
better one. What are you going to do: Spend two years reading every
guy's paper in the world on the subject? Or are you going to maybe
spend a few days looking, and then start? You aren't going to spend
two years looking for and reading all the papers. You're just going to
sit down and do it. And some guy somewhere is going to say,
``Outrageous! He didn't read my paper!''

So, there are several things here:
\begin{itemize}
\item We need collaboratively constructed and canonical MAPS. This is hard
to do but worth doing. Probably not strategic at the moment, there are
better PFTs to work on. (For example, figuring out how to just
colaboratively build a map with friends over a blank piece of paper,
and then writing about it, would be a good first step...)
\item We need a few canonical (perhaps competing, if they fork) ``bases''
for work.
\item We need the public to be involved -- indeed, PRIMARY -- for all this
stuff.
\item People need to teach shit visually, so people can get up to speed
and contributing as quick as possible. Thus, we need visual tools
and explaining tools light-years beyond anything we have now. Yes: I
realize this book is all text. Sorry! A good diagram could lop off a
ton of paragraphs here. But with today's tools and my skill using
them, it's much quicker for me to make a ton of paragraphs. YOU end up
losing here, because you have to read my tonnage.
\end{itemize}

I have a general picture of how all these things fit together: "Public
Field Technologies." I want to write about it in the future. It's not
on the immediate roster, though. Okay, back to wherever we were. Let
me figure this out...

Replay:

\begin{itemize}
\item proximity
\item tight vs. loose connections
\item mental association, NOT logical connection
\end{itemize}

Then we'll talk about half subjects.

You put your subjects on the map, like we described before about
maps. You put things that are sort of connected close to one another,
and those that are distantly connected far from one another.

This is based on YOUR MENTAL ASSOCIATIONS, not on any sort of ACEPOS
(Absolute Cosmic Eternal Perfect Ontological Structure.)
Examples. Examples are good.

\begin{verbatim}
------------------------------------------------------------

           Personal Subjects Map

                   Spirit
                      |(mind 1st)
          Ethics     MP.      Admonishment
                    /   \
 ---             Values  IMGN
 PPL)---.            \   /           (P*Ident)
 ---     `----------P*Psych
                      |
                   A.C.T.S.
     MTK            /  |
--- /   \          /   |                  Chrono
EDU)     \Notebooks    |                 /   \
---               `    |             P*Hist--Prsnl
                  |    |                      Records
                   `___|___
                  ( Systems)                GSMOCv2 2
------------------------------------------------------------
\end{verbatim}

I've removed some things which are difficult to draw in ASCII, such as
the fait ``Action'' and ``Rebound'' and accompanying ``old'' marker.

Also, this just doesn't do justice to the real deal, at all. Another
strike against computers with today's technology. But anyways -- I'll
try to make it clear.

The page marker is on the bottom right, it says ``GSMOCv2 2'' meaning
that this is version 2 of the GSMOC, and that it's \emph{page 2} in
particular.

I've titled this particular GSMOC page, though I didn't have to; The
title is ``Personal Subject Map". There are others: ''Communication
Subject Map`` (which is the page ``left'' of this one), ''Society Subject
Map" (to the left of communication), an untitled map beneath these
three, and beneath that is ``ComputerLand'', where I've got all sorts of
computer subjects.

The title is written in blue.

On the left, are two subject names with dashed lines instead of full
circles, indicating that the subjects aren't actually on this
page. Those are ``PPL'' (People) and ``EDU'' (Education), both of which
exist on the Communications page, again, to the ``left'' of the present
page.

I put ``left'' in quotes because the page is actually GSMOCv2-4. That
is, ``page 4''. Two pages after the present one. But it is arranged on
the front page of the GSMOC as if it were to the left. Think back to
the analogy of the USA map: California may well be West of Alabama,
but Alabama's map may show up first, if the map is presented in
alphabetical order. This is not a difficult concept.

I include the links to make the connections clear. You don't have to,
but I think it's a good idea.

``Systems'' is similar, but it's ``below'' rather than on the left.

Now, there are two special items on the map, ``Mind 1st'' and
``P*Ident''. Those are what I call ``half subjects''. I'll write more
about those in a moment, but basically, they are subjects that are
ONLY speed lists at this point in time, or perhaps they are
accumulating mass in ``unplaced'', but not yet a full, formally
recognized subject.

The two half subjects are not actually written on the page..! They are
written on the little strategy stickies. They have a fat red dot drawn
on them, to differentiate them from strategy stickies.

Half subjects, after all, may be retracted. They don't appear on the
GSR. They may be merged with something else. Whatever. They are in
flux. So they don't get inked onto the paper.

Actually, you know what -- I don't have much more to say about
half-subjects, so lets just consider that I've talked about them. Not
much more to them.

The half-subjects here are demonstrating what I was saying about
proximity and distance reflecting subject matter.

Mind 1\textsuperscript{st} is a speed list of thoughts pertaining to the notion that we
are primarily, in mind, mental beings. It has to do with recognizing
our identity as minds, that everything we experience, even the world,
is a bunch of symbols and mind-images and what not that are thrown at
us. I don't believe that minds are our \emph{core} identity (awareness),
but that we happen to be inhabiting them, and that our particular
configuration is unique to ourselves by various accumulations and
circumstances. We seem to be reshaping the world based on our mental
constructs; If we were omnipotent, or telekinetic or what not, it
would become very apparent to us that the world was based on our
minds, not on the environment that gave birth to us. So, all those
kinds of thoughts appear in ``Mind First.'' (My true calling, however,
could be titled ``Spirit first'', or ``Awareness First'', and is
encapsulated in the ``Spirit'' subject position.)

``Mind First'' is CLEARLY in the metaphysical domain. But I don't see it
quite as good as the eternal Spirit. There's a sort of semiotic rule
that above is considered higher, and so I am positioning ``Mind 1st''
above and to the right of MP (MetaPhysics), but beneath spirit. It's
closest to Metaphysics.

Now -- we're talking about Spirit and Ethics and Metaphysics and Values
and Imagination and what not.

Don't be frightened. Remember: I'm not saying that these connections
are REAL or anything, or even necessarily the logical best
connections. I'm saying that THESE are my mental ASSOCIATIONS. But
this is by no means an ACEPOS, or even an attempt at an ACEPOS. This
is just the way I hook things up in my mind.

There are times where I perform radical reconfigurations, and
everything changes. Not only in my notebooks, but sometimes in my life
as well..! The notebook is a MIRROR of whats going on in your
MIND. Manipulations in the notebook are manipulations in your mind,
and \emph{vice versa}. WHEN YOU ARE MANIPULATING YOURSELF, \emph{IT CAN BE A
GOOD IDEA TO DO SO BEFORE A MIRROR.} There is no brain mirror (well,
okay, surgeons differ, but...), but there is this notebook, which is a
mind mirror. So there you go.

Think about that. I mean, I do. I contemplate that a lot. I think it's
profoundly interesting. If it means nothing to you; Well, okay. But I
think that's fascinating. Amazing.

But No. I'm not making the Kaballah here. This is not Ein Soph and
Kethery and Chokmah and Binah and the Void or whatever. If you want to
study that kind of stuff, great. The notebook system can help. Make a
section called ``Kaballah'', study in there, and link it into your
system on the GSMOC however you like. But the system above is not
Kaballah. It's a mind map. If you feel there are cosmic connections,
fine. Maybe ``Personal Identity'', ``Chronology'', ``Personal History'', and
``Personal Records'' can be modelled by some abstract timeless pattern.

Wait a sec -- what am I doing? I'm ``poo-poo''ing the idea of using other
people's established structures. I shouldn't be doing that. It's true-
I don't want to alienate people who are very anti-spiritualist and
anti-religious, and things like that. At the same time, I don't think
it's right to say no to people who ARE spiritualists. And there ARE
fundamental patterns. We know this in programming for a \emph{fact}; We
call them ``Design Patterns''. Similarly, in life, there are patterns as
well, and people have codified them into legends, and they have moved
people in powerful and useful ways, and have helped people organize
and understand their lives.

So, I take that back. If you are a Kabalist, and you naturally
associate things through the mental framework you've inherited from
the Kabalists' tradition, DO so. That comes naturally to you. We are
mirroring YOUR mind after all. So you should do it that way.

I would just suggest that you not feel confined to that structure.

At any rate: I want to make clear to EVERYBODY that this structure
above is just a map of things I think about. I think there's a
``Metaphyisical'' quality to my thoughts about values, and certainly my
thoughts about Imagination. Similarly, ethics is related to values,
but it is kind of seperate, and the same between ethics and
metaphysics, but more distant.

This conversation isn't nonsense: I'm demonstrating the working of the
mapping process. I could say it more literally, but as I said: I'm
just spitting this out. So, my apologies if I have offended anyone's
sensibilities. MAJOR apologies to everyone for writing so poorly.

So, that was a lengthy demonstration of how subjects are placed, and
configured. It's a combination of semiotics and proximity/distance and
other stuff. And the maps you make will make the most sense to you. As
it should be. No ACEPOS here.

Now I think I have one thing to add.

Sometimes, a subject comes directly out of another subject. ``PTUI!''
One subject (like Metaphysics) got so many thoughts that were easy to
isolate (like Values), that it SPAT IT OUT. So you have two subjects.

Sometimes, until the two subjects get some more distance from one
another, I will notate the link between the two with something other
than a line. It denotes ``Tight Coupling.''

Loose Coupling:

\begin{verbatim}
 .---------.                       .---------.
( SUBJECT A )---------------------( SUBJECT B )
 `---------'                       `---------'
\end{verbatim}

Tight Coupling:

\begin{verbatim}
 .---------. | | | | | .---------.
( SUBJECT A )| | | | |( SUBJECT B )
 `---------' | | | | | `---------'
\end{verbatim}

It doesn't have to be horizontal -- that's just an artifact of writing
with ASCII.

However, do notice that I've placed them closer, and that I use a
bunch of perpendicular lines (perpendicular to the direction of
connection) to denote the link.

Sometimes, over time, the coupling loosens. On the next version of the
map, I turn denote the loose connection on the map.

And remember: Very loose connections don't even get a line at all.

You just rely on proximity alone.

Now:
I've described that links appear because there are associations
between subjects. But it gets better than that! Did you know that
there's a VERY good way to determine the proximity/connection type, in
a very \emph{objective} way?!? As in: A computer could figure it out? It's
true!

Think about it: What tells us about the connection between subjects?

The P and P! The Purpose and Principles Page!

The MORE COMPLEX the protocol is between two subjects, THE GREATER
THEIR COUPLING! Or at least, it \emph{tends} that way.

So if you have a bunch of rules in the P and P over how thoughts divide
between two subjects, that's a good indicator that those subjects
tightly coupled. If there's just a single P and P reference, then it
probably means a ``normal'' or ``distant'' link, but still one worthy of
an explicit line.

So, there you go. You can see P and P at work while you determine your
GSMOC tensions.

Oh: I want to make sure something is very clear:

VARY the length of your links. You WANT to VARY them. There's this
sort of ``pristine computer ethic'' going around, that we want precision
and equal lengths and stuff. I agree: We want precision. But we don't
find it in uniformity. (A manifestation of the design vs. iteration
tension in Computer Programming...) We find it in the variable. We
want to be able to vary the length of the connection. Things that are
more closely related -- pull them in. Things that are further afield,
but still connected -- let the go out a bit.

Okay.

Oh; It just occured to me. I recently observed something on my
GSMOC. There's a pattern that I now call a ``Subject Feeder''. That is,
I was working on a bunch of individual subjects (GKI, Anarcho-Science,
Communes, Meetings, Electronic Collaboration...) and was getting some
names confused. I was routing Speeds between subjects a lot. "Does it
go there? No, it goes over there... (much later) What was I thinking?
This doesn't go here -- it goes THERE... (much later) What?! Where the
hell does this thought go?!" It was confusing, and that meant that I
had a framework flaw. So I reorganized my maps, standardized my names for
things, untill everything was clear, and found I had a new subject:
``PFT'', or ``Public Field Technologies.'' It was DIFFERENT than my
conception of anarcho-Science, and it was DIFFERENT than my thoughts
about the global knowledge infrastructure, and it was DIFFERENT than
all these other things. And it turned out to be what I am calling a
``Subject Feeder''. That is: It did a REALLY good job of showing WHERE
to put other thoughts. Sometimes I'd even through a speed into the
subject feeder's speed list, because I hadn't quite internalized the
new organization, to be able to quickly place the speed, and maybe I
had important things to do at the time. So I just threw it into the
subject feeder bin. Or in this case, ``PFT''. Then as I processed my
speeds, with map in hand, I would forward the speed to where it needed
to go. (Computers come in really handy around now. Anytime you are
transcribing shit, you're always kicking yourself wishing for
computers. I have things to say about this in the ``Computer'' chapter-
chapter VI.)

So you may witness the same thing: ``Subject Feeders''. Or whatever we
want to call them.

Okay. So, we've talked about:
\begin{itemize}
\item What the GSMOC looks like, what the GSR looks like.
\item What you do with a GSMOC and GSR
\item placing subjects on the GSMOC:
\begin{itemize}
  \item proximity
  \item tight vs. local connections
  \item mental associations, not logical connection (``ACEPOS'', in the
    extreme.)
  \item connection with P and P
\end{itemize}
\item 1/2 subjects
\item ``Feeder Subjects'' pattern
\end{itemize}

We've also talked about things I didn't plan originally and don't
really ``fit'' here:
\begin{itemize}
\item use of metaphysical structures.
\item information architecture's future.
\item more on PFTs.
\end{itemize}

So... It's talked about! We're done!

What else is here in the Extra-Subject Architecture..

\begin{itemize}
\item Subject Sectioning  -- how subjects are made, gestate, interconnect
\item Process of Constructing and Linking a New Subject
\item Special Subjects  -- Chrono, Strategy, Zeitgeist Tracking, and others.
\end{itemize}

Okay, just a moment. It's hot here in Seattle, this summer.. I need a
glass of water. Done.

So, Subject Sectioning.

Let's see -- what's on the map.

What follows is just gibberish in my lookup process:
\begin{verbatim}
--------------------------
S5 S14 S19 S82 S43 S44
Transfer: P2 S74
Interface: S15

S5: Purpose, Mis-sectioning, Re-sectioning, Growth Process, Can get
right first time?, Constellations (SYS:S4)
SYS-S4: (Carry-about, Systems, S4, looking up...)
SYS-S4: Programming and Bsns and Notebooks have in common- diving up
functionality/work/thoughts into constellations. Dealing w/ border
cases, ambiguity. Ideally multi-categorize.
S14: Dividing when big- many speed-thoughts = division likely
S19: Splitting Subjects - when 1 domain knowledge used make conclusion
in 2nd, 2nd is primary. If many-many try go more general. If 1 object
many contexts, the object. ex: ACT2:S9. Particular relevence in a
domain gets it. ex:ACT2:S39. If 1 object in 1 context, the context, to
aid in mapping. Action suggested in particular domain gets
it. ex:ACT2:S26 (or28?)
--- (ACT2 lookups)
ACT2-S9: (Archives, Action 2, S9, looking up...)
ACT2-S9: Ideals, Spirit, and Self Image. Spirit in Context of ideals  and 
Self-image (where am, where go). [Sent to ``Spirit'']
ATC2-S39: It's okay to hold social theories, just don't believe you
are right. Understand your ideas are non-scientific,
inductive. Connected S37, Forwarded to SOC.
ACT2-S37: ``Give up being right!'' -Super-complexity, Non-scientific
thinking, (science IMPOSSIBLE), w/in abstract domains of political
issues in life. Much happier w/o.
ACT2-S26: Keeping sched for interacting w/ friends. -->ATS
---
S82:Subjects can become big.l Things ordinarily small may take on
great importance in your life, and grow own place in framework. ex: If
x-tian, Bible may be whole subj, or even individual books in Bible!
S43:Thought-focus areas can be TIGHT (ex:IMGN) or LOOSE (ex:MP)
S44:Identifying Key Locusts and Peripheral Locusts
Transfer
P2: (POI Transfer entry)
S74: Moving Entrees. Don't renumber if don't have to. Open spaces AOK.
Interface
S15: Interfacing between subjects of note-keeping, and P and P
--------------------------
\end{verbatim}


Ahem.

There's NO WAY that I'd just come up with all that just by sitting
here typing away. I'd be lucky to have recalled half of it. I'm
CERTAIN I would have missed ``Splitting Subjects'' -- a very important
ruling that I have so internalized, that I've forgotten I know
it. Okay, so, here we go. Serializing time.

One way of cutting it up:

\begin{itemize}
\item Thoughts in a Subject
\begin{itemize}
  \item Subject Purposes
  \item Constellations of Thought within the Subject
\begin{itemize}
    \item How this is like other things in life.
      Skill in one is skill in the other.
\end{itemize}
  \item Mis-sectioned Thoughts
\begin{itemize}
    \item How did it happen?
    \item Interface between subjects. (P and P)
    \item Resectioning Mis-sectioned Thoughts
    \item POI Transfer
\end{itemize}
  \item Splitting Subjects
\end{itemize}
\item Growth Process of Subjects
\begin{itemize}
  \item Can we get the subjects right the first time? (No.)
  \item Dividing Subjects
\begin{itemize}
    \item Why subjects get big
    \item When to do it
    \item How to do it
\end{itemize}
  \item Tight and Loose Subjects
  \item The Ultimate Stage of Growth  -- The Book.
\end{itemize}
\end{itemize}

I like that. I'll use that as the guide for discussing Subject
sectioning.

First: ``Thoughts in a Subject.'' How you get a thought, and then figure
out where to put it, and then how to move it from one subject to
another when things get mixed up.

You get a thought, and you want to place it in the subject that's
purpose it best aligns with. Most of the time, you know where to
put the thought immediately, but some thoughts are tough. Your first
line of defence is the GSMOC. You look in the general vicinity of
where the thought goes, and see what your options are. 9 times out of
10, this tells you what you need to know. Ocassionally, you'll need to
go back to your P and P, and figure out what you intend. When you decide,
you may need to annotate your P and P, especially if it was silent when
you needed its help.

Now remember -- your P and P is to be interpreted \emph{loosely}. It says
``INCLUDE'' and ``EXCLUDE'', but it doesn't say ``ONLY include.'' It just
says ``INCLUDE.'' Anything between INCLUDE and EXCLUDE is up to YOU to
think about. Otherwise, we'd be spending all day writing a single P and P,
or constantly updating the P and P.

But some times: Even with this ``loose'' interpretation of the P and P, you
will STILL not be able to place the thought.

THat's GOOD! There are three things this can mean.

\begin{enumerate}
\item You've got to put the thought on the ``unplacable list''. A
   mysterious list that I never gave second thought to, appears before
   all the speeds, and forgot to write about. Dou! Doesn't even appear
   on my notebooks SMOC.
   This is the case for ``out there'' thoughts, that have nothing to do
   with anything you've been writing.
   This is the boring case.
\item You're splitting thoughts. More on that below.
\item This is the exciting case. Your thought has ``broken'' your
   conceptualization system. There's something wrong with the way
   you've been thinking about things, and you need to go in and
   reorganize your thoughts, in such a way that it can account for
   this new thought of yours.
\end{enumerate}

It's a lot of work some times, but you'll find that the ``automatic''
processes of the notebook system are not the major benefits. YES, they
HELP, but they BRING YOU to the places where you can clearly see the
HARD work that needs to be done. Incidental thought is great, but
there's difficult work to be done in Intentional thought as well. (I
say difficult because: The process is not well understood (I'm working
on it), and because: It generally takes a while to do, especially
relative to Incidental thought, which requires minimal
``work''). Reorganizing can be time consuming, but the rewards are
enormous: You are moving to a whole new way of thinking about whatever
it is that is important to you! A revolutionary
change. Congradulations!

But it is still a lot of work.

So, you place your thoughts by purpose, and it may involve some
difficult cross-subject work, but for the most part, 99\% of the time,
it's a pretty easy process.

YOUR THOUGHTS BUILD CONSTELLATIONS WITHIN YOUR SUBJECTS.

This is a very important, and a very powerful metaphor. This is
something that I have found true in many fields. I believe this is
something that AI resesarchers should be paying attention to (though
they probably are already).

Oh: By the way. I think that anyone who is interested in AI should
read this book and practice it. This process is very, very, very
mechanical. That's boring to us humans, but computers love it. And I
think that computer AI would be performing a much more complex version
of the process. But I think that the basic framework \emph{may very well
be} this process, just more formal. Okay, I'm done pretending like I
know about the field of AI. (I don't.)

When you are building a computer program, you have all these little
``thoughts''. But they aren't thoughts: They are requirements, promises,
stuff like that. You can scatter them out, then organize. Collect
related ones. There's usually multiple ways to organize them,
different ways of considering the same exact thing. You put closely
related ones together, and that's called ``cohesion.'' And others are
apart, but connected, and that's ``coupling''. Sound familiar? They are
little constellations of requirement. The process is half art and half
science.

Same for business construction, where it's basically the same process,
but with people.

So, why mention this? Because it's general. If you know the practice
in one, you know it in the others; Here too as well.

Now: What happens when you mis-section your thoughts?

First, you want to think about what that means. Is this something that
you are doing a lot? Perhaps you need to clarify your organization, or
you want to set up some sort of process for thoughts in this domain,
until you've internallized the organization.

Next: Determine where the thought is going. Say you're moving
something from PFT (Public Field Technologies) into ECollab
(Electronic Collaboration). You made a mistake by putting it into PFT,
and it's a straightforward matter to move it to ECollab.

First, you put the thought in the ECollab speeds. Give it the next
available number. Then, cross it out of PFT, in red. Also in red,
write ``ECollab S57''. Or whatever subject and Speed number is the
target. Simple enough. You want to keep the record of where it went,
because you might have links to the original.

If you are ABSOLUTELY CERTAIN that you've NEVER mentioned the original
anywhere else, you've never linked to it, then you can completely
cross it out, without linking to the new location as well. But I
usually just link it; I don't like to worry about absolute certainty.

Your POI are a little more difficult. If you've written a POI, but
later find out that you put the entire POI in the wrong place, the
process is a little more involved.

What you do is you take the POI out of the first, insert an out-card
in it's place (pointing to the new location), and then insert the POI
in the new location. You also have some renumbering to do.

Now -- what's the new location?

If there's not already a POI with the given number, you can just take
that number.

For example, if I'm moving a ``PFT P7'' to be ``ECollab P7'', but ECollab
only has 4 POI (ECollab P1, ECollab P2, ECollab P3, ECollab P4), then
just let the POI\# remain the same.

Who cares if there are gaps? Put it in the 7th slot on the TOC, or
don't list it on the TOC at all, untill the 6th has been reached, at
which point you can list P7.

But sometimes you just have to renumber the whole thing to a new POI\#.

When renumbering, whenever possible, you want to CROSS OUT rather than
WHITE OUT, or eliminate. That is, you'd like to be able to see the
history, if at all possible, without fucking this up too much.

But, like all rules, some times you'll just have to say "history be
damned."

Look to the future.

Finally, in this area of placing thoughts in subjects and
resectioning, I want to talk about some tough nuts to crack.

I call it ``Splitting Subjects''.

This is a thought that isn't merely ``out there.'' NOR does it somehow
break your system.

It's just clearly highly relevant to more than one subject.

I have built some general rules for such thoughts, and you may want to
make up some of your own.

Now hold on -- this is pretty abstract. Don't worry, I'll give examples
of each. They are:

\begin{enumerate}
\item When 1 domains knowledge is used to make conclusions in a 2nd
   domain, the 2nd domain is primary.
\item If there is a many-many relationship among domains, try to go with
   the most \emph{general} domain, if at all possible.
\item If there is 1 object in MANY contexts, try to put the object in
   it's own, native domain. If such a thing exists.
\item But: If you have 1 object in ONE context, put the object in the
   CONTEXT.
\item If there is a PARTICULARLY relevant domain, that domain gets it.
   In order to aid in mapping.
\item An action in a particular domain: the domain gets it.
\end{enumerate}

Yes, that is pretty heady and out there. It's also fairly arbitrary-
you can and should make up your own rules, as you come up with
them. Just be consistent: Record them somewhere. (I would imagine that
you would have a ``Notebooks'' subject, like my own, or something like
it.) This is just something to go from.

So, examples.

\point 1) "When 1 domains knowledge is used to make conclusions in a 2nd
    domain, the 2nd domain is primary."

So, say you have some thoughts about ethical rules in electronic
collaboration. I happen to have both ``Ethics'' and ``ECollab'' in my
notebook. Which should it go in? Ethics, because I want to collect all
my thoughts about ethics into a coherent framework, or ECollab,
because I want to be aware of ethical considerations when I work on
Electronic Collaboration?

I put it into ECollab, and I use this rule to help me make decisions
like that. There is \emph{certainly} such a thing as thinking about Ethics
abstractly.

This is highlighting a limitation of paper systems: It's difficult to
put a thought in two places at once. Computer systems, when they
overcome their too-costly restrictions, will solve this problem for us.

When you obey this rule, you don't need to place a link on the
GSMOC. If you did so, it would quickly become a big enormous tangled
mess.

If you want to, you can run over to the Ethics subject and put a link
on the map to your thought way o'er in Electronic Collaboration. But I
wouldn't do that. Such a promise to yourself would be waaay too costly
to keep. We must live with the imperfect. TOLERATE ERRORS. If this is
hard for you, start fucking things up by attaching imaginary false
links in one place (I guess). Start making up links that go to
creatively unrelated places. That's my solution to your dilemma, at
least.

So: Applying subj 1 in subj 2, put thought in subj 2.

\point 2) If there is a many-many relationship among domains, try to go with
   the most \emph{general} domain, if at all possible.

HOWEVER, if you have some situation that is incredibly complex, with
multiple applicabilities, just go for the most general one you can
find.

This is just a messy situation. See if maybe another rule applies, if
you don't like this.

\point 3) If there is 1 object in MANY contexts, try to put the object in
   it's own, native domain. If such a thing exists.

So, I have an actual real example from real life here.
I had a thought like this: "Reflect on the role of Spirit in the
Context of Ideals and Self-Image -- where it is at, where it is going."

So, it was fighting amidst ``Spirit'', ``Values", and ''Personal
Identity." (Actually, at the time, Personal Identity didn't exist, so
technically, it was between ``Spirit'', ``Values", and ''Personal
Psychology". But that's just nit-picky.)

Where did that go? It went to ``Spirit''. The focus of the thought was
clearly on spirit, in these other contexts, so I put it in Spirit.

\point 4) But: If you have 1 object in ONE context, put the object in the
   CONTEXT.

This is just a special case of rule 1 (two subjects, go with the one
that's the context).

\point 5) If there is a PARTICULARLY relevant domain, that domain gets it.
   In order to aid in mapping.

I had a thought that "it's okay to have social theories, just
understand that all positions are fragile, and necessarily, by the
shape of things, non-scientific, inductive."

I put that in ``Society". (Later, it would have gone into ''Social
Ideology``.) I WOULD have placed it in "Science'' or epistimology, or
something, but I REALLY WANTED to see this thought whenever I thought
about society -- it's an important self-admonishment.

So, put your thoughts wherever they will serve best.

\point 6) An action in a particular domain: the domain gets it.

If you have a ``possible action'' that you can apply in a particular
domain, put it in that particular domain. This is again a special case
of rule 1: Put specific things in specific places.

A good general ``super-rule", I guess then, is ''Put specific things in
specific places, and general things in general places."

And \emph{that} is what I call ``Splitting Subjects''. I have the image of
throwing a stone thought tablet over a pile of rocks, and seeing where
it rests.

I guess it's a bad name. In future versions of this text, I can call
it something better.

So: We've talked about how thoughts roll over subjects. How they get
in, how they settle somewhere, how they get kicked out, how they go
somewhere else. This is all in the context of extra-subject
architecture. It doesn't make much sense to talk about this within the
framework of a single subject.

Next, I'm going to talk about the growth process of subjects:

\begin{itemize}
\item Can we get the subjects right the first time? (No.)
\item Dividing Subjects
\begin{itemize}
  \item Why subjects get big
  \item When to do it
  \item How to do it
\end{itemize}
\item Tight and Loose Subjects
\item The Ultimate Stage of Growth  -- The Book.
\end{itemize}

So, you have a subject. You pick a good, 1-3 word title.

It can be as specific or general as you like.

Frequently, you start out general, and then, with time, as you collect
thoughts, you extract items that are specific.

For example, I started with ``MetaPhysics''. After about a hundred
speeds collected, however, I pulled out ``Spirit'', ``Ethics'',
``Imagination'', and ``Personal Psychology.''

You pull them out by looking at your thought ``stars'': You'll identify
``constellations'', and you'll seperate out those that are clearly
seperate and distinct from the others.

Some times you'll pull out things that are still pretty
interconnected. That's still useful -- just document the
interconnections on the P and P, and you might want to draw in some
``wormholes'' on your maps, linking across subjects.

Now: ``Can you get a subject right the first time?'' That is: Can you
predict how your thoughts are going to come out?

I have found that it's rarely the case that I can do a very good job
of it beforehand. That's because of what I call the "symphony
principle." (If someone knows a canonical name for this, let me know,
lion@speakeasy.org.) It works like this: If you don't listen to
symphony music a lot, and then you hear some, you say, "Oh, it's
symphony music.`` And then if you hear another, you say, ''Oh, it's more
symphony music." There's little difference between the two pieces. But
if you listen to it \emph{a lot}, then you start to discriminate
better. You can start to identify major trends big differences. With
time, you can even identify subtle differences. With even more time,
those previously subtle differences become vast enormous chasms, and
you are picking out \emph{still subtler} differences. The same is true for
art. If you pay attention to an art style for a while, you start to
understand things that other people don't -- the meaning of a slightly
thicker line, a slightly different shade, a slightly different
position. Or clothing. Some people are experts at what clothing
communicates to ourselves, or to others, and they can make subtle
differences that we cannot cognize, yet still influence us. Lots and
lots of subtelty.

And the same thing happens with our notebooks. We approach a
subject. It has meaning to us, for some reason. We feel drawn to
it. But it is amorphous, we cannot divide it. If we \emph{could} divide it,
we'd be much cleverer people. But we cannot, at the moment. So we
approach it, and ``subscribe'' to that sound stream of thought. We start
writing it, and then graphing the night sky. We telescope in on it,
and find pinpoints of still smaller stars, and we identify
constellations, and constellations within constellations.

So: I don't believe we can just ``get it right the first time.'' We have
to listen to it first.

So: If in doubt, go general in the beginning. Then, extract subjects
within as you discover difference.

Now, how do you seperate out subjects?

Just as we have described before. You take the speed thoughts, and
transcribe them (argh!) to new speed lists. Make sure that when you
cross them off from the source speed list, that you include a note
about where it's going -- perhaps the initials of the destination
subject, and the speed \#. Move other resources as well -- REF, PJ, RS,
POI, everything. Place out cards that point to the new location.

It is tedius, but not nearly as tedius as keeping several subjects in
the same exact notebook.

Transcribing speeds is the most tedius. In the computer section, I'll
describe a program that, if written, could take 90\% of the notebooks
work off of our hands. Basically: Most of the work in the notebook
system is based on \emph{speeds}. But speeds are also the most
non-graphical element of the system! If we just program in a good
\emph{mapping} system, which requires some graphic manipulation, but not a
whole \emph{lot}, then we could enter the speeds from pan-subj's, quickly,
because we're typing, and then we never ever have to transcribe
again... 90\% of the work disappears in a puff of smoke. Periodicly
print out the maps and speed lists, hole punch'em, and put them into
the notebooks... But I'll talk about this program more in the
computers section..!

When you perform this pattern, of listening to a subject, and then
seperating out the constellations, you find yourself frequently with
what I call ``tight'' and ``loose'' subjects.

That is, the ones pulled out are ``tight'' -- that have a specific
focus. But the source subject -- that's still ``loose''. It serves as a
sort of ``anything that's not one of those things.'' The loose source
can serve as a source, \emph{yet again}, but attention distributes out to
the tighter subjects. Over time, the tighter subjects take on a life
of their own, and occasionally spawn off even more subjects.

I had Society, which had a child named "Global Information
Architecture``, which had children named "Anarcho-Science'' (about the
motion of traditional study from corp/gov/mil/university to the public
field) and ``Public Field Technology'' (about fields that can help
people benefit and organize themselves.) PFT then dramaticly reshaped
the GSMOC. So, these things take on lives of their own, mirroring what
occurs in your mind.

Sometimes, however, a subject reaches an ``fullness''. You feel good
about what is present. You do not want to lose it. It is reasonably
complete. You may have improvements or changes in mind, but you want
to preserve what you have made.

At that point, I believe, the appropriate thing to do is: To Write a
Book!

I mean, that's what I'm doing right now. My ``Notebooks'' section
reached a certain ``fullness''. Obviously, my notes were not perfect,
but ``intentional'' thinking (rather than ``incidental'') can fill out
some of the things I either felt was missing, or too obvious to me to
have noted (but that a stranger might not know about).

After you have written a book, you have a solid base for further
progress.

I think everyone should do this is they have a subject that is not
researched yet. Hell, there are even some justifications if something
\emph{is} researched yet.

You may see where this leads: I'm talking about not just the
integration of thoughts within \emph{your} mind, but \emph{between} us as
well. I'm talking about social organizing. Right?

One of the interesting things about the notebook system is that,
unlike most other endeavers, \emph{you don't really need a purpose.}
Napoleon Hill said that major efforts require "Definiteness of
Purpose". But what can we say? We live in a universe without clear
purpose. But maybe, buried within ourselves, is some motion, that is
pushing us. I don't know if this is ``purpose'' in the same way, but it
feels to me like there is some kind of motion, and that it is
expressing itself. I agree that this is metaphysical, but I am only
describing what I have observed. When I start keeping notes, I need no
purpose. You need no purpose to analyze the traffic in your head -- the
traffic of all these thoughts zooming about. But you start analyzing
them, and you start detecting patterns, and seeing what's
happening. And it's amazing. Perhaps you \emph{do} see a direction there.

You never had a sense of ``purpose'' before, but you \emph{do} see that there
is a direction inside of yourself, and you see it visually, mapped out
in front of you. Anything you think about -- maps out in front of
you. If you have an unconsious urge, it will appear, in your
notebooks. Low frequency patterns get captured and mapped. Whatever is
inside of you.

Perhaps I'm going off the deep end here, and perhaps I'm just seeing
something in myself, and overly generalizing it to all
people. Nonetheless, I wouldn't feel right if I didn't write about
what I saw. You need not believe me, nor agree, nor see the same thing
yourself in order to practice the notebook system.

But let me know if you have any thoughts on the matter.

I have big dreams for these ideas, but I also have difficulty seeing
this as anything more than an intellectual novelty item right now.

Now: Back to writing books.

You don't have to write books in order to show to other people. You
can just write it for yourself.

That way, you feel okay saying that the subject is ``done''. You can put
the notes in archive, and have no fears that your notes will be
incomprehensible to you a decade later.

But I would put it online, for people to see.

We have no way of mapping our thoughts together, yet, online. We have
google indexing, we have DMOZ, but these are not quite the same thing
as \emph{maps}, they are TOCs and indexes, they have great limitations in
making \emph{structure} clear. We talked about this already.

We also have the ``References'' problem to address on a social level -- that a
single reference covers many-many-many topics. That's why we sometimes, in
the REF TOCs, that we may pick out a single chapter of a book as a
reference, or distribute a single book's contents over several
subjects REF TOCs. Similarly, posting our books online, we will have
difficulty linking contents from a single source across multiple
maps. However, I believe we can and will link contents effectively,
far more effectively than we do so so far.

This book, itself, is a single element of what I call the PFT's. It is
on the personal end of the spectrum from individual to global
empowerment. (That is, if I'm not completely bonkers, and that this
system is actually of some use to somebody..! Though I would not be
surprised at all if this is actually a \emph{poor} system, and that there
are far better ways of doing things than I have described here.)

There is also the question of: "Are we amassing some knowledge,
laboring to build something, when someone elsewhere, whom we cannot
access, for whatever reason, has already far surpassed what we are
laboring to build?"

Was it useful for Indians to invent bows and arrows, when people across
the globe had guns? I think it \emph{was} useful for them. Because,
locally, within their sphere, it was better than anything they had
before. The people who are distant and inaccessible are just that:
Distant and inaccessible. We can only affect what we affect: Our local
sphere.

I believe we should take efforts to be as strategic and open and as
international as possible.

I believe that these thoughts are not original to me, that these are
ideas that many people have, everywhere. I don't like it when some
writer says, "Ah, these are my great ideas. Others have gone before
me, but these are my ideas, extending on those before." Many people
don't have the resources to articulate what they've had. I've seen
many times that it is clear in a field what steps need to be taken,
though not everyone is jumping to write about them -- it just seems like
common knowledge. Individuals and corporations jump to claim the
credit for themselves. I believe all of this will change in the
future. Not because it would be ``nice'' if it did, but because as our
ability to communicate, out here in the ``public field'', is increasing
dramaticly. The reasons why corporations ``work'' is because they are
organized by communications, dollars, and geographic proximity. But
proximity is becomming meaningless. And there are \emph{far more} people
out here in the public field, than the dollars can support in the
comparatively small private networks. Further, the communications gap
is narrowing -- with instant messaging and video networks, people who
just happen to be interested, but are across the globe, can organize
just as well as the guys at the company, struggling to find a meeting
time. So I believe it is very possible that we can, if we take action,
make our public field powerful.

But nobody knows the future.

I've diverged considerably; Most of this should have gone into the
``Theory of Notebooks'' chapter -- chapter V.

Sorry; I'm getting ahead of myself. If I take the time to reorganize
the book, or if a reader re-writes the book (you are welcome to do so,
please respect the public license if it applies, and feel free to
write about the same exact things in your own words, with your own
thoughts), then we can put these things in the right places.

We talked about subject sectioning, how subjects are made,
interconnect, gestate.

Next is a very brief section, about the process of constructing and
linking a new subject. FINALLY, we talk about special subjects, like
chrono, Strategy, Zeitgest Tracking, and others.

\theme{CONSTRUCTING AND LINKING A NEW SUBJECT}

This is just a checklist, for you to follow in your mind, as you
create a new subject. No need to be exact, of course. Tolerate errors.

\begin{itemize}
\item Make a new tab for the subject, and put it in the common store.
\item Put your pre-existing pages (for example, 1/2 subject speed lists,
  unplaced notes) in the new tab.
\item Place an entry for the subject in the GSR, and the GSMOC, in the
  carry-about binder.
\item Write up a P and P, and place it in the P and P section of the common-store
  binder, up front.
\item Take out one of your Subject Speed pages, and set it up for the new
  subject, supposing you don't already have one. Put it in the
  carry-about binder.
\end{itemize}

That's about it. I don't remember if there is anything more you need
to do at the time.

It doesn't need to be a big process. For about 1/2 of my subjects, I
don't have P and P pages. It works just fine; Many subjects are
self-explanatory, and not intertwined with other subjects.

Finally:

\theme{SPECIAL SUBJECTS}

\begin{itemize}
\item Chrono
\item Strategy
\item Zeitgeist Tracking
\item and others
\end{itemize}

There are some subjects that are special.

Sometimes, you have subjects that cut across several subjects, such as
``strategy". That's where I've collected my ideas about ''What should I
do? What would be most strategic?" Within individual subjects, you may
have strategy ideas as well, particular to that individual subject.

My strategy subject was just a TOC and a bunch of chronologically
sorted pages, numbered simply ``1,2,3...''

My chrono subject was the same way, except instead of being about
strategy, it was basically just a standard diary or journal: How I
felt, what was going on around me, etc.,.

I also kept in my notebook my calendar month pages (from my
implementation of the GTD system), as months passed, and my Zeitgeist
lists.

``Zeitgeist Tracking'' is an interesting thing that I did. At the start
of every month, you make a blank sheet, and write "Zeitgeist, Mar
2003" (if this is a Zeitgeist sheet for the month of March, 2003.)

Whenever I found myself in a subject for a while, I'd put a brief
2-3 word description of the type of thing I was thinking about, if
it wasn't already present on there.

At the end of the month, you have a list of about 10-20 general items,
showing what you've been thinking about. You store the page in the
Chrono notebook as the month rolls over, and you can look back and see
how your thinking has changed over time -- what you are thinking about,
what is important to you, etc.,.

If I were to do this all again, I would probably make up a special
segment for the Zeitgeists (``ZG''?) within Chrono, and I'd use CEP (the
experimental Chronological Episodic segment) to track threads through
time. It can definitely be structured better.

There are no notebook police: You can make up your own special
subjects. \verb| {:)}= |

Done! We've talked about the physical representation of the complete
system, the GSMOC, the subject registery, how subjects are made,
gestate, interconnect, the process of constructing new subjects, and
the special subjects. Woo Hoo!

\chapter{Theory of Notebooks}

In this chapter, I want to talk about:

\begin{itemize}
\item Why Notebooks? What are we trying to do?
\item How do the notebooks work on an abstract level?
\item Ideas about thinking.
\end{itemize}

This is a short chapter; I've already intertwined many of the ideas
about how things work in the previous chapters.

\theme{WHY NOTEBOOKS?}

Major reasons:

\begin{itemize}
\item Reduce Repeated Thinking
\item Prevent Lost Porgress
\item Observe Mind
\end{itemize}

Minor reasons:
\begin{itemize}
\item Storage and Retrieval
\item Living Strategicly
\end{itemize}

``Repeated thinking'' is when we think the same thoughts, over, and
over, and over again, without really getting anywhere.

``Lost progress'' is when we have solved a situation before, but we go
back and return to it.

Frequently we find ourselves in one frame -- a frame that we've already
solved before. Surely, if we kept records and mapped our thoughts, we
could keep a map in mind, and identify our position on the map and the
transitions to solved states.

This applies technically; It may be possible to apply psychologically
as well. My technical attempts have been very successful, but my
psychological attempts have had only very limited success.

The best function has been to keep myself internally ``up-to-date'' with
my thoughts; I did not lose much knowledge.

I can give a demonstration of something that I gained:

I discovered at one point that large scale social ethics are too
complex to calculate, and that social ideologies are necessarily
unscientific.

It something that I had discovered before in my life, but somehow had
``lost''. This time, when I learned it, it was clearly there in my
notebooks. Any time a thought came that ignored that conclusion, it
had to be placed into the integrated structure. It would eventually
come to the test of the principle that large scale social ethics are
too complex to calculate, and thus died, or was cast as a member of a
particular social ideology; Either one I subscribed to or didn't. But
I recognized over time that even though particular social ideologies
are more in line with my values than others, that the whole thing is
so complex that I cannot say with a degree of certainty which is
better than another.

That progress was not lost, as it was before.

Now, I am not using the binders. It has been a few weeks since I have
stopped. I am not testing everything against the binders: Is
everything lost?

I don't believe so.

(We're getting into ideas about thinking, so we'll just continue into
it.)

I mentioned that the binders have a ``freezing'' affect on the
mind. That's true -- it can stagnate your mental growth. (Perhaps there
is a way around this, by computerizing parts of the system, as I
describe below, in order to make it more fluid.) But by the same
token, it is also reinforcing.

I am no master of psychology or metaphysics, by any stretch. But, I
have found a model of learning that is agreeable to me.

Michael Ende wrote that we learn things and forget them, learn things
and forget them. His idea was that these build layers in our mind
that, though we can't see them, are still there, and support us. Thus,
they still benefit us.

Performing the notebook system for a period of time seems to be like
organizing your conscious thoughts, and etching a foundation in your
mind. Whether it is good or bad -- it can be remade later.

Since we're here, I have a couple other ideas about thinking that I
want to explore:

\begin{itemize}
\item Repeating thoughts, incidental thought, and active thinking.
\item Is our thinking process structured?
\end{itemize}

I'll start with:

Is our thinking process structured? Just at a glance, it seems rather
chaotic: There's a thought here, a thought there, and another
one. However, it once occurred to me to find a pattern in my thinking.

I found one, and built a collection of markup icons to label the
different pieces. I am not alone in this idea; Robert Horn did
something similar, but he did it to technical documentation, rather
than to an individual writing his thoughts, thinking a problem
through.\footnote{See: ``What Kinds of Writing Have a Future?''
  by Robert Horn
% this link is dead: http://www.stanford.edu/~rhorn/images/spchWhat\%20Kinds\%20of\%20Writing.pdf
  \url{https://web.stanford.edu/~rhorn/a/topic/stwrtng_infomap/spchWhatKindsofWriting.pdf}
}

I have not seen his full classification system for elements of
explanation, but I can describe my own loose structure for elements
involved in my thought process:

It includes:

\begin{itemize}
\item Incentive
\item Problem (to solve)
\item Question
\item Goal

\item Starting Point, Reflection, Contemplation, Brainstorm
\item Collapse, Organize, Order

\item Articulation Request  -- (``Articulate this hidden thought'')
\item Analysis/Disetion Request  -- (``Analyze this situation into parts'')

\item Map, Model, Vision, Structure
\item Rule, Principle, Law
\item Name, Definition
\item Exclusion, Boundary

\item See Also, Connection
\item Quotation

\item Hazard, Warning
\item Rebound, Redirection, Leads To

\item Research Request

\item List
\item Point

\item Concluding Principle, Therefor
\item Possible Action
\item Possible Future
\item Concluding Question
\item Concluding Goal
\item Concluding Problem

\item Flag
\item Double-Check Request
\end{itemize}

These are the primitive elements of thinking that I have found so far,
and I have found a few common aggregates as well.

This is beyond the domain of a book on notebook systems, but I thought
it was related enough that I should include the idea.

I use the above icons (graphics not included, sadly) to identify the
major elements of my thinking, and how they piece together. They
appear most commonly in POI and on speed lists (my ``psy'' column).


\theme{Next: There are different ``thought moods''.}

WARNING: This is not something that I have tracked, and not something
that I have given any organized thought to. So what follows is off the
cuff, and may well be wrong.

Most of our time, we're either doing work and focusing on the very
next task to complete (if we are fortunate), or we're repeating
thoughts about just whatever. Those thoughts are largely
uninteresting, rehashing whatever we already know in a particular
domain. After a while, however, we can come across a ``solution'', or
glimpse a new vision, or get something that is truly useful. We
stumble across a puzzle piece.

\emph{Somehow}, I don't understand quite how, it seems that the mind gives
you solutions if it knows that they will be implemented. If you are
habitually throwing out good ideas, over time, it seems to me, your
mind stops giving you solutions. But by keeping your thoughts in a
trusted system, the mind gets happy about solving problems, and gives
you more pieces.

This is what I have observed; Now that I am not keeping my notes, and
have been tossing good ideas left and right, they appear to be
dissipating.

I don't worry about this; I know that I can pick up my notebooks
whenever I like, and good thoughts will start surfacing again.

There's also a certain mental mobility possible when you aren't
keeping notes. While you can ``lose ground'', you can also ``leap high''
in ways that the note keeping process makes difficult.

There frequently come points in the day, where, all of a sudden, a
million thoughts come to you at once. I don't think that it's that
your brain suddenly gets hyper; Rather, I think it's that one idea
triggers another idea, and then that another, and the frequently
trigger ideas across subjects -- across great mental distances even, and
there's this sort of chain reaction. Recording it is quite a wonderful
experience, unless you don't happen to have a piece of paper around,
in which case it's a miserable experience as you start pegging
thoughts and compacting them into itty-bitty syllables.

So, we've talked about ``day-to-day repeating'', about how good ideas
come spontaneously, and about when we are concentrating (the icons of
thought fit in here.)

Finally, one of my favorite topics in theory:

\theme{HOW DO NOTEBOOKS WORK ON AN ABSTRACT LEVEL?}

As opposed to the previous section, this is something I have thought
out well, and carefully integrated.

We have the subject matter -- information, knowledge, and wisdom.
And then we have the process of handling all that, with the aid of the
notekeeping system.

First, the subject matter:
\begin{itemize}
\item Information
\item Knowledge
\item Wisdom
\end{itemize}

Information is little pieces of idea that you have. Your speed lists
will probably consist mostly of information.

After you collect a bunch of information, you'll detect patterns and
relationships. As you integrate ideas together, it becomes far greater
than the sum of the individual pieces -- it becomes \emph{knowledge}.

What is Wisdom? I define it as \emph{knowledge} that has been \emph{integrated
with your life.} Or ``integrated with your living systems.''

That is: IF your knowledge makes a difference in how you life, THEN it
is not just knowledge, but Wisdom as well.

The old Dungeons and Dragons handbook never had such a description, but
suggested by analogy. To paraphrase: "Knowledge is knowing about
rain. Wisdom is knowing to come in when it rains." By my criteria for
Wisdom, it explains the D and D analogy perfectly. The person who knows
about rain is thinking about rain as collections of integrated
information, but the person who is wise is actually DOING something
based on that knowledge.

Now, the notebook system is a system for manufacture Knowledge and
Wisdom!

Here's a diagram of how it works:

\begin{verbatim}
    --
    * .          --.     ___*___
  L *  .          /|    /       \
    *   `-.      //|   /  --*--  \
  I -.     `-__ //     | /     \ |
    * `--.     */    \ |/---*---\|
  F *     `----*------\*'        OO------->
    *     _____*------/*`___*___'OO    `
  E -----'     *\    / |\       /|      |
    *      __.~ \\     | \     / |      |
    *     /      \\|   \  --*-- //    (   )
    -----'        \|    \       /     (   )
                 --`     ---*---        |
               ^       ^                |
   <-----------^-------^---------------'
\end{verbatim}

How's that! Do you like it? Isn't that great?

What, you want me to explain it to you? Okay. {:)}=

Lets start at the left:

That's your life. Your life emits many many thoughts through your
mind. The thoughts are the little stars: *'s. However, most of the
thoughts are uninteresting.

The first step is to collect the good thoughts together. You put them
onto your pan-subject speeds and your speed lists.

Then, through various mechanisms (such as the subject speed lists, and
the ``hint'', maps, and just looking over your speed lists and
recognizing), your thoughts are placed into ``locals'', places where
related thoughts are found.

That's represented by the three major arrows, pulling the collected
thoughts into specific locations.

After a few thoughts are collected in the same place, you think about
them. That's the ``opening part'' of the diagram: There are two little
stars at the end of the middle big arrow, and those thoughts are then
reflected upon, likely in a POI entry.

Whoah! Suddenly your two thoughts became a gigantic number of
thoughts!

This is a time for \emph{EXPANSION}. You're brainstorming, the thoughts are
flying. You don't necessarily want to \emph{limit} your thinking right now:
This is time for explosive hot thought-on-thought action.

Your job is to write down what these things make you think of. You
have ``reflect ons'', ``analyze this's'', lists to build, all sorts of
things. It blowing out.

After you reach a certain point, you've written up your major ideas.

Now, you can see them all, and you can start \emph{organizing} them.

This goes in the OPPOSITE direction: Instead of making ideas, you are
reducing them, by packing them together into a synthesis. Some will go
away, some will be compacted, some will be found to be identical
representations of the same thing, etc., etc.,.

Finally, you end up with a \emph{synythesis}: the square arrangement of
four O's on my diagram.

Congradulations! You have a conclusion, a synthesis!  Not necessarily-
it may be that you've created a model, too, or something
else. Regardless, you have a product.

It can be ``shipped'' -- arrow to the right.

But what happens now is that it feeds back into your system.

Perhaps, by building your synthesis, several previously existing Speed
thoughts are unimportant, since the POI does such a good job of
encapsulating them. You can put the POI\# on the map, take off the
S\#'s. If you like to ``preserve'' the S\#'s, you can put them in their
place in the final POI entry where they end up.

There are many things you can do.

The resulting thoughts \emph{also} affect your thought-collection
system. Perhaps new subjects are suggested, or new constallations
within existing subjects. Perhaps new maps need to be made.

And finally, your new synthesis affects your life. This is akin to
Michael Ende's ``learning and forgetting.'' The thoughts coming out of
your head will now be different, by the new addition.

This is my understanding of the process.

There is room for greater understanding:
\begin{itemize}
\item Identify the types of thoughts that come through. As I mentioned
above -- I've attempted to identify some basic elements of problem
solving.
\item Identify the types of synthesis, and the impact they have on other
sections.
\end{itemize}

I cannot shake the feeling that this model has implications for the AI
community. If the types of thought and synthesis' were understood
better, it may be mechanically implementable.

There is still the ``life'' region of the map -- where the ideas come from
in the first place. I think that by some analysis of personal
psychology, that could be understood better, and abstracted for
computers. Something to do with placing attention by strategic
consideration and the mere position of the organism.

I have now described my system to my satisfaction.

The next two chapters are on the Question of Computers, and how to get
started.

\chapter{The Question of Computers}

Ah, my favorite chapter. \verb| {:)}= |

This is where the future is. Not just the DISTANT future -- but the near
future as well.

But face it: Computers suck at giving us the detailed and speedy
drawing capabilities we need to think fluidly on paper. It just takes
way too long to diddle with selecting text, changing font size and
colors, quality, and lets not even talk about drawing, even IF you are
armed with a WACOM.

For anything that requires small levels of drawing power, you're
pretty much going to have to ignore the computer.

HOWEVER: THE MOST TEDIUS PARTS OF THIS NOTEBOOK SYSTEM HAVE \emph{NOTHING}
TO DO WITH DRAWING!

What are the most tedius parts of the system?

\begin{itemize}
\item Maintaining TOC's.
\item Transcribing Speeds
\item Redrawing Maps
\end{itemize}

Of those three, the first two take up about 90\% of the maintenance
time, and are major pains in the ass!

Fortunately, COMPUTERS EXCELL AT THOSE THINGS!

Maps is a little more difficult, because it requires \emph{some} drawing
power. However, I believe that map construction \emph{too} can be safely
taken over by the computer.

HERE'S MY VISION:

During the day, you use the ultra-convenient (but in the present pure
paper notebook system, \emph{dangerous}) pan-subject speeds. You don't
number them, you just keep them.

Then, at night, you go to your computer, and, with your mad 133t
typing skills, \emph{quickly} transcribe the pan-subject speeds (which are
90\% text, 9\% icon, and an occasional image) into ASCII (sorry, there
went our icons and images...). You run the magic program, and bam!

It spits out updated pages over your printer.

You take out the pages, hole punch 'em, and put them into your
binder. Yah!

The speed lists that come out are ULTRA-DENSE, but in a READABLE size
(that you chose). With columns, you're fitting \emph{100 speeds} to the
page!

And when you want to split a subject into pieces, you just tell the
computer where you want your speeds to go:

\begin{verbatim}
   MTK
   S
   13 Eth
   19 Spt
   43 Eth
   44 Eth
   49 Imgn
   51 Imgn
   62 Eth
   93 Eth
   121 Spt
\end{verbatim}

That could mean, "In Mental Technique, regarding the Speed thoughts,
send \#13 to Ethics, \#19 to Spirit, \#43 to Ethics, \#44 to Ethics, \#49
to Imagination, \#51 to Imagination, \#62 to Ethics, \#93 to Ethics, \#121
to Spirit..."

\emph{So much Faster!}

Right now, doing that work will require at least 10, maybe 20 minutes,
as you write the speed thoughts allllll over again.

Such speed! Such automation!

And the best part is:

This isn't complicated AT ALL!

A programmer of 1 YEAR could do this!

Do it in C\# or Python. Hell, you can use ``using Word;'' and command
Word to write the text in, and perform the printout.

I've thought a little about the architecture of such a program, I'll
describe it in just a moment, in a ``Geek Block''. If you are not a
programmer, just skip it when we get there. If you ARE a programmer,
and you think you might want to implement this, READ THAT SECTION.

And it's NOT just about the Speed Lists-

Put your POI titles in there too!

And put your REFs in there too!

If there's a TOC for it, put it in there!

It can print out high quality laser printer ultra-info-density tables
for us. Just imagine having all of that power on a single page, and so
\emph{quickly}!

Let me tell you something:

I LOVE my notebook system. You already know that. And I wouldn't
switch my paper notebook system for my old computer systems, ever.

But the \emph{hybrid} I am describing -- the hybrid will tap the incredible
power that the computer system has. What's that power?

RAW SPEED. It's Low on Quality, (and sadly, for POI's and RS's and REF
entries, that's the real killer, despite the speed.) but it's FAST.

I used to capture HUNDREDS of thoughts per day, EASILY, with my weird
file. It's just so damn easy to control-tab into a text buffer and
start yakking away. Next thing you know, you've control-tabbed back
and you didn't even notice you were gone. You DIDN'T have to fish out
a page. You DIDN'T have to open a binder. You DIDN't have to pick up a
pen. You just started yacking. And it was stored. You didn't even
think about it afterwards.

So many thoughts.

The problem was that you couldn't make diagrams and stuff, and there
was no way to map.

But we have an interface to those capabilities, namely, the Printer.

So I have high hopes for THE HYBRID.

But WAIT! There's MORE!

After that -- After we have this system going, which will get rid of
just ENORMOUS quantities of the LABOR of the systemm, and bring in SO
many missed thoughts-

It gets even better!

After this is going on -- We can build something cooler.

We've already taken the TOC's and Speeds of our hands. Next to go is:

The MOCs!

...(breeze and tumbleweed)....

The MOCs? But aren't they ... Graphical?

Doesn't that mean... Graphical input?
Isn't that what the computer \emph{can't} do well?

Well, yes, that's true.

BUT:

Our maps using ONLY A TINY LITTLE BIT OF GRAPHICS.

That is, we are drawing LINES, and changing the signs of TEXT, and
using a FEW STANDARD ICONS.

With some UI work, I believe strongly that we can make this program,
and make it easy to use to boot.

It'll be easier to use than Dia or Visio. How can that be? Because we
are performing a LIMITED set of STANDARD operations on our maps.

I believe those operations are VERY finite, and can be mapped over key
combinations very easily. It shouldn't be hard.

AND IT CAN BE MADE TO INTEROPERATE WITH THE PRECEEDING SYSTEM -- that
is, they system that just maintains TOCs and Speed Lists.

Imagine this:

You open up the ``mapping'' program, and are faced with the GSMOC.

You see the GSMOC, and in the distance, in the distance are floating
the ``unplaced'' ideas, in a long list. But you aren't concerned about
that right now; Your strategy tabs are reminding you that it's really
important that you are focusing on Global Knowledge Infrastructure
right now. ``Ah, yes.''

You click on GKI and find yourself within the next map, the GKI. You
see the front page map, and you see a bunch of unplaced thoughts in a
list.

You start clicking and dragging the thoughts through the map. You can
even doubly or tripply place them, if you have multiple views on the
same subject, and naturally thoughts apply in multiple places.

You can build wormholes in empty space pointing to other worm
holes. You can access a clipart library, and place clipart in.

You can hide or show text for speeds. If you roll your mouse over an
iconified speed, the entire text of the speed appears in a window.
You can click the speed to switch between ``titled'' and ``iconified''
form.

With this tool, you never have to redraw a map, because the blemishes
have built up.

And when you are done with a session, the printer will print out the
updated pages, so that you can keep the physical notebook current.

If you are carrying a subject with you, you can then access your
beautiful maps, and show them to people, and quickly use them in the
heat of a conversation.

Now HOW'S THAT?!?

I think that's awesome!

I want to do it!

I want it!

The first program is EASY. The second one is a LITTLE MORE
DIFFICULT. A UI Master could really help build the 2nd.

But the first: A First Year Programmer could do it, ALONE, without too
much trouble.

So.

If this was done, the notes system may take so little labor, that we
might not even be paralyzed by it.

Or not. After all, I don't necessarily know that it's the LABOR of the
system that induces paralysis. It may very well be that "map
switching" is still cumbersome, and induces mental paralysis, or that
the mere fact of \emph{taking the notes} is interruptive enough to induce
mental paralysis.

I don't know.

But, I'd like to DO this improved system, and we can FIND OUT for
ourselves, first hand.

\pause
% ---
\hrule

Reflect.

\hrule
% ---
\pause

Let me know if you start a project to implement this system.

I may start implementing it myself, I may not. Who knows what I'll do
after I finish this version of this book. I may revise it. I may go on
to work on the programmed system described here. Or I may go on to
work on organizing a Visual Lanuage conference here in Seattle.

We'll see what happens.

Finally, (well, not ``finally'' -- there's still the ``GEEK BLOCK'' to
write,) I want to briefly describe this system, as applied in the far
future -- maybe a decade or two away.

In this far future, we have ``magic paper.'' We have paper that we can
roll out, and it has a computer system in it, and we can write on it
\emph{as if} it were paper, and it would work just like paper.

So there would be no difference between physical paper and this magic
paper, except that the magic paper is better.

What would the notebook system be like?

Well, it's a little hard to say. I hope the hybrid system, once
completed, and then once \emph{evolved} beyond what I can see right now,
will have more to say about that.

But basically, the binders could be eliminated completely.

Here's my wish list for the future system.

\begin{itemize}
\item Multi-categorization.
\item Scroll wheels (or something like it) on the side.
\item \emph{Infinite} canvas, and a zoom dial.
\item Multicolor/font selections on the pen.
\item Text OCR to dimensions, font, quality, italics and bold (style)
\item square $\leftrightarrow$ sloppy niceness tools.
\item Icon programmability.
\end{itemize}

In turn:

Multi-categorization:
I am ASTONISHED by how many ``MODERN'' notebook systems allow you to
write your thoughts, and then...

...throw them into a category tree.

Really. You should be able to throw them into zillions of category
trees and maps.. And not by a ``directory tree'' mouse navigation
system, but with speedy tab-completions and smart DWIM (to handle
typos and stuff). And you should be able to navigate to all of the
maps that the thought participates in \emph{quickly}.

Because UI mechanisms are in their infancy these days, it's
understandable that you can't throw thoughts into maps. But there's
poor excuse for not being able to throw them into multiple tree
structures. It's just so EASY to implement, and yet ALL of these
``award winning'' computer notebook systems don't let you do it.

I mean, I've \emph{done} it. I made the weird file. And a casual Internet
browser (and later friend) made a simple program that interpreted my
weird fine, and drew up the category trees. It was neat: I downloaded
his program, and could navigate my thoughts! Wow! Cool! So it's
definitely not complex.

So dammnit; If they put in all these cool features in the next decade
or two, the least they can do is allow very simple
multi-categorization..!

Scroll Wheels:

Call me a luddite, but I do believe there are times when physical
things ARE better than just drawings of them on a computer screen.

I'd like a nice physical scroll wheel on the left side and on the
bottom side (or whatever -- just pick two perpendicular sides) that I
can scroll to move about the screen.

Maybe not my brightest idea. Maybe it'd be better to put the "hand
grab" button on the pen itself.

INFINITE Canvas and Zoom Dial:

Okay -- again -- maybe the zoom \emph{dial} isn't the best idea.

BUT THE INFINITE CANVAS STAYS!

If you can pan and zoom easily, then there is no reason to limit the
size of a paper.

Now, slightly more radical:

I'd like to be able to zoom in and out \emph{infinitely}.

Yeah.

I'd like to be able to fit an entire world in the dot of a lower case
``i''.

Or I'd like to be able to zoom in, and the GIGANTIC words fade away,
and are replaced with the zoom in of the contents beneath.

I really want this, because I'd like to pack LOTS of detail. I'd like
to have a big 2d diagram that is an overview of a system, and then
zoom in on a portion, and see THAT system in it's entirety. You should
be able to say, ``Visible within depth range 1000-100'', and then
content smoothly alphas into oblivion outside of that zoom range.

So there.

The color wheel may have been dumb, but THAT wasn't.

* Multicolor/font selections on the pen.

I don't know why the Wacom pen doesn't have any buttons or a selector on it. Wh
o
wants to move the the pen over to the side of the screen to change
color? Just flip a switch. You can switch between 4 settings, perhaps:
Red, Blue, Green, and Black. (Whoah! So original!)

* Text OCR to dimensions, font, quality, italics and bold (style)

The OCR needs to discriminate the DIMENSIONS of text that it is
``nice-ifying'', and try to match font as well. It should recognize
italics. If you are pressing hard on the pen, it should be bold.

If possible, it would be neat if it could simulate \emph{distortion} -- if
you write sloppily, the text would come OUT a little sloppily.

People assume that we want all text to look nice.
In reality: WE DON'T! Most don't know it consciously, but when you are
thinking about something seriously, YOU WRITE BETTER. And when you are
thinking offhand, YOU WRITE SLOPPILY. And when you look at the page,
YOU CAN DISCRIMINATE GOOD FROM BAD by the QUALITY OF THE TEXT.

THAT IS NOT AN ABILITY WE WANT TO LOSE!

Thus, it would be good if the OCR system can detect "degree of
sloppiness", and then simulate the sloppiness after it's done it's OCR
thing.

Of course, there is going to be text sooo sloppy, (or context sooo
unsure), that the OCR system can do nothing but leave the input as it
was. Se la vi.

Sometimes I can't read my own handwriting. How could OCR? Se la vi.

* square $\leftrightarrow$ sloppy niceness tools.

Finally, I want a setting on the pen for when you are drawing. On the
pen, you flip the switch, (or maybe it's on the paper, whatever), and
now the system knows to ``nicify'' what you are drawing.

You draw a sloppy square, it renders a square square. Sloppy circle to
circle circle. Sloppy oval to oval oval. Sloppy poly to nice
poly. Nice nice nice. Even sloppy text to nice text. (It would turn
off the ``distortion detection''.)

That helps you build maps and stuff.

* Icon Programability.

As our visual language grows and grows, our references to icons and
clip art will likely grow and grow.

You will need to be able to say, "When I draw a sloppy circle with a
sloppy arrow pointing roughly to the right inside of the sloppy
circle, I want you to replace it with this nice pre-established green
pristine circle with a precice standard arrow pointing exactly to the
right inside of the green pristine circle."

Or, ``If I draw the kanji for a man in here'' (-that's just two subtle
strokes-), "I want an nice icon of a man to appear here in it's place,
sized to the dimensions of the kanji, and line balanced evenly with
the rest of the nice text on this page."

Okay. So.
You've seen my laundry list.
That's the far future. With luck, it's within the next 2 decades.

When these days arrive -- what great things we can do! What coordination
we will be capable of! What collaboration!

What \emph{SELF-ORGANIZATION} we will be capable of when we can FINALLY
COMMUNICATE CLEARLY and UNIVERSALLY.

We will be able to THINK with NEW CLARITY, and GAIN from the THINKING
OF OTHERS.

Finally, at the end of the chapter, is the Geek Block.

GEEK BLOCK BEGINS

A possible architecture for TODAY.

I would code it in C\# or Python.

\begin{verbatim}
               system resources
                 (printer)
                    |             TXT XML YAML SQL
     by file        |           ___/___/___/____/
         \          |          /
          INPUT----()---Database Access--+-- Lists
         /         |                     \-- Entrees
    by GUI/CLI     |                     \-- Subjects/IDs
                   |                     \-- Output Configuration
                 Output...
                 |    |                      .
          changed    duplicates
           pages |    |                   '
                 |    |                .
         --------------------       .
          Page Output System  -  -
           |           |
         .doc        .txt
\end{verbatim}


You can input by file (changes), or by GUI/CLI (invocation,
diognostics, output requests).


INPUT

Change requests:
\begin{itemize}
\item add entries
\item change/update/annotate entries
\item move entries
\item mark to archive
\item new subjects
\end{itemize}

Output requests:
\begin{itemize}
\item request duplicates of pages
\item request all pages updated since last request
\end{itemize}

Invoaction:
\begin{itemize}
\item update from the latest request, and return a receipt.
\end{itemize}

Diagnostics:
\begin{itemize}
\item debug information
\item manual vision/manipulation of internal databases
\item database statistics
\end{itemize}


\theme{DATABASE}

Abstract the database system.
May want to use pure interfaces, or a bridge pattern.

Can use XML, YAML, SQL, TXT file, or whatever, to be the form of the
database.

I myself would go with YAML.

The database keeps track of your lists (TOCs, speed lists), the
entrees in the lists (Speeds, TOC entries; their contents, hints, and
titles), the subjects and their abbreviations, and the present output
configuration (txt? doc? rtf? font? size? color? blahblahblah?).


\theme{OUTPUT SYSTEM}

START SIMPLE. While you could do a lot of cool tricky output things,
I'd start dirt simple. For a word.doc output, just say "2 columns,
point 8 text, tabbed, fit 70 to the page."

Don't worry about making optimal use of space for now -- that gets
tricky when you start updating speeds -- making some longer -- and then
text rolls over onto the next page, and then \emph{that} page might roll
over, and you have to keep track of which pages need reprinting and
which don't -- yadda yadda yadda.

Just keep it dirt simple for this first trial.


And that's about it!

It's really a VERY simple program!

And it helps you out SO MUCH!

GEEK BLOCK ENDS


\chapter{Getting Started}
Let's face it: This system is big. This system is complex. This system
is: Formidable.

I originally didn't have this chapter in mind, but after talking with
my first adopter, found myself writing additional information about
how to start. So here it is.

The most important thing to do is:
START SMALL.

Here's the plan:

\begin{enumerate}
\item Make a pan-subjects speed list.
\item Identify Patterns. Make 1/2 Subjects. Make a GSMOC.
\item Create your first subject.
\item Onward and Upwards!
\end{enumerate}

\point 1) MAKE A PAN-SUBJECTS SPEED LIST.

You can either make one on your own, or, (and I highly recommend
this:), go to \url{http://speakeasy.org/~lion/nb}, and download the Word
.doc pan-subjects speed template.

Make/Print-out a few of them.

Then \emph{just start taking notes}.

As interesting ideas occur to you, \emph{just start taking notes}.

Don't worry about ``subject''. Hell, if you are uncomfortable with it,
don't even worry about ``hint''.

You're new to this. Cut yourself some slack!

Just do this for a while. Maybe a day or two, until you get about
20-50 speed thoughts, whatever you are comfortable with.


\point 2) IDENTIFY PATTERNS.  MAKE 1/2 SUBJECTS.  MAKE A GSMOC.

Go over your pan-subject speeds. Locate patterns-
Ask yourself:

\begin{itemize}
\item What kinds of things am I thinking about?
\item What do I care about?
\item What are some important studies among these thoughts?
\end{itemize}

From these, you should get some ideas for your first subjects.

\emph{DON'T WORRY} if you find that among 20 speed thoughts, you have 15
 different subjects..!

Don't worry AT ALL about the scatter!

What you will find, is that, as you develop your subjects, many will
*attract* your thoughts to them, and become *focuses*. It's sort of
like a gravitational mass: You start collecting some thoughts, and
then they start collecting more thoughts -- and then...

And don't worry, we're just putting these ideas into \emph{half}
subjects. That means we aren't doing tabs and what not here. We're
just maintaining speed lists for the subjects.

So, you're finding patterns.

Then, either make your own subject speed lists, or print some out
from the web page. (I recommend the print-outs.)

Don't worry -- they're free. I'm not charging for this. (Because I'm
craaaAAaazy Al!) Just get 'em and print 'em out.

Transcribe from the pan-subject speeds to the half subjects.

At this point, we're doing good, but we haven't really transcended the
``bucket'' level of notebook systems.

(Remember what I said at the beginning of this book: 99\% of the
notebook systems out there today are just the ``bucket'' system, or the
``diary'' system, or some basic combination of the two.)

Now, you should get a binder at this point. Put the subject speeds
into the binder. Put some blank template pages (blank pan-subject
speed pages and single subject speed pages) into the binder, and put
in your subjet speeds, alphabetically.

You'll also make GSMOCv1 -- the first version of your GSMOC. Remember,
the first version will contain your first GSR on it as well. No need
to alphabetise, the list is so small.

Now do this for a while.

Do this untill one of your subjects reaches, say, 10-20 speeds.

\point 3) CREATE YOUR FIRST SUBJECT.

Since you have only one, you can just store it in your carry-about
binder, at this point.

Make a SMOC, and start practicing the mapping process.

Make your first map of your thoughts -- you may want to reread the
walkthrough in chapter 3 of making a map, *BUT THIS TIME WITH YOUR
VERY OWN THOUGHTS!*

Now you'll start to see the POWER of this system.

The MAP is where it all comes together -- \emph{LITERALLY}!

\point 4) ONWARD AND UPWARD!

DO THIS AGAIN!

Make more subjects!

And start writing POIs!

Start getting a little strategic.

You'll also start having problems -- REREAD THIS BOOK!

You'll start seeing \emph{why} things are set up the way they are. You'll
start having dillemas in placing thoughts -- reread my suggested rules
for subject splitting, and see if they make sense to you.

You are just starting to learn this system! When you've DONE it
yourself, it's a totally different thing -- and VERY exciting!

You'll start to see the need for P and P, and you'll start having tensions
between subjects. REFLECT THEM ON THE GSMOC.

Keep building new half subjects, and then promoting them to full
subjects.

You'll find that time is limited -- NOW USE THOSE STICKY TABS!

Arrange to go on your shopping trip, if you haven't already, and get
the pieces that can help move you along.

By this point, you shouldn't need much help.

You'll see how the ideas at work here fit together, and your paper
intake will increase. You'll start integrating your study and research
into the system, and lining it up on the maps.

You'll probably start to see little things that I knew but forgot to
put in here, and you'll start to see what I mean by ``immobilizing''.

And yet, if you are anything like me, you'll perceive value in this,
and continue on.

TIME WILL PASS.

I predict:

The more you think and value and reflect, the more you will be
COMPELLED TO ACTION. You will gravitate towards something, or some
things, and you'll see things in your head that you've never seen
before. Things that are so real  -- you can touch them.

You may question the relative significance of the physical world,
compared to the realm of ideas. Things take on new dimensions.

And you are COMPELLED TO \emph{ACT} in NEW WAYS.

EVENTUALLY:

You will NEED TO STOP.

And you will do so.

And you will ACT.

And your acts will be organized.


I am just speculating now -- I am the only person who has ever practiced
the notebook system I am describing.

But let me tell you-

I HAVE NEVER COMPLETED A MAJOR PROJECT.

I mean, sure, there have been some minor projects. And I have
completed major WORK projects, that I was paid to complete.

But I have very rarely copmleted MAJOR projects. Not that I can
remember.

WELL, I'm IN THE LAST LINES OF THIS BOOK.

Granted, the book is rough.

Granted, the book is gritty.

But GOD DAMN IT, if I'm not writing on the 6,831st line..!

I, who have never written a complete book, have finished this thing.

Well, maybe I'm overthinking things, but this is not common in my
life. I know my weaknesses, I accept them.

But I have a theory: That the Notebook system is somehow a focusing
system of not only THOUGHTS, but ALSO of some sort of MENTAL
ENERGY. And that it BUILDS, and BUILDS, and BUILDS as you THINK,
THINK, and THINK.

And that when you are DONE THINKING, you are COMPELLED TO ACT.

And the action WORKS.

Again: I HAVE NEVER WRITTEN A BOOK.

Now: To be honest, it's a shabby book. I know. I'm not an experienced
writer. But here it is..! It's shabby, but it's a book! It's practice!

And I am immensly proud of this.

I suspect it is due to the notebook system working.

So:

Let me know how it works for you.

I am tired now. You can tell in my writing. This has been an
exhausting effort, this whole last week and a day. But it's over now.

I feel like I've just lectured for a whole week straight. With work in
between. So please forgive my tiredness.

But I want to say:

LET ME KNOW HOW IT GOES!

\pause
EMAIL me: \url{lion@speakeasy.org}.
Tell me, personally, how it goes.

\pause
And if you implement the computer system, or want to, or want to help,
or want to see how it's going, go to my web page:

\pause
\url{http://speakeasy.org/~lion/nb/}

\pause
...and check to see how it's going. And if it's NOT, maybe that's
probably because your the first to try: Email me at
lion@speakeasy.org, and let me know you want to start, and I'll link
people to you.

You can work together or seperate, however you like.

And feel free to call me, or write to me.

I LOVE TO HEAR FROM YOU!

\pause
\begin{flushleft}
  Lion Kimbro\\
  12038 31st Ave NE \#201\\
  Seattle, WA 98125\\
  \medskip

  206.440.0247  $\leftarrow$ YES, you can CALL! Just say "I read your notebooks
                   book."

\end{flushleft}
\pause
Call at whatever crazy hour.
Let me know.

Maybe you live in India or China or something. But if it isn't too
expensive, call and let me know. I'd love to hear from you.

And if you have any questions, or thoughts, or anything, feel free to
let me know what they are.

If you want to help with the web page, or make a discussion board, or
a wiki, or SOMETHING, let me know.

Oh! If you want to donate money! You can do that to. My paypal thingy
is also under ``lion@speakeasy.org''.

That'd be great. Then Kitty (my girlfriend) might also feel that the
notebook system was worth something. <laugh>

Okay.

Take care.

With Love,
  Lion \verb| {:)}= |

\appendix

\chapter{Acronyms}
\begin{description}

\item[ACEPOS] Absolute Cosmic Eternal Perfect Ontological
Structure. Something we try to avoid in this notebook system. The
structure maps YOUR brain, not the universe. Don't even try. Madness
that way lies. Leave it to the standards comittees -- it's not your job
here. (Unless you are using your notebooks to engineer a standard. In
which case, you already know about the madness.) A term I made up on
the fly, while writing this. I kind of like it now. ``ACEPOS.'' Heh! At
least it's not as cheesy as ``Universal Cosmic Habit Force.''

\item[ACTS] Act-Communication-Thought System. My own personal hybrid
between GTD, and the notebook system described in this book. I do not
describe the interface between the two; It isn't all that complicated,
and is of little interest to most readers of this book, I
suspect. Keep in mind that the notebook system here is anti-ethical to
GTD: GTD promotes action, this system DE-motes action. If you DO
maintain GTD as you perform this system, you will have to realize that
you are going to VASTLY reign in the GTD. Your action pages will
dramatically dwindle. On the other hand, your Someday/Maybe's will
baloon out vastly -- no small wonder than that as part of the interface,
I've absorbed the someday/maybes into the subjects' speed lists. If
people write to me asking for elaboration on the ACTS notebook-GTD
interface, I'll write about it. But for the most part: GTD just
whithers in the presence of this system. See ``The Question of
Computers'' for some software ideas that, if implemented, may open up
the possibility of having both doors open at the same time. I think my
most common use of the GTD system was for looking up books at
libraries, getting to web pages to look at, doing my chores, keeping
dates, and as a mechanism to remember when to refill my blank papers
and what not.

\item[A/S] Abbreviations/Shorthand. Where you will keep your shorthand
notes and abbreviations so that you can write quickly, and yet still
be able to figure out what you were talking about a few years
later. Each section has it's own A/S, and there is a global A/S across
all subjects as well. Global A/S should have a page for names, and
some hash pages for common abbreviations.

\item[FF] Final Fantasy. I wrote ``This is the way!''. I was quoting ``Final
Fantasy Tactics''. You perform ``jobs'', and your characters say things
like ``I had a feeling...'' and ``This is the way!'' Never mind. SOOOo
totally not important, save in some sort of strange schizophrenic
holistic universe way. In which case, THIS ENTRY is the CRUX of this
entire book. ``This is the way!'' You be the judge.

\item[GKI] Global Knowledge Infrastructure. Like PFT, something that
happens to be a subject of mine, but I like it, and I used the acronym
in the book, so I get to advertise the concept here. Heheheh..! I'm
tricky, aren't I? GKI is the study of who knows what, and where. It's
the study of how fields grow and fall. It looks at things like the CIA
and Corporate knowledge bases and public understanding and
universities and says ``What does that mean? Who's got the information?
Who's got the knowledge? How is it spreading? Where does it come
from?'' Ideas such as Anarcho-Science, studying the motion from
University/Gov/Mil/Corp to the public sphere are interesting. How
Computer Science makes more progress in the publc and corporate sphere
than it does in the university system. Something to pay attention to,
especially if you are interested in a Democratic society. And know
too: That there are enormous fields, VAST fields, that are completely
untouched. Why? Because there is no profit in them. But there IS
profit to the public. It just has to study these things..! The
government's not going to do it. The corporations sure as hell aren't
going to do it. We have to do it OURSELVES. See ``PFT.''

\item[GMOC] same as ``GSMOC.''

\item[GSMOC] the Grand Subject MOC. This is a map of every subject, and the
central strategy point. Every subject that is not obsolete (archived)
appears on the GSMOC. Frequently runs multiple pages, with the front
page being a map onto the other pages.

\item[GSR] the Grand Subject Registry. A two page (or more) list of all of
your subjects. It comes right after the GSMOC. It lists all subjects-
even those that may not appear on the map. You use it to attach flags
and other metadata to subjects identified by NAME.

\item[GTD] Getting Things Done, by David Allen. Incredible book. Integrates
best ideas from time management in ways that no other book does. Most
books just give you a piece of wisdom here, a piece of wisdom
there. This book puts it all together for you. Not really about
notebooks, because he focuses on ACTION, not THOUGHT. The GTD system
and the system I describe are REALLY at ODDS with one
another. Regardless, I still found utility in his GTD system..! Read
it! (Lookup also: ``ACTS'')

\item[I] Index. An alphabet based index. (Can contain numbers and glyphs as
well.)

\item[M] Same as ``MOC''. NEVER same as ``GSMOC.''

\item[MOC] Map of Contents. A visual map, frequently multiple pages (though
not uncommon to have just 1 page, for little-reached subjects), of
some domain. Generally, this is either a SMOC, GSMOC, or a piece of
one of those two. But rarely, within the notebook system, independent
of those two. The function of a MOC is to integrate, after all, so
they tend to be all-encompasing (within a subject, or for the GSMOC.)
The first page of a series of MOC pages is generally a map of the
reminder of the pages, and how they fit together.

\item[MTK] Mental Techniques. Not related to notebooks, but like PFT and
GKI, ... you've read it before. So anyways. This is stuff like forming
theoretical architectures of thought and using them to think fast. And
memory techniques. And any other sort of mental gymnastics or study
thereof. Read ``The Memory Book''. Great book. Cheap. Far better than
any other memory course I've ever seen, far better than the multi-\$100
courses I've seen out there. I could tell you many humorous
stories. But I'm not going to. Not here, anyways.  Read the
Internet Memory HOW-TO; It has many of the Memory Book techniques in
it, if you are impatient. But the book is better.

\item[PFT] Public Field Technologies. This isn't related to notebooks -- it
just happens to be something that I've had as a subject in my
notebook. Regardless, I am evangelical about the topic, and I \emph{DID}
use the acronym in my book, so I'll take this space to advertise
it. There are SOME THINGS that are just so incredibly cool, and have
to do with benefiting the WHOLE PUBLIC -- \emph{if} people are interested in
them, and want it. Self-help books are sort of a well know thing, but
tools for mental techniques and for keeping notebooks are not. And
that's all in the personal arena. Communities can help themselves too!
Learn about Ithaca HOURS and stuff like that. There are ALL sorts of
things that people can do. Refer to my list in the middle of the book
and learn about those things. This whole book, about notebooks, is an
example of a Public Field Technology -- IF it actually turns out to be
good for something, and IF people actually band together and study
it. If PFT's ``work'', then we'll have a radically different -- and FAR
more Democratic (in the REAL sense of the word) world in the
future. The Internet has such revolutionary potential, and its great
to see it turn Kinetic. PFT's are divided (by me) into roughly two
spheres: The scale from personal to global, and the sphere of group
collaboration, communication. For example ``Visual-Verbal Language'' is
a PFT, but doesn't really have a good position from individual to
global integration. So there you go.

\item[PJ] Project. Pages connected to a particular project.

\item[POI] Point of Interest. Usually multiple pages devoted to one,
limited subject (delimited by the title.) Found within a
Subject. Bound by a TOC, generally. Placed on the SMOC.

\item[P\&P] Purpose \& Principles. This is a very special page in each
subject. It is usually just one page; I have never had more than 1
sheet per entire subject. It describes the BOUNDARIES OF THE SUBJECT,
and how to delegate, and in some cases even split, issues that
transcend the edge. It features ``inclusions'' and ``exclusions'', either
in text or by diagram. P\&P's, like Speeds, rarely live within the
subject's paper layout (following the tab for the subject). Speeds are
collected into the carry-about binder, joined with every other
speed. Similarly, P\&P for all subjects are collected into a special
tab (labeled ``P\&P'') at the beginning of the common-store binder. You
refer to the P\&P whenever there is a question about whether something
belongs to a subject, or not, and you are trying to decide where it
goes.

\item[REF] Short for ``Reference''. Reference items are annotations of
books. If you can, just write in your books and be done with it. But
if it's a library book, or a borrowed book, or a web page, you're
going to have to keep your co-writing and your side notes on your own
pages. And then you put them in the ``REF'' section. The ref section is
unusual in that you don't do normal page numbering, and you don't even
do normal TOC-ing. You do your page numbering to match the way the
text is organized. And you assign the reference number not by the
number following the last thing you wrote about, but by the number
connected to the reference on your references list. Your references
list is like your Speeds, but it captures references. When you
actually get around to reading a particular book, you take the number of
the reference to be the first number in the page sequence. It doesn't
matter if it's the second book you read in the subject -- if it is
number 7, you will number the pages starting with ``7'', not 2. See also ``RS''.

\item[RS] Short for ``Research''. Research is like reference, but usually
focused towards an end, and not tied to a particular source
reference. Don't feel like you need to have a REF articulation for
each reference you touch. Just make sure you note the REF\#'s in the
RS, so you can find it later, and dance from book to web page to
person to book, and keep a linear flow in your RS\#. Have conventional
TOC's over material, and, like everything else, appear in the SMOC.

\item[SMOC] Subject Map of Contents. A map of contents over an entire
Subject. Ideally, points to every single resource in the subject
domain that is not in an archival stage.

\item[TOC] Table of Contents. I frequently wish that these things would
just DIE. Sadly, our computer software infrastructure and UI notions
don't seem to understand how to deal with anything that isn't text
well. We don't even have tools to make MOCs quickly and well. So we
are stuck with these archaic information destroying beasts. Basically,
these are just an index of titles connecting them to page
numbers. Icky icky icky icky. Use a MOC if you can. But you frequently
need a TOC as well, if for no other reason than to see what to number
your next item. (There are ways around this -- just keep a record of
``next numbers''.) Still, it is comforting to have TOC's over your POI's
and Ref's and RS's.

\item[V] I frequently use as short for ``Version''. Generally it's a lower
case ``v''.

\item[VER] same as ``V''.
\end{description}

\chapter{CVS Information}
The information that follows is taken directly from the CVS database and 
is intended only to ease the translation of the book into \LaTeX.
\begin{verbatim}

$Id: book.tex,v 1.5 2003/08/06 05:01:02 abg Exp $

$Log: book.tex,v $
Revision 1.5  2003/08/06 05:01:02  abg

Added (and debugged) the glossary.  Final big piece is in place.
Found out that my regexp for replacing the double quotes with \LaTeX quotes
was too greedy.  (Read: It did what I said, not what I meant.)  Built a
fixer that corrected the problem.
Fixed a few misspellings.
Removed the global sans-serif font directive now that I've seen how nice
pdflatex works.

Revision 1.4  2003/07/26 03:36:49  abg

Changed the default font to sans-serif to make it look better in PDFs.

This is not the font that is desired for actual book printing, it will
be removed for the 'book version' as well as the 'one-side' documenting.

Revision 1.3  2003/07/26 02:58:06  abg

Added a CVS chapter to the book to aid in version tracking.


\end{verbatim}

\end{document}


